\documentclass{article}
\usepackage{graphicx} % Required for inserting images
\usepackage{amsmath}
\usepackage{esint}
\usepackage{amssymb}
\usepackage{amsmath}



\author{ @usercdp }
\date{June 2023}

\begin{document}
\subsubsection{Areas}
$\displaystyle\iint_A h(x,y) dx dy = \displaystyle\int_{x=a}^b\int_{y=c}^d f(x)g(y) dxdy=\left(\displaystyle\int_{x=a}^b f(x)dx\right)\left(\displaystyle\int_{y=c}^d g(y)dy\right)$

\subsubsection{Polar coordinates}
$x = r\,cos\theta$, $y=r\,sin\theta$, or inverting

\vspace{5pt}

\noindent$r=\sqrt{x^2+y^2}$, $\theta=arctan\frac{y}{x}$

\subsubsection{Spherical coordinates}
$r^2=x^2+y^2+z^2$

\vspace{5pt}

\noindent$x=r\,sin\theta\cos\phi$, $y=r\,sin\theta sin\phi$, $z=r\,cos\theta$

\subsubsection{Volumes}
$V = \displaystyle\iiint_V dx dy dz = \left(\displaystyle\int_{x=0}^L dx\right)\left(\displaystyle\int_{y=0}^\ell dy\right)\left(\displaystyle\int_{z=0}^h dz\right)=L\ell h$

\subsubsection{Total differentials}
$df(x,y,...)=\frac{\partial f}{\partial x}dx+\frac{\partial f}{\partial y}dy+...$

\subsubsection{Chain rule}
$\frac{dz}{dt}=\frac{\partial z}{\partial x} \frac{dx}{dt}+\frac{\partial z}{\partial y}\frac{dy}{df}$

\subsubsection{Directional derivative}
$\frac{d\phi(x,y,z)}{ds}=\overrightarrow{\nabla}\phi(x,y,z)\cdot\mathbf{u}$

\vspace{5pt}

\noindent$\overrightarrow{\nabla}\phi=grad\phi=\frac{\partial\phi}{\partial x}\mathbf{i}+\frac{\partial\phi}{\partial y}\mathbf{j}+\frac{\partial\phi}{\partial z}\mathbf{k}$

\textbf{The gradient of a scalar field is a vector and can be equivalently written as}$\vec{\nabla}\phi$.

\subsubsection{Gradient in polar coordinates}
$\nabla f=\mathbf{e}_r\frac{\partial f}{\partial r} + \mathbf{e}_\theta\frac{1}{r}\frac{\partial f}{\partial \theta}$

\subsubsection{Gradient in cylindrical coordinates}
$\nabla f=\mathbf{e}_r\frac{\partial f}{\partial r} + \mathbf{e}_\theta\frac{1}{r}\frac{\partial f}{\partial \theta} + \mathbf{e}_z\frac{\partial f}{\partial z}$

\subsubsection{Divergence}
$\mathbf{\overrightarrow{\nabla}\cdot \overrightarrow{V}} = \frac{\partial V_x}{\partial x} + \frac{\partial V_y}{\partial y} + \frac{\partial V_z}{\partial z}$

\subsubsection{Curl or rotational}
$\mathbf{\overrightarrow{\nabla} \times \overrightarrow{V}} = \left| \begin{array}{ccc}
    \mathbf{i} & \mathbf{j} & \mathbf{k} \\
    \frac{\partial}{\partial x} & \frac{\partial}{\partial y} & \frac{\partial}{\partial z} \\
    V_x & V_y & V_z
\end{array} \right|$

\subsubsection{Laplacian of a scalar field}
$\nabla^2\phi\equiv\vec{\nabla}\cdot\vec{\nabla}\phi = \frac{\partial^2\phi}{\partial x^2} + \frac{\partial^2\phi}{\partial y^2} + \frac{\partial^2\phi}{\partial z^2}$

\subsubsection{Line integrals}
$W_{AB}=\displaystyle\int_A^B\vec{F}(s)\cdot d\vec{r}(S)$

\subsubsection{Conservative fields}
It is sufficient and necessary for a force to be conservative that $curl\,\vec{F}=\vec{0}$

\subsubsection{Green's theorem}
$\displaystyle\iint_A \left(\frac{\partial Q}{\partial x} - \frac{\partial P}{\partial y}\right)dxdy = \displaystyle\oint_{\partial A} (P\, dx + Q \, dy)$

\subsubsection{Divergence theorem in two and three dimensions}
$\displaystyle\iint_A div \, \vec{V}dxdy = \displaystyle\oint_{\partial A}\vec{V}\cdot \vec{n} ds$

\vspace{2pt}

\noindent$\displaystyle\iiint_V div (curl \, \vec{F})d^{(3)}\tau = \displaystyle\oiint_A curl \, \vec{F} \cdot \vec{n} d^{(2)}\sigma$

\vspace{2pt}

\noindent$\displaystyle\iiint_V div \vec{F}d^{(3)}\tau = \displaystyle\iint_A\vec{F}\cdot\vec{n}d^{(2)}\sigma$

\subsubsection{Stokes' theorem in two and three dimensions}
$\displaystyle\iint_A (curl \, \vec{V}) \cdot \vec{k}\, dx dy = \displaystyle\oint_{\partial A} \vec{V}\cdot d\vec{r}$

\noindent$\displaystyle\oiint_A curl \, \vec{F} \cdot \vec{n} d^{(2)}\sigma = \displaystyle\oint_{\circlearrowright}\vec{F}\cdot d \vec{r}$

\subsubsection{Divergence theorem on a cylinder}
$\displaystyle\iint_\Sigma \vec{V}\cdot \vec{n} d^{(2)}\sigma = - \displaystyle\iint_{2\,disks} \vec{V}\cdot\vec{n} d^{(2)}\sigma$

\subsubsection{Stokes' theorem in cylindrical coordinates}
$\displaystyle\iint_\Sigma curl \vec{F}\cdot\vec{n}d^{(2)}\sigma = \displaystyle\oint_{topdisk, \circlearrowright}\vec{F}\cdot d\vec{r} + \displaystyle\oint_{bottomdisk, \circlearrowleft}\vec{F}\cdot d\vec{r}$

\subsubsection{Stokes' theorem in spherical coordinates}
$\displaystyle\iint_{\smallfrown}curl \,\vec{F}\cdot\vec{n}d^{(2)}\sigma = + \displaystyle\iint_{topdisk} curl\,\vec{F}\cdot\vec{k}d^{(2)}\sigma = \displaystyle\int_\circlearrowleft\vec{F}\cdot d\vec{r} = \displaystyle\int_\circlearrowleft\vec{F}\cdot Rd\phi\vec{e}_\phi$

\vspace{4pt}

\noindent$\displaystyle\iint_{\smallsmile}curl \,\vec{F}\cdot\vec{n}d^{(2)}\sigma = - \displaystyle\iint_{bottomdisk} curl\,\vec{F}\cdot\vec{k}d^{(2)}\sigma = \displaystyle\int_\circlearrowleft\vec{F}\cdot d\vec{r}$

\subsection{Fourier series}
\begin{equation}
f(x)=\frac{a_0}{2}+\displaystyle\sum_{n=1}^{+\infty}a_n cos\frac{n\pi x}{L} + \displaystyle\sum_{n=1}^{+\infty}b_nsin\frac{n\pi x}{L}
\end{equation}

\subsubsection{Coefficients}
\begin{equation}
\begin{cases}
    a_0=\frac{1}{L}\displaystyle\int_{-L}^{L}f(x)dx, \quad a_n=\frac{1}{L}\displaystyle\int_{-L}^{L}f(x)cos\frac{n\pi x}{L}dx, \\ b_n = \frac{1}{L}\displaystyle\int_{-L}^{L}f(x)sin\frac{n\pi x}{L} dx
    \end{cases}
\end{equation}

\subsection{Complex form of Fourier series}
\centerline{$f(x)=\displaystyle\sum_{n=-\infty}^{+\infty}c_ne^{i\frac{n\pi x}{L}}$}

\subsubsection{Coefficient}
$c_n = \frac{1}{2L}\displaystyle\int_{-L}^{L}f(x)e^{-i\frac{n\pi x}{L}}dx$

\subsection{Dirichlet conditions}
If $f(x)$
$\bullet$ is a periodic function of period $2L$,\\
$\bullet$ is single-valued between $-L$ and $L$,\\
$\bullet$ has a finite number of maximum and minimum values, and a finite number of discontinuities,\\
$\bullet$ and if $\displaystyle\int_{_L}^L|f(x)|$ is finite,

then the Fourier series (1) with the coefficients given by (2) converges to $f(x)$ at all the points where $f(x)$ is continuous;

at jumps the Fourier series converges to the midpoint of the jump, including for jumps occurring at $\pm L$

\subsubsection{Parseval's theorem}
$\frac{1}{2L}\displaystyle\int_{-L}^L|f(x)|^2dx = \left(\frac{a_0}{2}\right)^2 + \frac{1}{2}\displaystyle\sum^{\infty}_{1}a_n^2 + \frac{1}{2}\displaystyle\sum_1^{\infty}b_n^2 = \displaystyle\sum_{-\infty}^{+\infty}|c_n|^2$
\end{document}
