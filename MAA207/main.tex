\documentclass{article}
\usepackage[a4paper, top=1cm, left=2cm, right=2cm, bottom=2cm]{geometry}
\usepackage{titling}
\usepackage{graphicx} % Required for inserting images
\usepackage{pgf}
\usepackage{import}
\usepackage{lmodern}
\usepackage[most]{tcolorbox}
\usepackage{amssymb}

\newcounter{counter}[subsection] % Define a new counter tied to sections
\newcommand{\counterbox}{\thesubsection.\arabic{counter}}

% \newcounter{defcount}[subsection] % Define a new counter tied to sections
% \newcommand{\defbox}{\thesubsection.\arabic{defcount}}

% \newcounter{propcount}[subsection] % Define a new counter tied to sections
% \newcommand{\propbox}{\thesubsection.\arabic{propcount}}

% \newcounter{corrcount}[subsection] % Define a new counter tied to sections
% \newcommand{\corrbox}{\thesubsection.\arabic{corrcount}}

\predate{}
\postdate{}

\title{MAA207 text}
\author{@usercdp}
\date{}

\newcommand{\K}{\mathbb{K}}
\newcommand{\N}{\mathbb{N}}
\newcommand{\C}{\mathbb{C}}
\newcommand{\R}{\mathbb{R}}
\newcommand{\Z}{\mathbb{Z}}
\newcommand{\inner}[2]{\langle #1, #2\rangle}

\newcommand{\theorem}[2]{ % Arguments: #1 = title, #2 = content
\stepcounter{counter}
    \begin{tcolorbox}[
        colback=red!10!white, % Background color for the box
        colframe=red!20, % Border color
        coltitle=black, % Title text color
        fonttitle=\bfseries, % Bold title font
        title={%
        \if\relax\detokenize{#1}\relax
            {Theorem \counterbox}%
        \else%
            {Theorem \counterbox: #1}%
        \fi%
    }, % Title text with full width
        enhanced, % For advanced settings
        boxed title style={
            colback=cyan!60!black, % Color of the bar
            outer arc=0pt, % No rounded corners
            arc=0pt, % No rounded corners
        },
        before upper={\noindent}, % Adjusting content position
    ]
    #2 % Content
    \end{tcolorbox}
    % \vspace{1pt}
}

\newcommand{\definition}[2]{ % Arguments: #1 = title, #2 = content
    \stepcounter{counter}
    % \renewcommand{\defbox}{\thesubsection.\arabic{defcount}}
    \begin{tcolorbox}[
        colback=green!10!white, % Background color for the box
        colframe=green!70!blue!30!white, % Border color
        coltitle=black, % Title text color
        fonttitle=\bfseries, % Bold title font
        title={{Definition \counterbox: #1}}, % Title text with full width
        enhanced, % For advanced settings
        boxed title style={
            colback=cyan!60!black, % Color of the bar
            outer arc=0pt, % No rounded corners
            arc=0pt, % No rounded corners
        },
        before upper={\noindent}, % Adjusting content position \hspace*{0.5em}
    ]
    #2 % Content
    \end{tcolorbox}
    % \vspace{1pt}
}

\newcommand{\proposition}[2]{ % Arguments: #1 = title, #2 = content
    \stepcounter{counter}
    \begin{tcolorbox}[
        colback=cyan!10!white, % Background color for the box
        colframe=cyan!20, % Border color
        coltitle=black, % Title text color
        fonttitle=\bfseries, % Bold title font
        title={
        \if\relax\detokenize{#1}\relax
            {Proposition \counterbox}
        \else
            {Proposition \counterbox: #1}
        \fi 
        },% Title text with full width
        enhanced, % For advanced settings
        boxed title style={
            colback=cyan!60!black, % Color of the bar
            outer arc=0pt, % No rounded corners
            arc=0pt, % No rounded corners
        },
        before upper={\noindent}, % Adjusting content position
    ]
    #2 % Content
    \end{tcolorbox}
    % \vspace{1pt}
}

\newcommand{\corrolary}[1]{ % Arguments: #1 = title, #2 = content
\stepcounter{counter}
    \begin{tcolorbox}[
        colback=blue!10!white, % Background color for the box
        colframe=blue!20, % Border color
        coltitle=black, % Title text color
        fonttitle=\bfseries, % Bold title font
        title=\raggedright{{Corrolary \counterbox}}, % Title text with full width
        enhanced, % For advanced settings
        boxed title style={
            colback=cyan!60!black, % Color of the bar
            outer arc=0pt, % No rounded corners
            arc=0pt, % No rounded corners
        },
        before upper={\noindent}, % Adjusting content position
    ]
    #1 % Content
    \end{tcolorbox}
    % \vspace{1pt}
}

\newcommand{\remark}[1]{ % Arguments: #1 = title, #2 = content
\stepcounter{counter}
    \begin{tcolorbox}[
        colback=black!10!white, % Background color for the box
        colframe=black!20, % Border color
        coltitle=black, % Title text color
        fonttitle=\bfseries, % Bold title font
        title=\raggedright{{Remark \counterbox}}, % Title text with full width
        enhanced, % For advanced settings
        boxed title style={
            colback=white!60!black, % Color of the bar
            outer arc=0pt, % No rounded corners
            arc=0pt, % No rounded corners
        },
        before upper={\noindent}, % Adjusting content position
    ]
    #1 % Content
    \end{tcolorbox}
    % \vspace{1pt}
}

\begin{document}

\maketitle

\section{Chapter I: Sequences and series of functions}

\subsection{Pointwise convergence and the problem of interchanging limits}

\definition{Pointwise convergence}{
(1) Let $(f_n)_{n\in\mathbb{N}}$ be a sequence of functions on $\Omega$. We say that the sequence $(f_n)_{n\in\mathbb{N}}$ is pointwise convergent if for all $x\in\Omega$, $(f_n(x))_{n\in\mathbb{N}}$ is a converging numerical sequence. If $(f_n)_{n\in\mathbb{N}}$ is pointwise convergent, we can define a function $f:\Omega\rightarrow\mathbb{K}$ by $$f(x)=\displaystyle\lim_{n\rightarrow\infty}f_n(x).$$ We say that $f$ is the (pointwise) \textbf{limit} of the sequence $(f_n)_{n\in\mathbb{N}}.$
\\

(2) Let $(g_n)_{n\in\mathbb{N}}$ be a sequence of functions $\Omega\rightarrow\mathbb{K}$. We say that the series $\sum g_n$ is \textbf{pointwise convergent} if for every $x\in\Omega$, the numerical series $\sum g_n(x)$ is convergent. If the series $\sum g_n$ is pointwise convergent, we can define $$g(x)=\displaystyle\sum_{n=0}^\infty g_n(x), x\in\Omega,$$ and we call $g$ the \textbf{sum} of the series $\sum g_n.$
\\

We can note that a series of functions $(\sum g_n)_n$ is pointwise convergent if and only if the sequence of partial sums $(S_N=\sum_{n\leq N}g_n)_N$ is pointwise convergent.
}

\vspace{-10pt}

\begin{figure}[htbp]
    \centering
  \scalebox{0.45}{%% Creator: Matplotlib, PGF backend
%%
%% To include the figure in your LaTeX document, write
%%   \input{<filename>.pgf}
%%
%% Make sure the required packages are loaded in your preamble
%%   \usepackage{pgf}
%%
%% Also ensure that all the required font packages are loaded; for instance,
%% the lmodern package is sometimes necessary when using math font.
%%   \usepackage{lmodern}
%%
%% Figures using additional raster images can only be included by \input if
%% they are in the same directory as the main LaTeX file. For loading figures
%% from other directories you can use the `import` package
%%   \usepackage{import}
%%
%% and then include the figures with
%%   \import{<path to file>}{<filename>.pgf}
%%
%% Matplotlib used the following preamble
%%   
%%   \usepackage{fontspec}
%%   \setmainfont{DejaVuSerif.ttf}[Path=\detokenize{/usr/local/Anaconda3-2023.07-1/lib/python3.11/site-packages/matplotlib/mpl-data/fonts/ttf/}]
%%   \setsansfont{DejaVuSans.ttf}[Path=\detokenize{/usr/local/Anaconda3-2023.07-1/lib/python3.11/site-packages/matplotlib/mpl-data/fonts/ttf/}]
%%   \setmonofont{DejaVuSansMono.ttf}[Path=\detokenize{/usr/local/Anaconda3-2023.07-1/lib/python3.11/site-packages/matplotlib/mpl-data/fonts/ttf/}]
%%   \makeatletter\@ifpackageloaded{underscore}{}{\usepackage[strings]{underscore}}\makeatother
%%
\begingroup%
\makeatletter%
\begin{pgfpicture}%
\pgfpathrectangle{\pgfpointorigin}{\pgfqpoint{6.400000in}{4.800000in}}%
\pgfusepath{use as bounding box, clip}%
\begin{pgfscope}%
\pgfsetbuttcap%
\pgfsetmiterjoin%
\definecolor{currentfill}{rgb}{1.000000,1.000000,1.000000}%
\pgfsetfillcolor{currentfill}%
\pgfsetlinewidth{0.000000pt}%
\definecolor{currentstroke}{rgb}{1.000000,1.000000,1.000000}%
\pgfsetstrokecolor{currentstroke}%
\pgfsetdash{}{0pt}%
\pgfpathmoveto{\pgfqpoint{0.000000in}{0.000000in}}%
\pgfpathlineto{\pgfqpoint{6.400000in}{0.000000in}}%
\pgfpathlineto{\pgfqpoint{6.400000in}{4.800000in}}%
\pgfpathlineto{\pgfqpoint{0.000000in}{4.800000in}}%
\pgfpathlineto{\pgfqpoint{0.000000in}{0.000000in}}%
\pgfpathclose%
\pgfusepath{fill}%
\end{pgfscope}%
\begin{pgfscope}%
\pgfsetbuttcap%
\pgfsetmiterjoin%
\definecolor{currentfill}{rgb}{1.000000,1.000000,1.000000}%
\pgfsetfillcolor{currentfill}%
\pgfsetlinewidth{0.000000pt}%
\definecolor{currentstroke}{rgb}{0.000000,0.000000,0.000000}%
\pgfsetstrokecolor{currentstroke}%
\pgfsetstrokeopacity{0.000000}%
\pgfsetdash{}{0pt}%
\pgfpathmoveto{\pgfqpoint{0.800000in}{0.528000in}}%
\pgfpathlineto{\pgfqpoint{5.760000in}{0.528000in}}%
\pgfpathlineto{\pgfqpoint{5.760000in}{4.224000in}}%
\pgfpathlineto{\pgfqpoint{0.800000in}{4.224000in}}%
\pgfpathlineto{\pgfqpoint{0.800000in}{0.528000in}}%
\pgfpathclose%
\pgfusepath{fill}%
\end{pgfscope}%
\begin{pgfscope}%
\pgfsetbuttcap%
\pgfsetroundjoin%
\definecolor{currentfill}{rgb}{0.000000,0.000000,0.000000}%
\pgfsetfillcolor{currentfill}%
\pgfsetlinewidth{0.803000pt}%
\definecolor{currentstroke}{rgb}{0.000000,0.000000,0.000000}%
\pgfsetstrokecolor{currentstroke}%
\pgfsetdash{}{0pt}%
\pgfsys@defobject{currentmarker}{\pgfqpoint{0.000000in}{-0.048611in}}{\pgfqpoint{0.000000in}{0.000000in}}{%
\pgfpathmoveto{\pgfqpoint{0.000000in}{0.000000in}}%
\pgfpathlineto{\pgfqpoint{0.000000in}{-0.048611in}}%
\pgfusepath{stroke,fill}%
}%
\begin{pgfscope}%
\pgfsys@transformshift{1.025455in}{0.528000in}%
\pgfsys@useobject{currentmarker}{}%
\end{pgfscope}%
\end{pgfscope}%
\begin{pgfscope}%
\definecolor{textcolor}{rgb}{0.000000,0.000000,0.000000}%
\pgfsetstrokecolor{textcolor}%
\pgfsetfillcolor{textcolor}%
\pgftext[x=1.025455in,y=0.430778in,,top]{\color{textcolor}\sffamily\fontsize{10.000000}{12.000000}\selectfont 0.0}%
\end{pgfscope}%
\begin{pgfscope}%
\pgfsetbuttcap%
\pgfsetroundjoin%
\definecolor{currentfill}{rgb}{0.000000,0.000000,0.000000}%
\pgfsetfillcolor{currentfill}%
\pgfsetlinewidth{0.803000pt}%
\definecolor{currentstroke}{rgb}{0.000000,0.000000,0.000000}%
\pgfsetstrokecolor{currentstroke}%
\pgfsetdash{}{0pt}%
\pgfsys@defobject{currentmarker}{\pgfqpoint{0.000000in}{-0.048611in}}{\pgfqpoint{0.000000in}{0.000000in}}{%
\pgfpathmoveto{\pgfqpoint{0.000000in}{0.000000in}}%
\pgfpathlineto{\pgfqpoint{0.000000in}{-0.048611in}}%
\pgfusepath{stroke,fill}%
}%
\begin{pgfscope}%
\pgfsys@transformshift{1.927273in}{0.528000in}%
\pgfsys@useobject{currentmarker}{}%
\end{pgfscope}%
\end{pgfscope}%
\begin{pgfscope}%
\definecolor{textcolor}{rgb}{0.000000,0.000000,0.000000}%
\pgfsetstrokecolor{textcolor}%
\pgfsetfillcolor{textcolor}%
\pgftext[x=1.927273in,y=0.430778in,,top]{\color{textcolor}\sffamily\fontsize{10.000000}{12.000000}\selectfont 0.2}%
\end{pgfscope}%
\begin{pgfscope}%
\pgfsetbuttcap%
\pgfsetroundjoin%
\definecolor{currentfill}{rgb}{0.000000,0.000000,0.000000}%
\pgfsetfillcolor{currentfill}%
\pgfsetlinewidth{0.803000pt}%
\definecolor{currentstroke}{rgb}{0.000000,0.000000,0.000000}%
\pgfsetstrokecolor{currentstroke}%
\pgfsetdash{}{0pt}%
\pgfsys@defobject{currentmarker}{\pgfqpoint{0.000000in}{-0.048611in}}{\pgfqpoint{0.000000in}{0.000000in}}{%
\pgfpathmoveto{\pgfqpoint{0.000000in}{0.000000in}}%
\pgfpathlineto{\pgfqpoint{0.000000in}{-0.048611in}}%
\pgfusepath{stroke,fill}%
}%
\begin{pgfscope}%
\pgfsys@transformshift{2.829091in}{0.528000in}%
\pgfsys@useobject{currentmarker}{}%
\end{pgfscope}%
\end{pgfscope}%
\begin{pgfscope}%
\definecolor{textcolor}{rgb}{0.000000,0.000000,0.000000}%
\pgfsetstrokecolor{textcolor}%
\pgfsetfillcolor{textcolor}%
\pgftext[x=2.829091in,y=0.430778in,,top]{\color{textcolor}\sffamily\fontsize{10.000000}{12.000000}\selectfont 0.4}%
\end{pgfscope}%
\begin{pgfscope}%
\pgfsetbuttcap%
\pgfsetroundjoin%
\definecolor{currentfill}{rgb}{0.000000,0.000000,0.000000}%
\pgfsetfillcolor{currentfill}%
\pgfsetlinewidth{0.803000pt}%
\definecolor{currentstroke}{rgb}{0.000000,0.000000,0.000000}%
\pgfsetstrokecolor{currentstroke}%
\pgfsetdash{}{0pt}%
\pgfsys@defobject{currentmarker}{\pgfqpoint{0.000000in}{-0.048611in}}{\pgfqpoint{0.000000in}{0.000000in}}{%
\pgfpathmoveto{\pgfqpoint{0.000000in}{0.000000in}}%
\pgfpathlineto{\pgfqpoint{0.000000in}{-0.048611in}}%
\pgfusepath{stroke,fill}%
}%
\begin{pgfscope}%
\pgfsys@transformshift{3.730909in}{0.528000in}%
\pgfsys@useobject{currentmarker}{}%
\end{pgfscope}%
\end{pgfscope}%
\begin{pgfscope}%
\definecolor{textcolor}{rgb}{0.000000,0.000000,0.000000}%
\pgfsetstrokecolor{textcolor}%
\pgfsetfillcolor{textcolor}%
\pgftext[x=3.730909in,y=0.430778in,,top]{\color{textcolor}\sffamily\fontsize{10.000000}{12.000000}\selectfont 0.6}%
\end{pgfscope}%
\begin{pgfscope}%
\pgfsetbuttcap%
\pgfsetroundjoin%
\definecolor{currentfill}{rgb}{0.000000,0.000000,0.000000}%
\pgfsetfillcolor{currentfill}%
\pgfsetlinewidth{0.803000pt}%
\definecolor{currentstroke}{rgb}{0.000000,0.000000,0.000000}%
\pgfsetstrokecolor{currentstroke}%
\pgfsetdash{}{0pt}%
\pgfsys@defobject{currentmarker}{\pgfqpoint{0.000000in}{-0.048611in}}{\pgfqpoint{0.000000in}{0.000000in}}{%
\pgfpathmoveto{\pgfqpoint{0.000000in}{0.000000in}}%
\pgfpathlineto{\pgfqpoint{0.000000in}{-0.048611in}}%
\pgfusepath{stroke,fill}%
}%
\begin{pgfscope}%
\pgfsys@transformshift{4.632727in}{0.528000in}%
\pgfsys@useobject{currentmarker}{}%
\end{pgfscope}%
\end{pgfscope}%
\begin{pgfscope}%
\definecolor{textcolor}{rgb}{0.000000,0.000000,0.000000}%
\pgfsetstrokecolor{textcolor}%
\pgfsetfillcolor{textcolor}%
\pgftext[x=4.632727in,y=0.430778in,,top]{\color{textcolor}\sffamily\fontsize{10.000000}{12.000000}\selectfont 0.8}%
\end{pgfscope}%
\begin{pgfscope}%
\pgfsetbuttcap%
\pgfsetroundjoin%
\definecolor{currentfill}{rgb}{0.000000,0.000000,0.000000}%
\pgfsetfillcolor{currentfill}%
\pgfsetlinewidth{0.803000pt}%
\definecolor{currentstroke}{rgb}{0.000000,0.000000,0.000000}%
\pgfsetstrokecolor{currentstroke}%
\pgfsetdash{}{0pt}%
\pgfsys@defobject{currentmarker}{\pgfqpoint{0.000000in}{-0.048611in}}{\pgfqpoint{0.000000in}{0.000000in}}{%
\pgfpathmoveto{\pgfqpoint{0.000000in}{0.000000in}}%
\pgfpathlineto{\pgfqpoint{0.000000in}{-0.048611in}}%
\pgfusepath{stroke,fill}%
}%
\begin{pgfscope}%
\pgfsys@transformshift{5.534545in}{0.528000in}%
\pgfsys@useobject{currentmarker}{}%
\end{pgfscope}%
\end{pgfscope}%
\begin{pgfscope}%
\definecolor{textcolor}{rgb}{0.000000,0.000000,0.000000}%
\pgfsetstrokecolor{textcolor}%
\pgfsetfillcolor{textcolor}%
\pgftext[x=5.534545in,y=0.430778in,,top]{\color{textcolor}\sffamily\fontsize{10.000000}{12.000000}\selectfont 1.0}%
\end{pgfscope}%
\begin{pgfscope}%
\definecolor{textcolor}{rgb}{0.000000,0.000000,0.000000}%
\pgfsetstrokecolor{textcolor}%
\pgfsetfillcolor{textcolor}%
\pgftext[x=3.280000in,y=0.240809in,,top]{\color{textcolor}\sffamily\fontsize{10.000000}{12.000000}\selectfont x}%
\end{pgfscope}%
\begin{pgfscope}%
\pgfsetbuttcap%
\pgfsetroundjoin%
\definecolor{currentfill}{rgb}{0.000000,0.000000,0.000000}%
\pgfsetfillcolor{currentfill}%
\pgfsetlinewidth{0.803000pt}%
\definecolor{currentstroke}{rgb}{0.000000,0.000000,0.000000}%
\pgfsetstrokecolor{currentstroke}%
\pgfsetdash{}{0pt}%
\pgfsys@defobject{currentmarker}{\pgfqpoint{-0.048611in}{0.000000in}}{\pgfqpoint{-0.000000in}{0.000000in}}{%
\pgfpathmoveto{\pgfqpoint{-0.000000in}{0.000000in}}%
\pgfpathlineto{\pgfqpoint{-0.048611in}{0.000000in}}%
\pgfusepath{stroke,fill}%
}%
\begin{pgfscope}%
\pgfsys@transformshift{0.800000in}{0.696000in}%
\pgfsys@useobject{currentmarker}{}%
\end{pgfscope}%
\end{pgfscope}%
\begin{pgfscope}%
\definecolor{textcolor}{rgb}{0.000000,0.000000,0.000000}%
\pgfsetstrokecolor{textcolor}%
\pgfsetfillcolor{textcolor}%
\pgftext[x=0.481898in, y=0.643238in, left, base]{\color{textcolor}\sffamily\fontsize{10.000000}{12.000000}\selectfont 0.0}%
\end{pgfscope}%
\begin{pgfscope}%
\pgfsetbuttcap%
\pgfsetroundjoin%
\definecolor{currentfill}{rgb}{0.000000,0.000000,0.000000}%
\pgfsetfillcolor{currentfill}%
\pgfsetlinewidth{0.803000pt}%
\definecolor{currentstroke}{rgb}{0.000000,0.000000,0.000000}%
\pgfsetstrokecolor{currentstroke}%
\pgfsetdash{}{0pt}%
\pgfsys@defobject{currentmarker}{\pgfqpoint{-0.048611in}{0.000000in}}{\pgfqpoint{-0.000000in}{0.000000in}}{%
\pgfpathmoveto{\pgfqpoint{-0.000000in}{0.000000in}}%
\pgfpathlineto{\pgfqpoint{-0.048611in}{0.000000in}}%
\pgfusepath{stroke,fill}%
}%
\begin{pgfscope}%
\pgfsys@transformshift{0.800000in}{1.368000in}%
\pgfsys@useobject{currentmarker}{}%
\end{pgfscope}%
\end{pgfscope}%
\begin{pgfscope}%
\definecolor{textcolor}{rgb}{0.000000,0.000000,0.000000}%
\pgfsetstrokecolor{textcolor}%
\pgfsetfillcolor{textcolor}%
\pgftext[x=0.481898in, y=1.315238in, left, base]{\color{textcolor}\sffamily\fontsize{10.000000}{12.000000}\selectfont 0.2}%
\end{pgfscope}%
\begin{pgfscope}%
\pgfsetbuttcap%
\pgfsetroundjoin%
\definecolor{currentfill}{rgb}{0.000000,0.000000,0.000000}%
\pgfsetfillcolor{currentfill}%
\pgfsetlinewidth{0.803000pt}%
\definecolor{currentstroke}{rgb}{0.000000,0.000000,0.000000}%
\pgfsetstrokecolor{currentstroke}%
\pgfsetdash{}{0pt}%
\pgfsys@defobject{currentmarker}{\pgfqpoint{-0.048611in}{0.000000in}}{\pgfqpoint{-0.000000in}{0.000000in}}{%
\pgfpathmoveto{\pgfqpoint{-0.000000in}{0.000000in}}%
\pgfpathlineto{\pgfqpoint{-0.048611in}{0.000000in}}%
\pgfusepath{stroke,fill}%
}%
\begin{pgfscope}%
\pgfsys@transformshift{0.800000in}{2.040000in}%
\pgfsys@useobject{currentmarker}{}%
\end{pgfscope}%
\end{pgfscope}%
\begin{pgfscope}%
\definecolor{textcolor}{rgb}{0.000000,0.000000,0.000000}%
\pgfsetstrokecolor{textcolor}%
\pgfsetfillcolor{textcolor}%
\pgftext[x=0.481898in, y=1.987238in, left, base]{\color{textcolor}\sffamily\fontsize{10.000000}{12.000000}\selectfont 0.4}%
\end{pgfscope}%
\begin{pgfscope}%
\pgfsetbuttcap%
\pgfsetroundjoin%
\definecolor{currentfill}{rgb}{0.000000,0.000000,0.000000}%
\pgfsetfillcolor{currentfill}%
\pgfsetlinewidth{0.803000pt}%
\definecolor{currentstroke}{rgb}{0.000000,0.000000,0.000000}%
\pgfsetstrokecolor{currentstroke}%
\pgfsetdash{}{0pt}%
\pgfsys@defobject{currentmarker}{\pgfqpoint{-0.048611in}{0.000000in}}{\pgfqpoint{-0.000000in}{0.000000in}}{%
\pgfpathmoveto{\pgfqpoint{-0.000000in}{0.000000in}}%
\pgfpathlineto{\pgfqpoint{-0.048611in}{0.000000in}}%
\pgfusepath{stroke,fill}%
}%
\begin{pgfscope}%
\pgfsys@transformshift{0.800000in}{2.712000in}%
\pgfsys@useobject{currentmarker}{}%
\end{pgfscope}%
\end{pgfscope}%
\begin{pgfscope}%
\definecolor{textcolor}{rgb}{0.000000,0.000000,0.000000}%
\pgfsetstrokecolor{textcolor}%
\pgfsetfillcolor{textcolor}%
\pgftext[x=0.481898in, y=2.659238in, left, base]{\color{textcolor}\sffamily\fontsize{10.000000}{12.000000}\selectfont 0.6}%
\end{pgfscope}%
\begin{pgfscope}%
\pgfsetbuttcap%
\pgfsetroundjoin%
\definecolor{currentfill}{rgb}{0.000000,0.000000,0.000000}%
\pgfsetfillcolor{currentfill}%
\pgfsetlinewidth{0.803000pt}%
\definecolor{currentstroke}{rgb}{0.000000,0.000000,0.000000}%
\pgfsetstrokecolor{currentstroke}%
\pgfsetdash{}{0pt}%
\pgfsys@defobject{currentmarker}{\pgfqpoint{-0.048611in}{0.000000in}}{\pgfqpoint{-0.000000in}{0.000000in}}{%
\pgfpathmoveto{\pgfqpoint{-0.000000in}{0.000000in}}%
\pgfpathlineto{\pgfqpoint{-0.048611in}{0.000000in}}%
\pgfusepath{stroke,fill}%
}%
\begin{pgfscope}%
\pgfsys@transformshift{0.800000in}{3.384000in}%
\pgfsys@useobject{currentmarker}{}%
\end{pgfscope}%
\end{pgfscope}%
\begin{pgfscope}%
\definecolor{textcolor}{rgb}{0.000000,0.000000,0.000000}%
\pgfsetstrokecolor{textcolor}%
\pgfsetfillcolor{textcolor}%
\pgftext[x=0.481898in, y=3.331238in, left, base]{\color{textcolor}\sffamily\fontsize{10.000000}{12.000000}\selectfont 0.8}%
\end{pgfscope}%
\begin{pgfscope}%
\pgfsetbuttcap%
\pgfsetroundjoin%
\definecolor{currentfill}{rgb}{0.000000,0.000000,0.000000}%
\pgfsetfillcolor{currentfill}%
\pgfsetlinewidth{0.803000pt}%
\definecolor{currentstroke}{rgb}{0.000000,0.000000,0.000000}%
\pgfsetstrokecolor{currentstroke}%
\pgfsetdash{}{0pt}%
\pgfsys@defobject{currentmarker}{\pgfqpoint{-0.048611in}{0.000000in}}{\pgfqpoint{-0.000000in}{0.000000in}}{%
\pgfpathmoveto{\pgfqpoint{-0.000000in}{0.000000in}}%
\pgfpathlineto{\pgfqpoint{-0.048611in}{0.000000in}}%
\pgfusepath{stroke,fill}%
}%
\begin{pgfscope}%
\pgfsys@transformshift{0.800000in}{4.056000in}%
\pgfsys@useobject{currentmarker}{}%
\end{pgfscope}%
\end{pgfscope}%
\begin{pgfscope}%
\definecolor{textcolor}{rgb}{0.000000,0.000000,0.000000}%
\pgfsetstrokecolor{textcolor}%
\pgfsetfillcolor{textcolor}%
\pgftext[x=0.481898in, y=4.003238in, left, base]{\color{textcolor}\sffamily\fontsize{10.000000}{12.000000}\selectfont 1.0}%
\end{pgfscope}%
\begin{pgfscope}%
\definecolor{textcolor}{rgb}{0.000000,0.000000,0.000000}%
\pgfsetstrokecolor{textcolor}%
\pgfsetfillcolor{textcolor}%
\pgftext[x=0.426343in,y=2.376000in,,bottom,rotate=90.000000]{\color{textcolor}\sffamily\fontsize{10.000000}{12.000000}\selectfont f_n(x)}%
\end{pgfscope}%
\begin{pgfscope}%
\pgfpathrectangle{\pgfqpoint{0.800000in}{0.528000in}}{\pgfqpoint{4.960000in}{3.696000in}}%
\pgfusepath{clip}%
\pgfsetrectcap%
\pgfsetroundjoin%
\pgfsetlinewidth{1.505625pt}%
\definecolor{currentstroke}{rgb}{0.121569,0.466667,0.705882}%
\pgfsetstrokecolor{currentstroke}%
\pgfsetdash{}{0pt}%
\pgfpathmoveto{\pgfqpoint{1.025455in}{0.696000in}}%
\pgfpathlineto{\pgfqpoint{5.534545in}{4.056000in}}%
\pgfpathlineto{\pgfqpoint{5.534545in}{4.056000in}}%
\pgfusepath{stroke}%
\end{pgfscope}%
\begin{pgfscope}%
\pgfpathrectangle{\pgfqpoint{0.800000in}{0.528000in}}{\pgfqpoint{4.960000in}{3.696000in}}%
\pgfusepath{clip}%
\pgfsetrectcap%
\pgfsetroundjoin%
\pgfsetlinewidth{1.505625pt}%
\definecolor{currentstroke}{rgb}{1.000000,0.498039,0.054902}%
\pgfsetstrokecolor{currentstroke}%
\pgfsetdash{}{0pt}%
\pgfpathmoveto{\pgfqpoint{1.025455in}{0.696000in}}%
\pgfpathlineto{\pgfqpoint{1.104561in}{0.697034in}}%
\pgfpathlineto{\pgfqpoint{1.183668in}{0.700137in}}%
\pgfpathlineto{\pgfqpoint{1.262775in}{0.705307in}}%
\pgfpathlineto{\pgfqpoint{1.341882in}{0.712547in}}%
\pgfpathlineto{\pgfqpoint{1.420989in}{0.721854in}}%
\pgfpathlineto{\pgfqpoint{1.500096in}{0.733230in}}%
\pgfpathlineto{\pgfqpoint{1.579203in}{0.746674in}}%
\pgfpathlineto{\pgfqpoint{1.658309in}{0.762187in}}%
\pgfpathlineto{\pgfqpoint{1.737416in}{0.779767in}}%
\pgfpathlineto{\pgfqpoint{1.816523in}{0.799416in}}%
\pgfpathlineto{\pgfqpoint{1.895630in}{0.821134in}}%
\pgfpathlineto{\pgfqpoint{1.974737in}{0.844920in}}%
\pgfpathlineto{\pgfqpoint{2.053844in}{0.870774in}}%
\pgfpathlineto{\pgfqpoint{2.132951in}{0.898696in}}%
\pgfpathlineto{\pgfqpoint{2.223358in}{0.933140in}}%
\pgfpathlineto{\pgfqpoint{2.313766in}{0.970286in}}%
\pgfpathlineto{\pgfqpoint{2.404174in}{1.010133in}}%
\pgfpathlineto{\pgfqpoint{2.494582in}{1.052681in}}%
\pgfpathlineto{\pgfqpoint{2.584990in}{1.097931in}}%
\pgfpathlineto{\pgfqpoint{2.675398in}{1.145883in}}%
\pgfpathlineto{\pgfqpoint{2.765805in}{1.196536in}}%
\pgfpathlineto{\pgfqpoint{2.856213in}{1.249890in}}%
\pgfpathlineto{\pgfqpoint{2.946621in}{1.305946in}}%
\pgfpathlineto{\pgfqpoint{3.037029in}{1.364703in}}%
\pgfpathlineto{\pgfqpoint{3.127437in}{1.426162in}}%
\pgfpathlineto{\pgfqpoint{3.217845in}{1.490323in}}%
\pgfpathlineto{\pgfqpoint{3.308252in}{1.557185in}}%
\pgfpathlineto{\pgfqpoint{3.398660in}{1.626748in}}%
\pgfpathlineto{\pgfqpoint{3.489068in}{1.699013in}}%
\pgfpathlineto{\pgfqpoint{3.579476in}{1.773979in}}%
\pgfpathlineto{\pgfqpoint{3.669884in}{1.851647in}}%
\pgfpathlineto{\pgfqpoint{3.760292in}{1.932016in}}%
\pgfpathlineto{\pgfqpoint{3.850699in}{2.015087in}}%
\pgfpathlineto{\pgfqpoint{3.952408in}{2.111771in}}%
\pgfpathlineto{\pgfqpoint{4.054117in}{2.211874in}}%
\pgfpathlineto{\pgfqpoint{4.155826in}{2.315396in}}%
\pgfpathlineto{\pgfqpoint{4.257535in}{2.422337in}}%
\pgfpathlineto{\pgfqpoint{4.359244in}{2.532697in}}%
\pgfpathlineto{\pgfqpoint{4.460952in}{2.646476in}}%
\pgfpathlineto{\pgfqpoint{4.562661in}{2.763674in}}%
\pgfpathlineto{\pgfqpoint{4.664370in}{2.884292in}}%
\pgfpathlineto{\pgfqpoint{4.766079in}{3.008328in}}%
\pgfpathlineto{\pgfqpoint{4.867788in}{3.135784in}}%
\pgfpathlineto{\pgfqpoint{4.969496in}{3.266658in}}%
\pgfpathlineto{\pgfqpoint{5.071205in}{3.400952in}}%
\pgfpathlineto{\pgfqpoint{5.172914in}{3.538665in}}%
\pgfpathlineto{\pgfqpoint{5.274623in}{3.679796in}}%
\pgfpathlineto{\pgfqpoint{5.387633in}{3.840619in}}%
\pgfpathlineto{\pgfqpoint{5.500643in}{4.005664in}}%
\pgfpathlineto{\pgfqpoint{5.534545in}{4.056000in}}%
\pgfpathlineto{\pgfqpoint{5.534545in}{4.056000in}}%
\pgfusepath{stroke}%
\end{pgfscope}%
\begin{pgfscope}%
\pgfpathrectangle{\pgfqpoint{0.800000in}{0.528000in}}{\pgfqpoint{4.960000in}{3.696000in}}%
\pgfusepath{clip}%
\pgfsetrectcap%
\pgfsetroundjoin%
\pgfsetlinewidth{1.505625pt}%
\definecolor{currentstroke}{rgb}{0.172549,0.627451,0.172549}%
\pgfsetstrokecolor{currentstroke}%
\pgfsetdash{}{0pt}%
\pgfpathmoveto{\pgfqpoint{1.025455in}{0.696000in}}%
\pgfpathlineto{\pgfqpoint{1.624406in}{0.697046in}}%
\pgfpathlineto{\pgfqpoint{1.850426in}{0.699765in}}%
\pgfpathlineto{\pgfqpoint{2.019941in}{0.703950in}}%
\pgfpathlineto{\pgfqpoint{2.166853in}{0.709795in}}%
\pgfpathlineto{\pgfqpoint{2.291164in}{0.716860in}}%
\pgfpathlineto{\pgfqpoint{2.404174in}{0.725369in}}%
\pgfpathlineto{\pgfqpoint{2.505883in}{0.735042in}}%
\pgfpathlineto{\pgfqpoint{2.607592in}{0.746928in}}%
\pgfpathlineto{\pgfqpoint{2.698000in}{0.759605in}}%
\pgfpathlineto{\pgfqpoint{2.788407in}{0.774514in}}%
\pgfpathlineto{\pgfqpoint{2.878815in}{0.791901in}}%
\pgfpathlineto{\pgfqpoint{2.957922in}{0.809353in}}%
\pgfpathlineto{\pgfqpoint{3.037029in}{0.829085in}}%
\pgfpathlineto{\pgfqpoint{3.116136in}{0.851287in}}%
\pgfpathlineto{\pgfqpoint{3.183942in}{0.872434in}}%
\pgfpathlineto{\pgfqpoint{3.251748in}{0.895670in}}%
\pgfpathlineto{\pgfqpoint{3.319553in}{0.921129in}}%
\pgfpathlineto{\pgfqpoint{3.387359in}{0.948949in}}%
\pgfpathlineto{\pgfqpoint{3.455165in}{0.979271in}}%
\pgfpathlineto{\pgfqpoint{3.522971in}{1.012240in}}%
\pgfpathlineto{\pgfqpoint{3.590777in}{1.048007in}}%
\pgfpathlineto{\pgfqpoint{3.658583in}{1.086725in}}%
\pgfpathlineto{\pgfqpoint{3.715088in}{1.121359in}}%
\pgfpathlineto{\pgfqpoint{3.771593in}{1.158246in}}%
\pgfpathlineto{\pgfqpoint{3.828098in}{1.197481in}}%
\pgfpathlineto{\pgfqpoint{3.884602in}{1.239163in}}%
\pgfpathlineto{\pgfqpoint{3.941107in}{1.283390in}}%
\pgfpathlineto{\pgfqpoint{3.997612in}{1.330265in}}%
\pgfpathlineto{\pgfqpoint{4.054117in}{1.379891in}}%
\pgfpathlineto{\pgfqpoint{4.110622in}{1.432374in}}%
\pgfpathlineto{\pgfqpoint{4.167127in}{1.487821in}}%
\pgfpathlineto{\pgfqpoint{4.223632in}{1.546342in}}%
\pgfpathlineto{\pgfqpoint{4.280137in}{1.608048in}}%
\pgfpathlineto{\pgfqpoint{4.336642in}{1.673053in}}%
\pgfpathlineto{\pgfqpoint{4.393147in}{1.741473in}}%
\pgfpathlineto{\pgfqpoint{4.449651in}{1.813425in}}%
\pgfpathlineto{\pgfqpoint{4.506156in}{1.889028in}}%
\pgfpathlineto{\pgfqpoint{4.562661in}{1.968404in}}%
\pgfpathlineto{\pgfqpoint{4.619166in}{2.051677in}}%
\pgfpathlineto{\pgfqpoint{4.675671in}{2.138972in}}%
\pgfpathlineto{\pgfqpoint{4.732176in}{2.230416in}}%
\pgfpathlineto{\pgfqpoint{4.788681in}{2.326139in}}%
\pgfpathlineto{\pgfqpoint{4.845186in}{2.426272in}}%
\pgfpathlineto{\pgfqpoint{4.901691in}{2.530950in}}%
\pgfpathlineto{\pgfqpoint{4.958195in}{2.640306in}}%
\pgfpathlineto{\pgfqpoint{5.014700in}{2.754479in}}%
\pgfpathlineto{\pgfqpoint{5.071205in}{2.873608in}}%
\pgfpathlineto{\pgfqpoint{5.127710in}{2.997835in}}%
\pgfpathlineto{\pgfqpoint{5.184215in}{3.127302in}}%
\pgfpathlineto{\pgfqpoint{5.240720in}{3.262156in}}%
\pgfpathlineto{\pgfqpoint{5.297225in}{3.402543in}}%
\pgfpathlineto{\pgfqpoint{5.353730in}{3.548613in}}%
\pgfpathlineto{\pgfqpoint{5.410235in}{3.700517in}}%
\pgfpathlineto{\pgfqpoint{5.466740in}{3.858408in}}%
\pgfpathlineto{\pgfqpoint{5.523244in}{4.022442in}}%
\pgfpathlineto{\pgfqpoint{5.534545in}{4.056000in}}%
\pgfpathlineto{\pgfqpoint{5.534545in}{4.056000in}}%
\pgfusepath{stroke}%
\end{pgfscope}%
\begin{pgfscope}%
\pgfpathrectangle{\pgfqpoint{0.800000in}{0.528000in}}{\pgfqpoint{4.960000in}{3.696000in}}%
\pgfusepath{clip}%
\pgfsetrectcap%
\pgfsetroundjoin%
\pgfsetlinewidth{1.505625pt}%
\definecolor{currentstroke}{rgb}{0.839216,0.152941,0.156863}%
\pgfsetstrokecolor{currentstroke}%
\pgfsetdash{}{0pt}%
\pgfpathmoveto{\pgfqpoint{1.025455in}{0.696000in}}%
\pgfpathlineto{\pgfqpoint{2.675398in}{0.697080in}}%
\pgfpathlineto{\pgfqpoint{2.935320in}{0.699481in}}%
\pgfpathlineto{\pgfqpoint{3.116136in}{0.703177in}}%
\pgfpathlineto{\pgfqpoint{3.263049in}{0.708356in}}%
\pgfpathlineto{\pgfqpoint{3.387359in}{0.715043in}}%
\pgfpathlineto{\pgfqpoint{3.489068in}{0.722681in}}%
\pgfpathlineto{\pgfqpoint{3.579476in}{0.731598in}}%
\pgfpathlineto{\pgfqpoint{3.658583in}{0.741436in}}%
\pgfpathlineto{\pgfqpoint{3.737690in}{0.753577in}}%
\pgfpathlineto{\pgfqpoint{3.805496in}{0.766152in}}%
\pgfpathlineto{\pgfqpoint{3.873301in}{0.781067in}}%
\pgfpathlineto{\pgfqpoint{3.929806in}{0.795545in}}%
\pgfpathlineto{\pgfqpoint{3.986311in}{0.812136in}}%
\pgfpathlineto{\pgfqpoint{4.042816in}{0.831097in}}%
\pgfpathlineto{\pgfqpoint{4.099321in}{0.852714in}}%
\pgfpathlineto{\pgfqpoint{4.144525in}{0.872128in}}%
\pgfpathlineto{\pgfqpoint{4.189729in}{0.893615in}}%
\pgfpathlineto{\pgfqpoint{4.234933in}{0.917362in}}%
\pgfpathlineto{\pgfqpoint{4.280137in}{0.943569in}}%
\pgfpathlineto{\pgfqpoint{4.325341in}{0.972451in}}%
\pgfpathlineto{\pgfqpoint{4.370545in}{1.004241in}}%
\pgfpathlineto{\pgfqpoint{4.415748in}{1.039183in}}%
\pgfpathlineto{\pgfqpoint{4.460952in}{1.077544in}}%
\pgfpathlineto{\pgfqpoint{4.506156in}{1.119606in}}%
\pgfpathlineto{\pgfqpoint{4.551360in}{1.165670in}}%
\pgfpathlineto{\pgfqpoint{4.585263in}{1.203038in}}%
\pgfpathlineto{\pgfqpoint{4.619166in}{1.242982in}}%
\pgfpathlineto{\pgfqpoint{4.653069in}{1.285653in}}%
\pgfpathlineto{\pgfqpoint{4.686972in}{1.331208in}}%
\pgfpathlineto{\pgfqpoint{4.720875in}{1.379814in}}%
\pgfpathlineto{\pgfqpoint{4.754778in}{1.431644in}}%
\pgfpathlineto{\pgfqpoint{4.788681in}{1.486879in}}%
\pgfpathlineto{\pgfqpoint{4.822584in}{1.545709in}}%
\pgfpathlineto{\pgfqpoint{4.856487in}{1.608334in}}%
\pgfpathlineto{\pgfqpoint{4.890390in}{1.674960in}}%
\pgfpathlineto{\pgfqpoint{4.924293in}{1.745806in}}%
\pgfpathlineto{\pgfqpoint{4.958195in}{1.821097in}}%
\pgfpathlineto{\pgfqpoint{4.992098in}{1.901072in}}%
\pgfpathlineto{\pgfqpoint{5.026001in}{1.985978in}}%
\pgfpathlineto{\pgfqpoint{5.059904in}{2.076072in}}%
\pgfpathlineto{\pgfqpoint{5.093807in}{2.171625in}}%
\pgfpathlineto{\pgfqpoint{5.127710in}{2.272918in}}%
\pgfpathlineto{\pgfqpoint{5.161613in}{2.380243in}}%
\pgfpathlineto{\pgfqpoint{5.195516in}{2.493906in}}%
\pgfpathlineto{\pgfqpoint{5.229419in}{2.614225in}}%
\pgfpathlineto{\pgfqpoint{5.263322in}{2.741532in}}%
\pgfpathlineto{\pgfqpoint{5.297225in}{2.876171in}}%
\pgfpathlineto{\pgfqpoint{5.331128in}{3.018501in}}%
\pgfpathlineto{\pgfqpoint{5.365031in}{3.168896in}}%
\pgfpathlineto{\pgfqpoint{5.398934in}{3.327745in}}%
\pgfpathlineto{\pgfqpoint{5.432837in}{3.495452in}}%
\pgfpathlineto{\pgfqpoint{5.466740in}{3.672436in}}%
\pgfpathlineto{\pgfqpoint{5.500643in}{3.859134in}}%
\pgfpathlineto{\pgfqpoint{5.534545in}{4.056000in}}%
\pgfpathlineto{\pgfqpoint{5.534545in}{4.056000in}}%
\pgfusepath{stroke}%
\end{pgfscope}%
\begin{pgfscope}%
\pgfpathrectangle{\pgfqpoint{0.800000in}{0.528000in}}{\pgfqpoint{4.960000in}{3.696000in}}%
\pgfusepath{clip}%
\pgfsetrectcap%
\pgfsetroundjoin%
\pgfsetlinewidth{1.505625pt}%
\definecolor{currentstroke}{rgb}{0.580392,0.403922,0.741176}%
\pgfsetstrokecolor{currentstroke}%
\pgfsetdash{}{0pt}%
\pgfpathmoveto{\pgfqpoint{1.025455in}{0.696000in}}%
\pgfpathlineto{\pgfqpoint{3.748991in}{0.697054in}}%
\pgfpathlineto{\pgfqpoint{3.952408in}{0.699339in}}%
\pgfpathlineto{\pgfqpoint{4.088020in}{0.702891in}}%
\pgfpathlineto{\pgfqpoint{4.189729in}{0.707623in}}%
\pgfpathlineto{\pgfqpoint{4.268836in}{0.713254in}}%
\pgfpathlineto{\pgfqpoint{4.336642in}{0.720025in}}%
\pgfpathlineto{\pgfqpoint{4.393147in}{0.727494in}}%
\pgfpathlineto{\pgfqpoint{4.449651in}{0.737101in}}%
\pgfpathlineto{\pgfqpoint{4.494855in}{0.746698in}}%
\pgfpathlineto{\pgfqpoint{4.540059in}{0.758365in}}%
\pgfpathlineto{\pgfqpoint{4.585263in}{0.772514in}}%
\pgfpathlineto{\pgfqpoint{4.619166in}{0.785044in}}%
\pgfpathlineto{\pgfqpoint{4.653069in}{0.799479in}}%
\pgfpathlineto{\pgfqpoint{4.686972in}{0.816086in}}%
\pgfpathlineto{\pgfqpoint{4.720875in}{0.835167in}}%
\pgfpathlineto{\pgfqpoint{4.754778in}{0.857063in}}%
\pgfpathlineto{\pgfqpoint{4.788681in}{0.882158in}}%
\pgfpathlineto{\pgfqpoint{4.822584in}{0.910883in}}%
\pgfpathlineto{\pgfqpoint{4.856487in}{0.943724in}}%
\pgfpathlineto{\pgfqpoint{4.879089in}{0.968172in}}%
\pgfpathlineto{\pgfqpoint{4.901691in}{0.994868in}}%
\pgfpathlineto{\pgfqpoint{4.924293in}{1.024003in}}%
\pgfpathlineto{\pgfqpoint{4.946895in}{1.055786in}}%
\pgfpathlineto{\pgfqpoint{4.969496in}{1.090439in}}%
\pgfpathlineto{\pgfqpoint{4.992098in}{1.128202in}}%
\pgfpathlineto{\pgfqpoint{5.014700in}{1.169335in}}%
\pgfpathlineto{\pgfqpoint{5.037302in}{1.214116in}}%
\pgfpathlineto{\pgfqpoint{5.059904in}{1.262845in}}%
\pgfpathlineto{\pgfqpoint{5.082506in}{1.315846in}}%
\pgfpathlineto{\pgfqpoint{5.105108in}{1.373467in}}%
\pgfpathlineto{\pgfqpoint{5.127710in}{1.436080in}}%
\pgfpathlineto{\pgfqpoint{5.150312in}{1.504088in}}%
\pgfpathlineto{\pgfqpoint{5.172914in}{1.577922in}}%
\pgfpathlineto{\pgfqpoint{5.195516in}{1.658044in}}%
\pgfpathlineto{\pgfqpoint{5.218118in}{1.744952in}}%
\pgfpathlineto{\pgfqpoint{5.240720in}{1.839179in}}%
\pgfpathlineto{\pgfqpoint{5.263322in}{1.941298in}}%
\pgfpathlineto{\pgfqpoint{5.285924in}{2.051922in}}%
\pgfpathlineto{\pgfqpoint{5.308526in}{2.171708in}}%
\pgfpathlineto{\pgfqpoint{5.331128in}{2.301360in}}%
\pgfpathlineto{\pgfqpoint{5.353730in}{2.441635in}}%
\pgfpathlineto{\pgfqpoint{5.376332in}{2.593338in}}%
\pgfpathlineto{\pgfqpoint{5.398934in}{2.757334in}}%
\pgfpathlineto{\pgfqpoint{5.421536in}{2.934549in}}%
\pgfpathlineto{\pgfqpoint{5.444138in}{3.125971in}}%
\pgfpathlineto{\pgfqpoint{5.466740in}{3.332658in}}%
\pgfpathlineto{\pgfqpoint{5.489342in}{3.555740in}}%
\pgfpathlineto{\pgfqpoint{5.511943in}{3.796424in}}%
\pgfpathlineto{\pgfqpoint{5.534545in}{4.056000in}}%
\pgfpathlineto{\pgfqpoint{5.534545in}{4.056000in}}%
\pgfusepath{stroke}%
\end{pgfscope}%
\begin{pgfscope}%
\pgfpathrectangle{\pgfqpoint{0.800000in}{0.528000in}}{\pgfqpoint{4.960000in}{3.696000in}}%
\pgfusepath{clip}%
\pgfsetrectcap%
\pgfsetroundjoin%
\pgfsetlinewidth{1.505625pt}%
\definecolor{currentstroke}{rgb}{0.549020,0.337255,0.294118}%
\pgfsetstrokecolor{currentstroke}%
\pgfsetdash{}{0pt}%
\pgfpathmoveto{\pgfqpoint{1.025455in}{0.696000in}}%
\pgfpathlineto{\pgfqpoint{4.528758in}{0.697044in}}%
\pgfpathlineto{\pgfqpoint{4.653069in}{0.699187in}}%
\pgfpathlineto{\pgfqpoint{4.732176in}{0.702356in}}%
\pgfpathlineto{\pgfqpoint{4.799982in}{0.707353in}}%
\pgfpathlineto{\pgfqpoint{4.845186in}{0.712617in}}%
\pgfpathlineto{\pgfqpoint{4.890390in}{0.720213in}}%
\pgfpathlineto{\pgfqpoint{4.924293in}{0.728020in}}%
\pgfpathlineto{\pgfqpoint{4.958195in}{0.738242in}}%
\pgfpathlineto{\pgfqpoint{4.992098in}{0.751595in}}%
\pgfpathlineto{\pgfqpoint{5.014700in}{0.762680in}}%
\pgfpathlineto{\pgfqpoint{5.037302in}{0.775894in}}%
\pgfpathlineto{\pgfqpoint{5.059904in}{0.791629in}}%
\pgfpathlineto{\pgfqpoint{5.082506in}{0.810348in}}%
\pgfpathlineto{\pgfqpoint{5.105108in}{0.832596in}}%
\pgfpathlineto{\pgfqpoint{5.127710in}{0.859012in}}%
\pgfpathlineto{\pgfqpoint{5.150312in}{0.890347in}}%
\pgfpathlineto{\pgfqpoint{5.172914in}{0.927484in}}%
\pgfpathlineto{\pgfqpoint{5.195516in}{0.971455in}}%
\pgfpathlineto{\pgfqpoint{5.218118in}{1.023470in}}%
\pgfpathlineto{\pgfqpoint{5.240720in}{1.084946in}}%
\pgfpathlineto{\pgfqpoint{5.263322in}{1.157538in}}%
\pgfpathlineto{\pgfqpoint{5.285924in}{1.243180in}}%
\pgfpathlineto{\pgfqpoint{5.308526in}{1.344129in}}%
\pgfpathlineto{\pgfqpoint{5.331128in}{1.463018in}}%
\pgfpathlineto{\pgfqpoint{5.353730in}{1.602917in}}%
\pgfpathlineto{\pgfqpoint{5.376332in}{1.767396in}}%
\pgfpathlineto{\pgfqpoint{5.398934in}{1.960613in}}%
\pgfpathlineto{\pgfqpoint{5.421536in}{2.187399in}}%
\pgfpathlineto{\pgfqpoint{5.444138in}{2.453369in}}%
\pgfpathlineto{\pgfqpoint{5.466740in}{2.765037in}}%
\pgfpathlineto{\pgfqpoint{5.489342in}{3.129961in}}%
\pgfpathlineto{\pgfqpoint{5.511943in}{3.556901in}}%
\pgfpathlineto{\pgfqpoint{5.534545in}{4.056000in}}%
\pgfpathlineto{\pgfqpoint{5.534545in}{4.056000in}}%
\pgfusepath{stroke}%
\end{pgfscope}%
\begin{pgfscope}%
\pgfpathrectangle{\pgfqpoint{0.800000in}{0.528000in}}{\pgfqpoint{4.960000in}{3.696000in}}%
\pgfusepath{clip}%
\pgfsetbuttcap%
\pgfsetroundjoin%
\pgfsetlinewidth{1.505625pt}%
\definecolor{currentstroke}{rgb}{0.000000,0.000000,0.000000}%
\pgfsetstrokecolor{currentstroke}%
\pgfsetdash{{5.550000pt}{2.400000pt}}{0.000000pt}%
\pgfpathmoveto{\pgfqpoint{1.025455in}{0.696000in}}%
\pgfpathlineto{\pgfqpoint{5.523244in}{0.696000in}}%
\pgfpathlineto{\pgfqpoint{5.534545in}{4.056000in}}%
\pgfpathlineto{\pgfqpoint{5.534545in}{4.056000in}}%
\pgfusepath{stroke}%
\end{pgfscope}%
\begin{pgfscope}%
\pgfsetrectcap%
\pgfsetmiterjoin%
\pgfsetlinewidth{0.803000pt}%
\definecolor{currentstroke}{rgb}{0.000000,0.000000,0.000000}%
\pgfsetstrokecolor{currentstroke}%
\pgfsetdash{}{0pt}%
\pgfpathmoveto{\pgfqpoint{0.800000in}{0.528000in}}%
\pgfpathlineto{\pgfqpoint{0.800000in}{4.224000in}}%
\pgfusepath{stroke}%
\end{pgfscope}%
\begin{pgfscope}%
\pgfsetrectcap%
\pgfsetmiterjoin%
\pgfsetlinewidth{0.803000pt}%
\definecolor{currentstroke}{rgb}{0.000000,0.000000,0.000000}%
\pgfsetstrokecolor{currentstroke}%
\pgfsetdash{}{0pt}%
\pgfpathmoveto{\pgfqpoint{5.760000in}{0.528000in}}%
\pgfpathlineto{\pgfqpoint{5.760000in}{4.224000in}}%
\pgfusepath{stroke}%
\end{pgfscope}%
\begin{pgfscope}%
\pgfsetrectcap%
\pgfsetmiterjoin%
\pgfsetlinewidth{0.803000pt}%
\definecolor{currentstroke}{rgb}{0.000000,0.000000,0.000000}%
\pgfsetstrokecolor{currentstroke}%
\pgfsetdash{}{0pt}%
\pgfpathmoveto{\pgfqpoint{0.800000in}{0.528000in}}%
\pgfpathlineto{\pgfqpoint{5.760000in}{0.528000in}}%
\pgfusepath{stroke}%
\end{pgfscope}%
\begin{pgfscope}%
\pgfsetrectcap%
\pgfsetmiterjoin%
\pgfsetlinewidth{0.803000pt}%
\definecolor{currentstroke}{rgb}{0.000000,0.000000,0.000000}%
\pgfsetstrokecolor{currentstroke}%
\pgfsetdash{}{0pt}%
\pgfpathmoveto{\pgfqpoint{0.800000in}{4.224000in}}%
\pgfpathlineto{\pgfqpoint{5.760000in}{4.224000in}}%
\pgfusepath{stroke}%
\end{pgfscope}%
\begin{pgfscope}%
\definecolor{textcolor}{rgb}{0.000000,0.000000,0.000000}%
\pgfsetstrokecolor{textcolor}%
\pgfsetfillcolor{textcolor}%
\pgftext[x=3.280000in,y=4.307333in,,base]{\color{textcolor}\sffamily\fontsize{12.000000}{14.400000}\selectfont Pointwise Convergence of \(\displaystyle f_n(x) = x^n\)}%
\end{pgfscope}%
\begin{pgfscope}%
\pgfsetbuttcap%
\pgfsetmiterjoin%
\definecolor{currentfill}{rgb}{1.000000,1.000000,1.000000}%
\pgfsetfillcolor{currentfill}%
\pgfsetfillopacity{0.800000}%
\pgfsetlinewidth{1.003750pt}%
\definecolor{currentstroke}{rgb}{0.800000,0.800000,0.800000}%
\pgfsetstrokecolor{currentstroke}%
\pgfsetstrokeopacity{0.800000}%
\pgfsetdash{}{0pt}%
\pgfpathmoveto{\pgfqpoint{0.897222in}{2.685888in}}%
\pgfpathlineto{\pgfqpoint{2.297477in}{2.685888in}}%
\pgfpathquadraticcurveto{\pgfqpoint{2.325255in}{2.685888in}}{\pgfqpoint{2.325255in}{2.713666in}}%
\pgfpathlineto{\pgfqpoint{2.325255in}{4.126778in}}%
\pgfpathquadraticcurveto{\pgfqpoint{2.325255in}{4.154556in}}{\pgfqpoint{2.297477in}{4.154556in}}%
\pgfpathlineto{\pgfqpoint{0.897222in}{4.154556in}}%
\pgfpathquadraticcurveto{\pgfqpoint{0.869444in}{4.154556in}}{\pgfqpoint{0.869444in}{4.126778in}}%
\pgfpathlineto{\pgfqpoint{0.869444in}{2.713666in}}%
\pgfpathquadraticcurveto{\pgfqpoint{0.869444in}{2.685888in}}{\pgfqpoint{0.897222in}{2.685888in}}%
\pgfpathlineto{\pgfqpoint{0.897222in}{2.685888in}}%
\pgfpathclose%
\pgfusepath{stroke,fill}%
\end{pgfscope}%
\begin{pgfscope}%
\pgfsetrectcap%
\pgfsetroundjoin%
\pgfsetlinewidth{1.505625pt}%
\definecolor{currentstroke}{rgb}{0.121569,0.466667,0.705882}%
\pgfsetstrokecolor{currentstroke}%
\pgfsetdash{}{0pt}%
\pgfpathmoveto{\pgfqpoint{0.925000in}{4.042088in}}%
\pgfpathlineto{\pgfqpoint{1.063889in}{4.042088in}}%
\pgfpathlineto{\pgfqpoint{1.202778in}{4.042088in}}%
\pgfusepath{stroke}%
\end{pgfscope}%
\begin{pgfscope}%
\definecolor{textcolor}{rgb}{0.000000,0.000000,0.000000}%
\pgfsetstrokecolor{textcolor}%
\pgfsetfillcolor{textcolor}%
\pgftext[x=1.313889in,y=3.993477in,left,base]{\color{textcolor}\sffamily\fontsize{10.000000}{12.000000}\selectfont n = 1}%
\end{pgfscope}%
\begin{pgfscope}%
\pgfsetrectcap%
\pgfsetroundjoin%
\pgfsetlinewidth{1.505625pt}%
\definecolor{currentstroke}{rgb}{1.000000,0.498039,0.054902}%
\pgfsetstrokecolor{currentstroke}%
\pgfsetdash{}{0pt}%
\pgfpathmoveto{\pgfqpoint{0.925000in}{3.838231in}}%
\pgfpathlineto{\pgfqpoint{1.063889in}{3.838231in}}%
\pgfpathlineto{\pgfqpoint{1.202778in}{3.838231in}}%
\pgfusepath{stroke}%
\end{pgfscope}%
\begin{pgfscope}%
\definecolor{textcolor}{rgb}{0.000000,0.000000,0.000000}%
\pgfsetstrokecolor{textcolor}%
\pgfsetfillcolor{textcolor}%
\pgftext[x=1.313889in,y=3.789620in,left,base]{\color{textcolor}\sffamily\fontsize{10.000000}{12.000000}\selectfont n = 2}%
\end{pgfscope}%
\begin{pgfscope}%
\pgfsetrectcap%
\pgfsetroundjoin%
\pgfsetlinewidth{1.505625pt}%
\definecolor{currentstroke}{rgb}{0.172549,0.627451,0.172549}%
\pgfsetstrokecolor{currentstroke}%
\pgfsetdash{}{0pt}%
\pgfpathmoveto{\pgfqpoint{0.925000in}{3.634374in}}%
\pgfpathlineto{\pgfqpoint{1.063889in}{3.634374in}}%
\pgfpathlineto{\pgfqpoint{1.202778in}{3.634374in}}%
\pgfusepath{stroke}%
\end{pgfscope}%
\begin{pgfscope}%
\definecolor{textcolor}{rgb}{0.000000,0.000000,0.000000}%
\pgfsetstrokecolor{textcolor}%
\pgfsetfillcolor{textcolor}%
\pgftext[x=1.313889in,y=3.585762in,left,base]{\color{textcolor}\sffamily\fontsize{10.000000}{12.000000}\selectfont n = 4}%
\end{pgfscope}%
\begin{pgfscope}%
\pgfsetrectcap%
\pgfsetroundjoin%
\pgfsetlinewidth{1.505625pt}%
\definecolor{currentstroke}{rgb}{0.839216,0.152941,0.156863}%
\pgfsetstrokecolor{currentstroke}%
\pgfsetdash{}{0pt}%
\pgfpathmoveto{\pgfqpoint{0.925000in}{3.430516in}}%
\pgfpathlineto{\pgfqpoint{1.063889in}{3.430516in}}%
\pgfpathlineto{\pgfqpoint{1.202778in}{3.430516in}}%
\pgfusepath{stroke}%
\end{pgfscope}%
\begin{pgfscope}%
\definecolor{textcolor}{rgb}{0.000000,0.000000,0.000000}%
\pgfsetstrokecolor{textcolor}%
\pgfsetfillcolor{textcolor}%
\pgftext[x=1.313889in,y=3.381905in,left,base]{\color{textcolor}\sffamily\fontsize{10.000000}{12.000000}\selectfont n = 8}%
\end{pgfscope}%
\begin{pgfscope}%
\pgfsetrectcap%
\pgfsetroundjoin%
\pgfsetlinewidth{1.505625pt}%
\definecolor{currentstroke}{rgb}{0.580392,0.403922,0.741176}%
\pgfsetstrokecolor{currentstroke}%
\pgfsetdash{}{0pt}%
\pgfpathmoveto{\pgfqpoint{0.925000in}{3.226659in}}%
\pgfpathlineto{\pgfqpoint{1.063889in}{3.226659in}}%
\pgfpathlineto{\pgfqpoint{1.202778in}{3.226659in}}%
\pgfusepath{stroke}%
\end{pgfscope}%
\begin{pgfscope}%
\definecolor{textcolor}{rgb}{0.000000,0.000000,0.000000}%
\pgfsetstrokecolor{textcolor}%
\pgfsetfillcolor{textcolor}%
\pgftext[x=1.313889in,y=3.178048in,left,base]{\color{textcolor}\sffamily\fontsize{10.000000}{12.000000}\selectfont n = 16}%
\end{pgfscope}%
\begin{pgfscope}%
\pgfsetrectcap%
\pgfsetroundjoin%
\pgfsetlinewidth{1.505625pt}%
\definecolor{currentstroke}{rgb}{0.549020,0.337255,0.294118}%
\pgfsetstrokecolor{currentstroke}%
\pgfsetdash{}{0pt}%
\pgfpathmoveto{\pgfqpoint{0.925000in}{3.022802in}}%
\pgfpathlineto{\pgfqpoint{1.063889in}{3.022802in}}%
\pgfpathlineto{\pgfqpoint{1.202778in}{3.022802in}}%
\pgfusepath{stroke}%
\end{pgfscope}%
\begin{pgfscope}%
\definecolor{textcolor}{rgb}{0.000000,0.000000,0.000000}%
\pgfsetstrokecolor{textcolor}%
\pgfsetfillcolor{textcolor}%
\pgftext[x=1.313889in,y=2.974191in,left,base]{\color{textcolor}\sffamily\fontsize{10.000000}{12.000000}\selectfont n = 32}%
\end{pgfscope}%
\begin{pgfscope}%
\pgfsetbuttcap%
\pgfsetroundjoin%
\pgfsetlinewidth{1.505625pt}%
\definecolor{currentstroke}{rgb}{0.000000,0.000000,0.000000}%
\pgfsetstrokecolor{currentstroke}%
\pgfsetdash{{5.550000pt}{2.400000pt}}{0.000000pt}%
\pgfpathmoveto{\pgfqpoint{0.925000in}{2.818945in}}%
\pgfpathlineto{\pgfqpoint{1.063889in}{2.818945in}}%
\pgfpathlineto{\pgfqpoint{1.202778in}{2.818945in}}%
\pgfusepath{stroke}%
\end{pgfscope}%
\begin{pgfscope}%
\definecolor{textcolor}{rgb}{0.000000,0.000000,0.000000}%
\pgfsetstrokecolor{textcolor}%
\pgfsetfillcolor{textcolor}%
\pgftext[x=1.313889in,y=2.770334in,left,base]{\color{textcolor}\sffamily\fontsize{10.000000}{12.000000}\selectfont Limit function}%
\end{pgfscope}%
\end{pgfpicture}%
\makeatother%
\endgroup%
}
  \vspace{-16pt}
  \caption{A sequence of converging functions}
    % \scalebox{0.4}{%% Creator: Matplotlib, PGF backend
%%
%% To include the figure in your LaTeX document, write
%%   \input{<filename>.pgf}
%%
%% Make sure the required packages are loaded in your preamble
%%   \usepackage{pgf}
%%
%% Also ensure that all the required font packages are loaded; for instance,
%% the lmodern package is sometimes necessary when using math font.
%%   \usepackage{lmodern}
%%
%% Figures using additional raster images can only be included by \input if
%% they are in the same directory as the main LaTeX file. For loading figures
%% from other directories you can use the `import` package
%%   \usepackage{import}
%%
%% and then include the figures with
%%   \import{<path to file>}{<filename>.pgf}
%%
%% Matplotlib used the following preamble
%%   
%%   \usepackage{fontspec}
%%   \setmainfont{DejaVuSerif.ttf}[Path=\detokenize{/usr/local/Anaconda3-2023.07-1/lib/python3.11/site-packages/matplotlib/mpl-data/fonts/ttf/}]
%%   \setsansfont{DejaVuSans.ttf}[Path=\detokenize{/usr/local/Anaconda3-2023.07-1/lib/python3.11/site-packages/matplotlib/mpl-data/fonts/ttf/}]
%%   \setmonofont{DejaVuSansMono.ttf}[Path=\detokenize{/usr/local/Anaconda3-2023.07-1/lib/python3.11/site-packages/matplotlib/mpl-data/fonts/ttf/}]
%%   \makeatletter\@ifpackageloaded{underscore}{}{\usepackage[strings]{underscore}}\makeatother
%%
\begingroup%
\makeatletter%
\begin{pgfpicture}%
\pgfpathrectangle{\pgfpointorigin}{\pgfqpoint{10.000000in}{6.000000in}}%
\pgfusepath{use as bounding box, clip}%
\begin{pgfscope}%
\pgfsetbuttcap%
\pgfsetmiterjoin%
\definecolor{currentfill}{rgb}{1.000000,1.000000,1.000000}%
\pgfsetfillcolor{currentfill}%
\pgfsetlinewidth{0.000000pt}%
\definecolor{currentstroke}{rgb}{1.000000,1.000000,1.000000}%
\pgfsetstrokecolor{currentstroke}%
\pgfsetdash{}{0pt}%
\pgfpathmoveto{\pgfqpoint{0.000000in}{0.000000in}}%
\pgfpathlineto{\pgfqpoint{10.000000in}{0.000000in}}%
\pgfpathlineto{\pgfqpoint{10.000000in}{6.000000in}}%
\pgfpathlineto{\pgfqpoint{0.000000in}{6.000000in}}%
\pgfpathlineto{\pgfqpoint{0.000000in}{0.000000in}}%
\pgfpathclose%
\pgfusepath{fill}%
\end{pgfscope}%
\begin{pgfscope}%
\pgfsetbuttcap%
\pgfsetmiterjoin%
\definecolor{currentfill}{rgb}{1.000000,1.000000,1.000000}%
\pgfsetfillcolor{currentfill}%
\pgfsetlinewidth{0.000000pt}%
\definecolor{currentstroke}{rgb}{0.000000,0.000000,0.000000}%
\pgfsetstrokecolor{currentstroke}%
\pgfsetstrokeopacity{0.000000}%
\pgfsetdash{}{0pt}%
\pgfpathmoveto{\pgfqpoint{1.250000in}{0.660000in}}%
\pgfpathlineto{\pgfqpoint{9.000000in}{0.660000in}}%
\pgfpathlineto{\pgfqpoint{9.000000in}{5.280000in}}%
\pgfpathlineto{\pgfqpoint{1.250000in}{5.280000in}}%
\pgfpathlineto{\pgfqpoint{1.250000in}{0.660000in}}%
\pgfpathclose%
\pgfusepath{fill}%
\end{pgfscope}%
\begin{pgfscope}%
\pgfpathrectangle{\pgfqpoint{1.250000in}{0.660000in}}{\pgfqpoint{7.750000in}{4.620000in}}%
\pgfusepath{clip}%
\pgfsetrectcap%
\pgfsetroundjoin%
\pgfsetlinewidth{0.803000pt}%
\definecolor{currentstroke}{rgb}{0.690196,0.690196,0.690196}%
\pgfsetstrokecolor{currentstroke}%
\pgfsetdash{}{0pt}%
\pgfpathmoveto{\pgfqpoint{1.602273in}{0.660000in}}%
\pgfpathlineto{\pgfqpoint{1.602273in}{5.280000in}}%
\pgfusepath{stroke}%
\end{pgfscope}%
\begin{pgfscope}%
\pgfsetbuttcap%
\pgfsetroundjoin%
\definecolor{currentfill}{rgb}{0.000000,0.000000,0.000000}%
\pgfsetfillcolor{currentfill}%
\pgfsetlinewidth{0.803000pt}%
\definecolor{currentstroke}{rgb}{0.000000,0.000000,0.000000}%
\pgfsetstrokecolor{currentstroke}%
\pgfsetdash{}{0pt}%
\pgfsys@defobject{currentmarker}{\pgfqpoint{0.000000in}{-0.048611in}}{\pgfqpoint{0.000000in}{0.000000in}}{%
\pgfpathmoveto{\pgfqpoint{0.000000in}{0.000000in}}%
\pgfpathlineto{\pgfqpoint{0.000000in}{-0.048611in}}%
\pgfusepath{stroke,fill}%
}%
\begin{pgfscope}%
\pgfsys@transformshift{1.602273in}{0.660000in}%
\pgfsys@useobject{currentmarker}{}%
\end{pgfscope}%
\end{pgfscope}%
\begin{pgfscope}%
\definecolor{textcolor}{rgb}{0.000000,0.000000,0.000000}%
\pgfsetstrokecolor{textcolor}%
\pgfsetfillcolor{textcolor}%
\pgftext[x=1.602273in,y=0.562778in,,top]{\color{textcolor}\sffamily\fontsize{10.000000}{12.000000}\selectfont 0.0}%
\end{pgfscope}%
\begin{pgfscope}%
\pgfpathrectangle{\pgfqpoint{1.250000in}{0.660000in}}{\pgfqpoint{7.750000in}{4.620000in}}%
\pgfusepath{clip}%
\pgfsetrectcap%
\pgfsetroundjoin%
\pgfsetlinewidth{0.803000pt}%
\definecolor{currentstroke}{rgb}{0.690196,0.690196,0.690196}%
\pgfsetstrokecolor{currentstroke}%
\pgfsetdash{}{0pt}%
\pgfpathmoveto{\pgfqpoint{2.723592in}{0.660000in}}%
\pgfpathlineto{\pgfqpoint{2.723592in}{5.280000in}}%
\pgfusepath{stroke}%
\end{pgfscope}%
\begin{pgfscope}%
\pgfsetbuttcap%
\pgfsetroundjoin%
\definecolor{currentfill}{rgb}{0.000000,0.000000,0.000000}%
\pgfsetfillcolor{currentfill}%
\pgfsetlinewidth{0.803000pt}%
\definecolor{currentstroke}{rgb}{0.000000,0.000000,0.000000}%
\pgfsetstrokecolor{currentstroke}%
\pgfsetdash{}{0pt}%
\pgfsys@defobject{currentmarker}{\pgfqpoint{0.000000in}{-0.048611in}}{\pgfqpoint{0.000000in}{0.000000in}}{%
\pgfpathmoveto{\pgfqpoint{0.000000in}{0.000000in}}%
\pgfpathlineto{\pgfqpoint{0.000000in}{-0.048611in}}%
\pgfusepath{stroke,fill}%
}%
\begin{pgfscope}%
\pgfsys@transformshift{2.723592in}{0.660000in}%
\pgfsys@useobject{currentmarker}{}%
\end{pgfscope}%
\end{pgfscope}%
\begin{pgfscope}%
\definecolor{textcolor}{rgb}{0.000000,0.000000,0.000000}%
\pgfsetstrokecolor{textcolor}%
\pgfsetfillcolor{textcolor}%
\pgftext[x=2.723592in,y=0.562778in,,top]{\color{textcolor}\sffamily\fontsize{10.000000}{12.000000}\selectfont 0.5}%
\end{pgfscope}%
\begin{pgfscope}%
\pgfpathrectangle{\pgfqpoint{1.250000in}{0.660000in}}{\pgfqpoint{7.750000in}{4.620000in}}%
\pgfusepath{clip}%
\pgfsetrectcap%
\pgfsetroundjoin%
\pgfsetlinewidth{0.803000pt}%
\definecolor{currentstroke}{rgb}{0.690196,0.690196,0.690196}%
\pgfsetstrokecolor{currentstroke}%
\pgfsetdash{}{0pt}%
\pgfpathmoveto{\pgfqpoint{3.844911in}{0.660000in}}%
\pgfpathlineto{\pgfqpoint{3.844911in}{5.280000in}}%
\pgfusepath{stroke}%
\end{pgfscope}%
\begin{pgfscope}%
\pgfsetbuttcap%
\pgfsetroundjoin%
\definecolor{currentfill}{rgb}{0.000000,0.000000,0.000000}%
\pgfsetfillcolor{currentfill}%
\pgfsetlinewidth{0.803000pt}%
\definecolor{currentstroke}{rgb}{0.000000,0.000000,0.000000}%
\pgfsetstrokecolor{currentstroke}%
\pgfsetdash{}{0pt}%
\pgfsys@defobject{currentmarker}{\pgfqpoint{0.000000in}{-0.048611in}}{\pgfqpoint{0.000000in}{0.000000in}}{%
\pgfpathmoveto{\pgfqpoint{0.000000in}{0.000000in}}%
\pgfpathlineto{\pgfqpoint{0.000000in}{-0.048611in}}%
\pgfusepath{stroke,fill}%
}%
\begin{pgfscope}%
\pgfsys@transformshift{3.844911in}{0.660000in}%
\pgfsys@useobject{currentmarker}{}%
\end{pgfscope}%
\end{pgfscope}%
\begin{pgfscope}%
\definecolor{textcolor}{rgb}{0.000000,0.000000,0.000000}%
\pgfsetstrokecolor{textcolor}%
\pgfsetfillcolor{textcolor}%
\pgftext[x=3.844911in,y=0.562778in,,top]{\color{textcolor}\sffamily\fontsize{10.000000}{12.000000}\selectfont 1.0}%
\end{pgfscope}%
\begin{pgfscope}%
\pgfpathrectangle{\pgfqpoint{1.250000in}{0.660000in}}{\pgfqpoint{7.750000in}{4.620000in}}%
\pgfusepath{clip}%
\pgfsetrectcap%
\pgfsetroundjoin%
\pgfsetlinewidth{0.803000pt}%
\definecolor{currentstroke}{rgb}{0.690196,0.690196,0.690196}%
\pgfsetstrokecolor{currentstroke}%
\pgfsetdash{}{0pt}%
\pgfpathmoveto{\pgfqpoint{4.966229in}{0.660000in}}%
\pgfpathlineto{\pgfqpoint{4.966229in}{5.280000in}}%
\pgfusepath{stroke}%
\end{pgfscope}%
\begin{pgfscope}%
\pgfsetbuttcap%
\pgfsetroundjoin%
\definecolor{currentfill}{rgb}{0.000000,0.000000,0.000000}%
\pgfsetfillcolor{currentfill}%
\pgfsetlinewidth{0.803000pt}%
\definecolor{currentstroke}{rgb}{0.000000,0.000000,0.000000}%
\pgfsetstrokecolor{currentstroke}%
\pgfsetdash{}{0pt}%
\pgfsys@defobject{currentmarker}{\pgfqpoint{0.000000in}{-0.048611in}}{\pgfqpoint{0.000000in}{0.000000in}}{%
\pgfpathmoveto{\pgfqpoint{0.000000in}{0.000000in}}%
\pgfpathlineto{\pgfqpoint{0.000000in}{-0.048611in}}%
\pgfusepath{stroke,fill}%
}%
\begin{pgfscope}%
\pgfsys@transformshift{4.966229in}{0.660000in}%
\pgfsys@useobject{currentmarker}{}%
\end{pgfscope}%
\end{pgfscope}%
\begin{pgfscope}%
\definecolor{textcolor}{rgb}{0.000000,0.000000,0.000000}%
\pgfsetstrokecolor{textcolor}%
\pgfsetfillcolor{textcolor}%
\pgftext[x=4.966229in,y=0.562778in,,top]{\color{textcolor}\sffamily\fontsize{10.000000}{12.000000}\selectfont 1.5}%
\end{pgfscope}%
\begin{pgfscope}%
\pgfpathrectangle{\pgfqpoint{1.250000in}{0.660000in}}{\pgfqpoint{7.750000in}{4.620000in}}%
\pgfusepath{clip}%
\pgfsetrectcap%
\pgfsetroundjoin%
\pgfsetlinewidth{0.803000pt}%
\definecolor{currentstroke}{rgb}{0.690196,0.690196,0.690196}%
\pgfsetstrokecolor{currentstroke}%
\pgfsetdash{}{0pt}%
\pgfpathmoveto{\pgfqpoint{6.087548in}{0.660000in}}%
\pgfpathlineto{\pgfqpoint{6.087548in}{5.280000in}}%
\pgfusepath{stroke}%
\end{pgfscope}%
\begin{pgfscope}%
\pgfsetbuttcap%
\pgfsetroundjoin%
\definecolor{currentfill}{rgb}{0.000000,0.000000,0.000000}%
\pgfsetfillcolor{currentfill}%
\pgfsetlinewidth{0.803000pt}%
\definecolor{currentstroke}{rgb}{0.000000,0.000000,0.000000}%
\pgfsetstrokecolor{currentstroke}%
\pgfsetdash{}{0pt}%
\pgfsys@defobject{currentmarker}{\pgfqpoint{0.000000in}{-0.048611in}}{\pgfqpoint{0.000000in}{0.000000in}}{%
\pgfpathmoveto{\pgfqpoint{0.000000in}{0.000000in}}%
\pgfpathlineto{\pgfqpoint{0.000000in}{-0.048611in}}%
\pgfusepath{stroke,fill}%
}%
\begin{pgfscope}%
\pgfsys@transformshift{6.087548in}{0.660000in}%
\pgfsys@useobject{currentmarker}{}%
\end{pgfscope}%
\end{pgfscope}%
\begin{pgfscope}%
\definecolor{textcolor}{rgb}{0.000000,0.000000,0.000000}%
\pgfsetstrokecolor{textcolor}%
\pgfsetfillcolor{textcolor}%
\pgftext[x=6.087548in,y=0.562778in,,top]{\color{textcolor}\sffamily\fontsize{10.000000}{12.000000}\selectfont 2.0}%
\end{pgfscope}%
\begin{pgfscope}%
\pgfpathrectangle{\pgfqpoint{1.250000in}{0.660000in}}{\pgfqpoint{7.750000in}{4.620000in}}%
\pgfusepath{clip}%
\pgfsetrectcap%
\pgfsetroundjoin%
\pgfsetlinewidth{0.803000pt}%
\definecolor{currentstroke}{rgb}{0.690196,0.690196,0.690196}%
\pgfsetstrokecolor{currentstroke}%
\pgfsetdash{}{0pt}%
\pgfpathmoveto{\pgfqpoint{7.208867in}{0.660000in}}%
\pgfpathlineto{\pgfqpoint{7.208867in}{5.280000in}}%
\pgfusepath{stroke}%
\end{pgfscope}%
\begin{pgfscope}%
\pgfsetbuttcap%
\pgfsetroundjoin%
\definecolor{currentfill}{rgb}{0.000000,0.000000,0.000000}%
\pgfsetfillcolor{currentfill}%
\pgfsetlinewidth{0.803000pt}%
\definecolor{currentstroke}{rgb}{0.000000,0.000000,0.000000}%
\pgfsetstrokecolor{currentstroke}%
\pgfsetdash{}{0pt}%
\pgfsys@defobject{currentmarker}{\pgfqpoint{0.000000in}{-0.048611in}}{\pgfqpoint{0.000000in}{0.000000in}}{%
\pgfpathmoveto{\pgfqpoint{0.000000in}{0.000000in}}%
\pgfpathlineto{\pgfqpoint{0.000000in}{-0.048611in}}%
\pgfusepath{stroke,fill}%
}%
\begin{pgfscope}%
\pgfsys@transformshift{7.208867in}{0.660000in}%
\pgfsys@useobject{currentmarker}{}%
\end{pgfscope}%
\end{pgfscope}%
\begin{pgfscope}%
\definecolor{textcolor}{rgb}{0.000000,0.000000,0.000000}%
\pgfsetstrokecolor{textcolor}%
\pgfsetfillcolor{textcolor}%
\pgftext[x=7.208867in,y=0.562778in,,top]{\color{textcolor}\sffamily\fontsize{10.000000}{12.000000}\selectfont 2.5}%
\end{pgfscope}%
\begin{pgfscope}%
\pgfpathrectangle{\pgfqpoint{1.250000in}{0.660000in}}{\pgfqpoint{7.750000in}{4.620000in}}%
\pgfusepath{clip}%
\pgfsetrectcap%
\pgfsetroundjoin%
\pgfsetlinewidth{0.803000pt}%
\definecolor{currentstroke}{rgb}{0.690196,0.690196,0.690196}%
\pgfsetstrokecolor{currentstroke}%
\pgfsetdash{}{0pt}%
\pgfpathmoveto{\pgfqpoint{8.330186in}{0.660000in}}%
\pgfpathlineto{\pgfqpoint{8.330186in}{5.280000in}}%
\pgfusepath{stroke}%
\end{pgfscope}%
\begin{pgfscope}%
\pgfsetbuttcap%
\pgfsetroundjoin%
\definecolor{currentfill}{rgb}{0.000000,0.000000,0.000000}%
\pgfsetfillcolor{currentfill}%
\pgfsetlinewidth{0.803000pt}%
\definecolor{currentstroke}{rgb}{0.000000,0.000000,0.000000}%
\pgfsetstrokecolor{currentstroke}%
\pgfsetdash{}{0pt}%
\pgfsys@defobject{currentmarker}{\pgfqpoint{0.000000in}{-0.048611in}}{\pgfqpoint{0.000000in}{0.000000in}}{%
\pgfpathmoveto{\pgfqpoint{0.000000in}{0.000000in}}%
\pgfpathlineto{\pgfqpoint{0.000000in}{-0.048611in}}%
\pgfusepath{stroke,fill}%
}%
\begin{pgfscope}%
\pgfsys@transformshift{8.330186in}{0.660000in}%
\pgfsys@useobject{currentmarker}{}%
\end{pgfscope}%
\end{pgfscope}%
\begin{pgfscope}%
\definecolor{textcolor}{rgb}{0.000000,0.000000,0.000000}%
\pgfsetstrokecolor{textcolor}%
\pgfsetfillcolor{textcolor}%
\pgftext[x=8.330186in,y=0.562778in,,top]{\color{textcolor}\sffamily\fontsize{10.000000}{12.000000}\selectfont 3.0}%
\end{pgfscope}%
\begin{pgfscope}%
\definecolor{textcolor}{rgb}{0.000000,0.000000,0.000000}%
\pgfsetstrokecolor{textcolor}%
\pgfsetfillcolor{textcolor}%
\pgftext[x=5.125000in,y=0.372809in,,top]{\color{textcolor}\sffamily\fontsize{10.000000}{12.000000}\selectfont x}%
\end{pgfscope}%
\begin{pgfscope}%
\pgfpathrectangle{\pgfqpoint{1.250000in}{0.660000in}}{\pgfqpoint{7.750000in}{4.620000in}}%
\pgfusepath{clip}%
\pgfsetrectcap%
\pgfsetroundjoin%
\pgfsetlinewidth{0.803000pt}%
\definecolor{currentstroke}{rgb}{0.690196,0.690196,0.690196}%
\pgfsetstrokecolor{currentstroke}%
\pgfsetdash{}{0pt}%
\pgfpathmoveto{\pgfqpoint{1.250000in}{0.870000in}}%
\pgfpathlineto{\pgfqpoint{9.000000in}{0.870000in}}%
\pgfusepath{stroke}%
\end{pgfscope}%
\begin{pgfscope}%
\pgfsetbuttcap%
\pgfsetroundjoin%
\definecolor{currentfill}{rgb}{0.000000,0.000000,0.000000}%
\pgfsetfillcolor{currentfill}%
\pgfsetlinewidth{0.803000pt}%
\definecolor{currentstroke}{rgb}{0.000000,0.000000,0.000000}%
\pgfsetstrokecolor{currentstroke}%
\pgfsetdash{}{0pt}%
\pgfsys@defobject{currentmarker}{\pgfqpoint{-0.048611in}{0.000000in}}{\pgfqpoint{-0.000000in}{0.000000in}}{%
\pgfpathmoveto{\pgfqpoint{-0.000000in}{0.000000in}}%
\pgfpathlineto{\pgfqpoint{-0.048611in}{0.000000in}}%
\pgfusepath{stroke,fill}%
}%
\begin{pgfscope}%
\pgfsys@transformshift{1.250000in}{0.870000in}%
\pgfsys@useobject{currentmarker}{}%
\end{pgfscope}%
\end{pgfscope}%
\begin{pgfscope}%
\definecolor{textcolor}{rgb}{0.000000,0.000000,0.000000}%
\pgfsetstrokecolor{textcolor}%
\pgfsetfillcolor{textcolor}%
\pgftext[x=1.064412in, y=0.817238in, left, base]{\color{textcolor}\sffamily\fontsize{10.000000}{12.000000}\selectfont 0}%
\end{pgfscope}%
\begin{pgfscope}%
\pgfpathrectangle{\pgfqpoint{1.250000in}{0.660000in}}{\pgfqpoint{7.750000in}{4.620000in}}%
\pgfusepath{clip}%
\pgfsetrectcap%
\pgfsetroundjoin%
\pgfsetlinewidth{0.803000pt}%
\definecolor{currentstroke}{rgb}{0.690196,0.690196,0.690196}%
\pgfsetstrokecolor{currentstroke}%
\pgfsetdash{}{0pt}%
\pgfpathmoveto{\pgfqpoint{1.250000in}{1.721098in}}%
\pgfpathlineto{\pgfqpoint{9.000000in}{1.721098in}}%
\pgfusepath{stroke}%
\end{pgfscope}%
\begin{pgfscope}%
\pgfsetbuttcap%
\pgfsetroundjoin%
\definecolor{currentfill}{rgb}{0.000000,0.000000,0.000000}%
\pgfsetfillcolor{currentfill}%
\pgfsetlinewidth{0.803000pt}%
\definecolor{currentstroke}{rgb}{0.000000,0.000000,0.000000}%
\pgfsetstrokecolor{currentstroke}%
\pgfsetdash{}{0pt}%
\pgfsys@defobject{currentmarker}{\pgfqpoint{-0.048611in}{0.000000in}}{\pgfqpoint{-0.000000in}{0.000000in}}{%
\pgfpathmoveto{\pgfqpoint{-0.000000in}{0.000000in}}%
\pgfpathlineto{\pgfqpoint{-0.048611in}{0.000000in}}%
\pgfusepath{stroke,fill}%
}%
\begin{pgfscope}%
\pgfsys@transformshift{1.250000in}{1.721098in}%
\pgfsys@useobject{currentmarker}{}%
\end{pgfscope}%
\end{pgfscope}%
\begin{pgfscope}%
\definecolor{textcolor}{rgb}{0.000000,0.000000,0.000000}%
\pgfsetstrokecolor{textcolor}%
\pgfsetfillcolor{textcolor}%
\pgftext[x=1.064412in, y=1.668336in, left, base]{\color{textcolor}\sffamily\fontsize{10.000000}{12.000000}\selectfont 1}%
\end{pgfscope}%
\begin{pgfscope}%
\pgfpathrectangle{\pgfqpoint{1.250000in}{0.660000in}}{\pgfqpoint{7.750000in}{4.620000in}}%
\pgfusepath{clip}%
\pgfsetrectcap%
\pgfsetroundjoin%
\pgfsetlinewidth{0.803000pt}%
\definecolor{currentstroke}{rgb}{0.690196,0.690196,0.690196}%
\pgfsetstrokecolor{currentstroke}%
\pgfsetdash{}{0pt}%
\pgfpathmoveto{\pgfqpoint{1.250000in}{2.572196in}}%
\pgfpathlineto{\pgfqpoint{9.000000in}{2.572196in}}%
\pgfusepath{stroke}%
\end{pgfscope}%
\begin{pgfscope}%
\pgfsetbuttcap%
\pgfsetroundjoin%
\definecolor{currentfill}{rgb}{0.000000,0.000000,0.000000}%
\pgfsetfillcolor{currentfill}%
\pgfsetlinewidth{0.803000pt}%
\definecolor{currentstroke}{rgb}{0.000000,0.000000,0.000000}%
\pgfsetstrokecolor{currentstroke}%
\pgfsetdash{}{0pt}%
\pgfsys@defobject{currentmarker}{\pgfqpoint{-0.048611in}{0.000000in}}{\pgfqpoint{-0.000000in}{0.000000in}}{%
\pgfpathmoveto{\pgfqpoint{-0.000000in}{0.000000in}}%
\pgfpathlineto{\pgfqpoint{-0.048611in}{0.000000in}}%
\pgfusepath{stroke,fill}%
}%
\begin{pgfscope}%
\pgfsys@transformshift{1.250000in}{2.572196in}%
\pgfsys@useobject{currentmarker}{}%
\end{pgfscope}%
\end{pgfscope}%
\begin{pgfscope}%
\definecolor{textcolor}{rgb}{0.000000,0.000000,0.000000}%
\pgfsetstrokecolor{textcolor}%
\pgfsetfillcolor{textcolor}%
\pgftext[x=1.064412in, y=2.519434in, left, base]{\color{textcolor}\sffamily\fontsize{10.000000}{12.000000}\selectfont 2}%
\end{pgfscope}%
\begin{pgfscope}%
\pgfpathrectangle{\pgfqpoint{1.250000in}{0.660000in}}{\pgfqpoint{7.750000in}{4.620000in}}%
\pgfusepath{clip}%
\pgfsetrectcap%
\pgfsetroundjoin%
\pgfsetlinewidth{0.803000pt}%
\definecolor{currentstroke}{rgb}{0.690196,0.690196,0.690196}%
\pgfsetstrokecolor{currentstroke}%
\pgfsetdash{}{0pt}%
\pgfpathmoveto{\pgfqpoint{1.250000in}{3.423294in}}%
\pgfpathlineto{\pgfqpoint{9.000000in}{3.423294in}}%
\pgfusepath{stroke}%
\end{pgfscope}%
\begin{pgfscope}%
\pgfsetbuttcap%
\pgfsetroundjoin%
\definecolor{currentfill}{rgb}{0.000000,0.000000,0.000000}%
\pgfsetfillcolor{currentfill}%
\pgfsetlinewidth{0.803000pt}%
\definecolor{currentstroke}{rgb}{0.000000,0.000000,0.000000}%
\pgfsetstrokecolor{currentstroke}%
\pgfsetdash{}{0pt}%
\pgfsys@defobject{currentmarker}{\pgfqpoint{-0.048611in}{0.000000in}}{\pgfqpoint{-0.000000in}{0.000000in}}{%
\pgfpathmoveto{\pgfqpoint{-0.000000in}{0.000000in}}%
\pgfpathlineto{\pgfqpoint{-0.048611in}{0.000000in}}%
\pgfusepath{stroke,fill}%
}%
\begin{pgfscope}%
\pgfsys@transformshift{1.250000in}{3.423294in}%
\pgfsys@useobject{currentmarker}{}%
\end{pgfscope}%
\end{pgfscope}%
\begin{pgfscope}%
\definecolor{textcolor}{rgb}{0.000000,0.000000,0.000000}%
\pgfsetstrokecolor{textcolor}%
\pgfsetfillcolor{textcolor}%
\pgftext[x=1.064412in, y=3.370532in, left, base]{\color{textcolor}\sffamily\fontsize{10.000000}{12.000000}\selectfont 3}%
\end{pgfscope}%
\begin{pgfscope}%
\pgfpathrectangle{\pgfqpoint{1.250000in}{0.660000in}}{\pgfqpoint{7.750000in}{4.620000in}}%
\pgfusepath{clip}%
\pgfsetrectcap%
\pgfsetroundjoin%
\pgfsetlinewidth{0.803000pt}%
\definecolor{currentstroke}{rgb}{0.690196,0.690196,0.690196}%
\pgfsetstrokecolor{currentstroke}%
\pgfsetdash{}{0pt}%
\pgfpathmoveto{\pgfqpoint{1.250000in}{4.274392in}}%
\pgfpathlineto{\pgfqpoint{9.000000in}{4.274392in}}%
\pgfusepath{stroke}%
\end{pgfscope}%
\begin{pgfscope}%
\pgfsetbuttcap%
\pgfsetroundjoin%
\definecolor{currentfill}{rgb}{0.000000,0.000000,0.000000}%
\pgfsetfillcolor{currentfill}%
\pgfsetlinewidth{0.803000pt}%
\definecolor{currentstroke}{rgb}{0.000000,0.000000,0.000000}%
\pgfsetstrokecolor{currentstroke}%
\pgfsetdash{}{0pt}%
\pgfsys@defobject{currentmarker}{\pgfqpoint{-0.048611in}{0.000000in}}{\pgfqpoint{-0.000000in}{0.000000in}}{%
\pgfpathmoveto{\pgfqpoint{-0.000000in}{0.000000in}}%
\pgfpathlineto{\pgfqpoint{-0.048611in}{0.000000in}}%
\pgfusepath{stroke,fill}%
}%
\begin{pgfscope}%
\pgfsys@transformshift{1.250000in}{4.274392in}%
\pgfsys@useobject{currentmarker}{}%
\end{pgfscope}%
\end{pgfscope}%
\begin{pgfscope}%
\definecolor{textcolor}{rgb}{0.000000,0.000000,0.000000}%
\pgfsetstrokecolor{textcolor}%
\pgfsetfillcolor{textcolor}%
\pgftext[x=1.064412in, y=4.221630in, left, base]{\color{textcolor}\sffamily\fontsize{10.000000}{12.000000}\selectfont 4}%
\end{pgfscope}%
\begin{pgfscope}%
\pgfpathrectangle{\pgfqpoint{1.250000in}{0.660000in}}{\pgfqpoint{7.750000in}{4.620000in}}%
\pgfusepath{clip}%
\pgfsetrectcap%
\pgfsetroundjoin%
\pgfsetlinewidth{0.803000pt}%
\definecolor{currentstroke}{rgb}{0.690196,0.690196,0.690196}%
\pgfsetstrokecolor{currentstroke}%
\pgfsetdash{}{0pt}%
\pgfpathmoveto{\pgfqpoint{1.250000in}{5.125490in}}%
\pgfpathlineto{\pgfqpoint{9.000000in}{5.125490in}}%
\pgfusepath{stroke}%
\end{pgfscope}%
\begin{pgfscope}%
\pgfsetbuttcap%
\pgfsetroundjoin%
\definecolor{currentfill}{rgb}{0.000000,0.000000,0.000000}%
\pgfsetfillcolor{currentfill}%
\pgfsetlinewidth{0.803000pt}%
\definecolor{currentstroke}{rgb}{0.000000,0.000000,0.000000}%
\pgfsetstrokecolor{currentstroke}%
\pgfsetdash{}{0pt}%
\pgfsys@defobject{currentmarker}{\pgfqpoint{-0.048611in}{0.000000in}}{\pgfqpoint{-0.000000in}{0.000000in}}{%
\pgfpathmoveto{\pgfqpoint{-0.000000in}{0.000000in}}%
\pgfpathlineto{\pgfqpoint{-0.048611in}{0.000000in}}%
\pgfusepath{stroke,fill}%
}%
\begin{pgfscope}%
\pgfsys@transformshift{1.250000in}{5.125490in}%
\pgfsys@useobject{currentmarker}{}%
\end{pgfscope}%
\end{pgfscope}%
\begin{pgfscope}%
\definecolor{textcolor}{rgb}{0.000000,0.000000,0.000000}%
\pgfsetstrokecolor{textcolor}%
\pgfsetfillcolor{textcolor}%
\pgftext[x=1.064412in, y=5.072728in, left, base]{\color{textcolor}\sffamily\fontsize{10.000000}{12.000000}\selectfont 5}%
\end{pgfscope}%
\begin{pgfscope}%
\definecolor{textcolor}{rgb}{0.000000,0.000000,0.000000}%
\pgfsetstrokecolor{textcolor}%
\pgfsetfillcolor{textcolor}%
\pgftext[x=1.008857in,y=2.970000in,,bottom,rotate=90.000000]{\color{textcolor}\sffamily\fontsize{10.000000}{12.000000}\selectfont Sum of \(\displaystyle f_n(x)\)}%
\end{pgfscope}%
\begin{pgfscope}%
\pgfpathrectangle{\pgfqpoint{1.250000in}{0.660000in}}{\pgfqpoint{7.750000in}{4.620000in}}%
\pgfusepath{clip}%
\pgfsetrectcap%
\pgfsetroundjoin%
\pgfsetlinewidth{1.505625pt}%
\definecolor{currentstroke}{rgb}{0.121569,0.466667,0.705882}%
\pgfsetstrokecolor{currentstroke}%
\pgfsetdash{}{0pt}%
\pgfpathmoveto{\pgfqpoint{1.602273in}{0.870000in}}%
\pgfpathlineto{\pgfqpoint{1.673439in}{0.897004in}}%
\pgfpathlineto{\pgfqpoint{1.744605in}{0.923980in}}%
\pgfpathlineto{\pgfqpoint{1.815771in}{0.950902in}}%
\pgfpathlineto{\pgfqpoint{1.886938in}{0.977743in}}%
\pgfpathlineto{\pgfqpoint{1.958104in}{1.004475in}}%
\pgfpathlineto{\pgfqpoint{2.029270in}{1.031071in}}%
\pgfpathlineto{\pgfqpoint{2.100436in}{1.057506in}}%
\pgfpathlineto{\pgfqpoint{2.171602in}{1.083752in}}%
\pgfpathlineto{\pgfqpoint{2.242769in}{1.109782in}}%
\pgfpathlineto{\pgfqpoint{2.313935in}{1.135571in}}%
\pgfpathlineto{\pgfqpoint{2.385101in}{1.161093in}}%
\pgfpathlineto{\pgfqpoint{2.456267in}{1.186321in}}%
\pgfpathlineto{\pgfqpoint{2.527433in}{1.211231in}}%
\pgfpathlineto{\pgfqpoint{2.598600in}{1.235798in}}%
\pgfpathlineto{\pgfqpoint{2.669766in}{1.259996in}}%
\pgfpathlineto{\pgfqpoint{2.740932in}{1.283801in}}%
\pgfpathlineto{\pgfqpoint{2.812098in}{1.307190in}}%
\pgfpathlineto{\pgfqpoint{2.883264in}{1.330138in}}%
\pgfpathlineto{\pgfqpoint{2.954431in}{1.352623in}}%
\pgfpathlineto{\pgfqpoint{3.025597in}{1.374623in}}%
\pgfpathlineto{\pgfqpoint{3.096763in}{1.396114in}}%
\pgfpathlineto{\pgfqpoint{3.167929in}{1.417075in}}%
\pgfpathlineto{\pgfqpoint{3.239096in}{1.437486in}}%
\pgfpathlineto{\pgfqpoint{3.310262in}{1.457325in}}%
\pgfpathlineto{\pgfqpoint{3.381428in}{1.476573in}}%
\pgfpathlineto{\pgfqpoint{3.452594in}{1.495209in}}%
\pgfpathlineto{\pgfqpoint{3.523760in}{1.513217in}}%
\pgfpathlineto{\pgfqpoint{3.594927in}{1.530577in}}%
\pgfpathlineto{\pgfqpoint{3.666093in}{1.547271in}}%
\pgfpathlineto{\pgfqpoint{3.737259in}{1.563284in}}%
\pgfpathlineto{\pgfqpoint{3.808425in}{1.578598in}}%
\pgfpathlineto{\pgfqpoint{3.879591in}{1.593200in}}%
\pgfpathlineto{\pgfqpoint{3.950758in}{1.607072in}}%
\pgfpathlineto{\pgfqpoint{4.021924in}{1.620203in}}%
\pgfpathlineto{\pgfqpoint{4.093090in}{1.632578in}}%
\pgfpathlineto{\pgfqpoint{4.164256in}{1.644186in}}%
\pgfpathlineto{\pgfqpoint{4.235422in}{1.655014in}}%
\pgfpathlineto{\pgfqpoint{4.306589in}{1.665051in}}%
\pgfpathlineto{\pgfqpoint{4.377755in}{1.674288in}}%
\pgfpathlineto{\pgfqpoint{4.448921in}{1.682715in}}%
\pgfpathlineto{\pgfqpoint{4.520087in}{1.690324in}}%
\pgfpathlineto{\pgfqpoint{4.591253in}{1.697107in}}%
\pgfpathlineto{\pgfqpoint{4.662420in}{1.703057in}}%
\pgfpathlineto{\pgfqpoint{4.733586in}{1.708168in}}%
\pgfpathlineto{\pgfqpoint{4.804752in}{1.712435in}}%
\pgfpathlineto{\pgfqpoint{4.875918in}{1.715854in}}%
\pgfpathlineto{\pgfqpoint{4.947084in}{1.718421in}}%
\pgfpathlineto{\pgfqpoint{5.018251in}{1.720134in}}%
\pgfpathlineto{\pgfqpoint{5.089417in}{1.720991in}}%
\pgfpathlineto{\pgfqpoint{5.160583in}{1.720991in}}%
\pgfpathlineto{\pgfqpoint{5.231749in}{1.720134in}}%
\pgfpathlineto{\pgfqpoint{5.302916in}{1.718421in}}%
\pgfpathlineto{\pgfqpoint{5.374082in}{1.715854in}}%
\pgfpathlineto{\pgfqpoint{5.445248in}{1.712435in}}%
\pgfpathlineto{\pgfqpoint{5.516414in}{1.708168in}}%
\pgfpathlineto{\pgfqpoint{5.587580in}{1.703057in}}%
\pgfpathlineto{\pgfqpoint{5.658747in}{1.697107in}}%
\pgfpathlineto{\pgfqpoint{5.729913in}{1.690324in}}%
\pgfpathlineto{\pgfqpoint{5.801079in}{1.682715in}}%
\pgfpathlineto{\pgfqpoint{5.872245in}{1.674288in}}%
\pgfpathlineto{\pgfqpoint{5.943411in}{1.665051in}}%
\pgfpathlineto{\pgfqpoint{6.014578in}{1.655014in}}%
\pgfpathlineto{\pgfqpoint{6.085744in}{1.644186in}}%
\pgfpathlineto{\pgfqpoint{6.156910in}{1.632578in}}%
\pgfpathlineto{\pgfqpoint{6.228076in}{1.620203in}}%
\pgfpathlineto{\pgfqpoint{6.299242in}{1.607072in}}%
\pgfpathlineto{\pgfqpoint{6.370409in}{1.593200in}}%
\pgfpathlineto{\pgfqpoint{6.441575in}{1.578598in}}%
\pgfpathlineto{\pgfqpoint{6.512741in}{1.563284in}}%
\pgfpathlineto{\pgfqpoint{6.583907in}{1.547271in}}%
\pgfpathlineto{\pgfqpoint{6.655073in}{1.530577in}}%
\pgfpathlineto{\pgfqpoint{6.726240in}{1.513217in}}%
\pgfpathlineto{\pgfqpoint{6.797406in}{1.495209in}}%
\pgfpathlineto{\pgfqpoint{6.868572in}{1.476573in}}%
\pgfpathlineto{\pgfqpoint{6.939738in}{1.457325in}}%
\pgfpathlineto{\pgfqpoint{7.010904in}{1.437486in}}%
\pgfpathlineto{\pgfqpoint{7.082071in}{1.417075in}}%
\pgfpathlineto{\pgfqpoint{7.153237in}{1.396114in}}%
\pgfpathlineto{\pgfqpoint{7.224403in}{1.374623in}}%
\pgfpathlineto{\pgfqpoint{7.295569in}{1.352623in}}%
\pgfpathlineto{\pgfqpoint{7.366736in}{1.330138in}}%
\pgfpathlineto{\pgfqpoint{7.437902in}{1.307190in}}%
\pgfpathlineto{\pgfqpoint{7.509068in}{1.283801in}}%
\pgfpathlineto{\pgfqpoint{7.580234in}{1.259996in}}%
\pgfpathlineto{\pgfqpoint{7.651400in}{1.235798in}}%
\pgfpathlineto{\pgfqpoint{7.722567in}{1.211231in}}%
\pgfpathlineto{\pgfqpoint{7.793733in}{1.186321in}}%
\pgfpathlineto{\pgfqpoint{7.864899in}{1.161093in}}%
\pgfpathlineto{\pgfqpoint{7.936065in}{1.135571in}}%
\pgfpathlineto{\pgfqpoint{8.007231in}{1.109782in}}%
\pgfpathlineto{\pgfqpoint{8.078398in}{1.083752in}}%
\pgfpathlineto{\pgfqpoint{8.149564in}{1.057506in}}%
\pgfpathlineto{\pgfqpoint{8.220730in}{1.031071in}}%
\pgfpathlineto{\pgfqpoint{8.291896in}{1.004475in}}%
\pgfpathlineto{\pgfqpoint{8.363062in}{0.977743in}}%
\pgfpathlineto{\pgfqpoint{8.434229in}{0.950902in}}%
\pgfpathlineto{\pgfqpoint{8.505395in}{0.923980in}}%
\pgfpathlineto{\pgfqpoint{8.576561in}{0.897004in}}%
\pgfpathlineto{\pgfqpoint{8.647727in}{0.870000in}}%
\pgfusepath{stroke}%
\end{pgfscope}%
\begin{pgfscope}%
\pgfpathrectangle{\pgfqpoint{1.250000in}{0.660000in}}{\pgfqpoint{7.750000in}{4.620000in}}%
\pgfusepath{clip}%
\pgfsetrectcap%
\pgfsetroundjoin%
\pgfsetlinewidth{1.505625pt}%
\definecolor{currentstroke}{rgb}{1.000000,0.498039,0.054902}%
\pgfsetstrokecolor{currentstroke}%
\pgfsetdash{}{0pt}%
\pgfpathmoveto{\pgfqpoint{1.602273in}{0.870000in}}%
\pgfpathlineto{\pgfqpoint{1.673439in}{1.004791in}}%
\pgfpathlineto{\pgfqpoint{1.744605in}{1.138094in}}%
\pgfpathlineto{\pgfqpoint{1.815771in}{1.268444in}}%
\pgfpathlineto{\pgfqpoint{1.886938in}{1.394434in}}%
\pgfpathlineto{\pgfqpoint{1.958104in}{1.514729in}}%
\pgfpathlineto{\pgfqpoint{2.029270in}{1.628099in}}%
\pgfpathlineto{\pgfqpoint{2.100436in}{1.733436in}}%
\pgfpathlineto{\pgfqpoint{2.171602in}{1.829770in}}%
\pgfpathlineto{\pgfqpoint{2.242769in}{1.916290in}}%
\pgfpathlineto{\pgfqpoint{2.313935in}{1.992354in}}%
\pgfpathlineto{\pgfqpoint{2.385101in}{2.057497in}}%
\pgfpathlineto{\pgfqpoint{2.456267in}{2.111437in}}%
\pgfpathlineto{\pgfqpoint{2.527433in}{2.154078in}}%
\pgfpathlineto{\pgfqpoint{2.598600in}{2.185507in}}%
\pgfpathlineto{\pgfqpoint{2.669766in}{2.205986in}}%
\pgfpathlineto{\pgfqpoint{2.740932in}{2.215948in}}%
\pgfpathlineto{\pgfqpoint{2.812098in}{2.215978in}}%
\pgfpathlineto{\pgfqpoint{2.883264in}{2.206804in}}%
\pgfpathlineto{\pgfqpoint{2.954431in}{2.189271in}}%
\pgfpathlineto{\pgfqpoint{3.025597in}{2.164332in}}%
\pgfpathlineto{\pgfqpoint{3.096763in}{2.133014in}}%
\pgfpathlineto{\pgfqpoint{3.167929in}{2.096405in}}%
\pgfpathlineto{\pgfqpoint{3.239096in}{2.055624in}}%
\pgfpathlineto{\pgfqpoint{3.310262in}{2.011800in}}%
\pgfpathlineto{\pgfqpoint{3.381428in}{1.966050in}}%
\pgfpathlineto{\pgfqpoint{3.452594in}{1.919453in}}%
\pgfpathlineto{\pgfqpoint{3.523760in}{1.873031in}}%
\pgfpathlineto{\pgfqpoint{3.594927in}{1.827731in}}%
\pgfpathlineto{\pgfqpoint{3.666093in}{1.784405in}}%
\pgfpathlineto{\pgfqpoint{3.737259in}{1.743800in}}%
\pgfpathlineto{\pgfqpoint{3.808425in}{1.706541in}}%
\pgfpathlineto{\pgfqpoint{3.879591in}{1.673128in}}%
\pgfpathlineto{\pgfqpoint{3.950758in}{1.643926in}}%
\pgfpathlineto{\pgfqpoint{4.021924in}{1.619166in}}%
\pgfpathlineto{\pgfqpoint{4.093090in}{1.598945in}}%
\pgfpathlineto{\pgfqpoint{4.164256in}{1.583231in}}%
\pgfpathlineto{\pgfqpoint{4.235422in}{1.571866in}}%
\pgfpathlineto{\pgfqpoint{4.306589in}{1.564582in}}%
\pgfpathlineto{\pgfqpoint{4.377755in}{1.561009in}}%
\pgfpathlineto{\pgfqpoint{4.448921in}{1.560689in}}%
\pgfpathlineto{\pgfqpoint{4.520087in}{1.563094in}}%
\pgfpathlineto{\pgfqpoint{4.591253in}{1.567642in}}%
\pgfpathlineto{\pgfqpoint{4.662420in}{1.573713in}}%
\pgfpathlineto{\pgfqpoint{4.733586in}{1.580670in}}%
\pgfpathlineto{\pgfqpoint{4.804752in}{1.587873in}}%
\pgfpathlineto{\pgfqpoint{4.875918in}{1.594701in}}%
\pgfpathlineto{\pgfqpoint{4.947084in}{1.600566in}}%
\pgfpathlineto{\pgfqpoint{5.018251in}{1.604927in}}%
\pgfpathlineto{\pgfqpoint{5.089417in}{1.607304in}}%
\pgfpathlineto{\pgfqpoint{5.160583in}{1.607290in}}%
\pgfpathlineto{\pgfqpoint{5.231749in}{1.604560in}}%
\pgfpathlineto{\pgfqpoint{5.302916in}{1.598877in}}%
\pgfpathlineto{\pgfqpoint{5.374082in}{1.590094in}}%
\pgfpathlineto{\pgfqpoint{5.445248in}{1.578160in}}%
\pgfpathlineto{\pgfqpoint{5.516414in}{1.563115in}}%
\pgfpathlineto{\pgfqpoint{5.587580in}{1.545087in}}%
\pgfpathlineto{\pgfqpoint{5.658747in}{1.524289in}}%
\pgfpathlineto{\pgfqpoint{5.729913in}{1.501006in}}%
\pgfpathlineto{\pgfqpoint{5.801079in}{1.475591in}}%
\pgfpathlineto{\pgfqpoint{5.872245in}{1.448448in}}%
\pgfpathlineto{\pgfqpoint{5.943411in}{1.420023in}}%
\pgfpathlineto{\pgfqpoint{6.014578in}{1.390789in}}%
\pgfpathlineto{\pgfqpoint{6.085744in}{1.361231in}}%
\pgfpathlineto{\pgfqpoint{6.156910in}{1.331836in}}%
\pgfpathlineto{\pgfqpoint{6.228076in}{1.303075in}}%
\pgfpathlineto{\pgfqpoint{6.299242in}{1.275390in}}%
\pgfpathlineto{\pgfqpoint{6.370409in}{1.249185in}}%
\pgfpathlineto{\pgfqpoint{6.441575in}{1.224814in}}%
\pgfpathlineto{\pgfqpoint{6.512741in}{1.202569in}}%
\pgfpathlineto{\pgfqpoint{6.583907in}{1.182676in}}%
\pgfpathlineto{\pgfqpoint{6.655073in}{1.165290in}}%
\pgfpathlineto{\pgfqpoint{6.726240in}{1.150487in}}%
\pgfpathlineto{\pgfqpoint{6.797406in}{1.138269in}}%
\pgfpathlineto{\pgfqpoint{6.868572in}{1.128561in}}%
\pgfpathlineto{\pgfqpoint{6.939738in}{1.121215in}}%
\pgfpathlineto{\pgfqpoint{7.010904in}{1.116017in}}%
\pgfpathlineto{\pgfqpoint{7.082071in}{1.112690in}}%
\pgfpathlineto{\pgfqpoint{7.153237in}{1.110909in}}%
\pgfpathlineto{\pgfqpoint{7.224403in}{1.110305in}}%
\pgfpathlineto{\pgfqpoint{7.295569in}{1.110477in}}%
\pgfpathlineto{\pgfqpoint{7.366736in}{1.111009in}}%
\pgfpathlineto{\pgfqpoint{7.437902in}{1.111476in}}%
\pgfpathlineto{\pgfqpoint{7.509068in}{1.111459in}}%
\pgfpathlineto{\pgfqpoint{7.580234in}{1.110558in}}%
\pgfpathlineto{\pgfqpoint{7.651400in}{1.108402in}}%
\pgfpathlineto{\pgfqpoint{7.722567in}{1.104658in}}%
\pgfpathlineto{\pgfqpoint{7.793733in}{1.099045in}}%
\pgfpathlineto{\pgfqpoint{7.864899in}{1.091337in}}%
\pgfpathlineto{\pgfqpoint{7.936065in}{1.081374in}}%
\pgfpathlineto{\pgfqpoint{8.007231in}{1.069059in}}%
\pgfpathlineto{\pgfqpoint{8.078398in}{1.054369in}}%
\pgfpathlineto{\pgfqpoint{8.149564in}{1.037350in}}%
\pgfpathlineto{\pgfqpoint{8.220730in}{1.018116in}}%
\pgfpathlineto{\pgfqpoint{8.291896in}{0.996847in}}%
\pgfpathlineto{\pgfqpoint{8.363062in}{0.973781in}}%
\pgfpathlineto{\pgfqpoint{8.434229in}{0.949213in}}%
\pgfpathlineto{\pgfqpoint{8.505395in}{0.923475in}}%
\pgfpathlineto{\pgfqpoint{8.576561in}{0.896940in}}%
\pgfpathlineto{\pgfqpoint{8.647727in}{0.870000in}}%
\pgfusepath{stroke}%
\end{pgfscope}%
\begin{pgfscope}%
\pgfpathrectangle{\pgfqpoint{1.250000in}{0.660000in}}{\pgfqpoint{7.750000in}{4.620000in}}%
\pgfusepath{clip}%
\pgfsetrectcap%
\pgfsetroundjoin%
\pgfsetlinewidth{1.505625pt}%
\definecolor{currentstroke}{rgb}{0.172549,0.627451,0.172549}%
\pgfsetstrokecolor{currentstroke}%
\pgfsetdash{}{0pt}%
\pgfpathmoveto{\pgfqpoint{1.602273in}{0.870000in}}%
\pgfpathlineto{\pgfqpoint{1.673439in}{1.138342in}}%
\pgfpathlineto{\pgfqpoint{1.744605in}{1.396385in}}%
\pgfpathlineto{\pgfqpoint{1.815771in}{1.634506in}}%
\pgfpathlineto{\pgfqpoint{1.886938in}{1.844381in}}%
\pgfpathlineto{\pgfqpoint{1.958104in}{2.019508in}}%
\pgfpathlineto{\pgfqpoint{2.029270in}{2.155588in}}%
\pgfpathlineto{\pgfqpoint{2.100436in}{2.250737in}}%
\pgfpathlineto{\pgfqpoint{2.171602in}{2.305520in}}%
\pgfpathlineto{\pgfqpoint{2.242769in}{2.322801in}}%
\pgfpathlineto{\pgfqpoint{2.313935in}{2.307425in}}%
\pgfpathlineto{\pgfqpoint{2.385101in}{2.265773in}}%
\pgfpathlineto{\pgfqpoint{2.456267in}{2.205212in}}%
\pgfpathlineto{\pgfqpoint{2.527433in}{2.133504in}}%
\pgfpathlineto{\pgfqpoint{2.598600in}{2.058209in}}%
\pgfpathlineto{\pgfqpoint{2.669766in}{1.986141in}}%
\pgfpathlineto{\pgfqpoint{2.740932in}{1.922917in}}%
\pgfpathlineto{\pgfqpoint{2.812098in}{1.872618in}}%
\pgfpathlineto{\pgfqpoint{2.883264in}{1.837613in}}%
\pgfpathlineto{\pgfqpoint{2.954431in}{1.818525in}}%
\pgfpathlineto{\pgfqpoint{3.025597in}{1.814347in}}%
\pgfpathlineto{\pgfqpoint{3.096763in}{1.822693in}}%
\pgfpathlineto{\pgfqpoint{3.167929in}{1.840144in}}%
\pgfpathlineto{\pgfqpoint{3.239096in}{1.862668in}}%
\pgfpathlineto{\pgfqpoint{3.310262in}{1.886057in}}%
\pgfpathlineto{\pgfqpoint{3.381428in}{1.906350in}}%
\pgfpathlineto{\pgfqpoint{3.452594in}{1.920206in}}%
\pgfpathlineto{\pgfqpoint{3.523760in}{1.925188in}}%
\pgfpathlineto{\pgfqpoint{3.594927in}{1.919946in}}%
\pgfpathlineto{\pgfqpoint{3.666093in}{1.904275in}}%
\pgfpathlineto{\pgfqpoint{3.737259in}{1.879066in}}%
\pgfpathlineto{\pgfqpoint{3.808425in}{1.846145in}}%
\pgfpathlineto{\pgfqpoint{3.879591in}{1.808033in}}%
\pgfpathlineto{\pgfqpoint{3.950758in}{1.767649in}}%
\pgfpathlineto{\pgfqpoint{4.021924in}{1.727987in}}%
\pgfpathlineto{\pgfqpoint{4.093090in}{1.691806in}}%
\pgfpathlineto{\pgfqpoint{4.164256in}{1.661354in}}%
\pgfpathlineto{\pgfqpoint{4.235422in}{1.638166in}}%
\pgfpathlineto{\pgfqpoint{4.306589in}{1.622935in}}%
\pgfpathlineto{\pgfqpoint{4.377755in}{1.615483in}}%
\pgfpathlineto{\pgfqpoint{4.448921in}{1.614814in}}%
\pgfpathlineto{\pgfqpoint{4.520087in}{1.619259in}}%
\pgfpathlineto{\pgfqpoint{4.591253in}{1.626676in}}%
\pgfpathlineto{\pgfqpoint{4.662420in}{1.634695in}}%
\pgfpathlineto{\pgfqpoint{4.733586in}{1.640976in}}%
\pgfpathlineto{\pgfqpoint{4.804752in}{1.643454in}}%
\pgfpathlineto{\pgfqpoint{4.875918in}{1.640546in}}%
\pgfpathlineto{\pgfqpoint{4.947084in}{1.631299in}}%
\pgfpathlineto{\pgfqpoint{5.018251in}{1.615472in}}%
\pgfpathlineto{\pgfqpoint{5.089417in}{1.593535in}}%
\pgfpathlineto{\pgfqpoint{5.160583in}{1.566595in}}%
\pgfpathlineto{\pgfqpoint{5.231749in}{1.536260in}}%
\pgfpathlineto{\pgfqpoint{5.302916in}{1.504453in}}%
\pgfpathlineto{\pgfqpoint{5.374082in}{1.473196in}}%
\pgfpathlineto{\pgfqpoint{5.445248in}{1.444395in}}%
\pgfpathlineto{\pgfqpoint{5.516414in}{1.419639in}}%
\pgfpathlineto{\pgfqpoint{5.587580in}{1.400037in}}%
\pgfpathlineto{\pgfqpoint{5.658747in}{1.386117in}}%
\pgfpathlineto{\pgfqpoint{5.729913in}{1.377782in}}%
\pgfpathlineto{\pgfqpoint{5.801079in}{1.374336in}}%
\pgfpathlineto{\pgfqpoint{5.872245in}{1.374573in}}%
\pgfpathlineto{\pgfqpoint{5.943411in}{1.376919in}}%
\pgfpathlineto{\pgfqpoint{6.014578in}{1.379605in}}%
\pgfpathlineto{\pgfqpoint{6.085744in}{1.380867in}}%
\pgfpathlineto{\pgfqpoint{6.156910in}{1.379129in}}%
\pgfpathlineto{\pgfqpoint{6.228076in}{1.373173in}}%
\pgfpathlineto{\pgfqpoint{6.299242in}{1.362259in}}%
\pgfpathlineto{\pgfqpoint{6.370409in}{1.346197in}}%
\pgfpathlineto{\pgfqpoint{6.441575in}{1.325356in}}%
\pgfpathlineto{\pgfqpoint{6.512741in}{1.300618in}}%
\pgfpathlineto{\pgfqpoint{6.583907in}{1.273269in}}%
\pgfpathlineto{\pgfqpoint{6.655073in}{1.244859in}}%
\pgfpathlineto{\pgfqpoint{6.726240in}{1.217028in}}%
\pgfpathlineto{\pgfqpoint{6.797406in}{1.191329in}}%
\pgfpathlineto{\pgfqpoint{6.868572in}{1.169060in}}%
\pgfpathlineto{\pgfqpoint{6.939738in}{1.151130in}}%
\pgfpathlineto{\pgfqpoint{7.010904in}{1.137967in}}%
\pgfpathlineto{\pgfqpoint{7.082071in}{1.129474in}}%
\pgfpathlineto{\pgfqpoint{7.153237in}{1.125050in}}%
\pgfpathlineto{\pgfqpoint{7.224403in}{1.123658in}}%
\pgfpathlineto{\pgfqpoint{7.295569in}{1.123943in}}%
\pgfpathlineto{\pgfqpoint{7.366736in}{1.124382in}}%
\pgfpathlineto{\pgfqpoint{7.437902in}{1.123451in}}%
\pgfpathlineto{\pgfqpoint{7.509068in}{1.119788in}}%
\pgfpathlineto{\pgfqpoint{7.580234in}{1.112341in}}%
\pgfpathlineto{\pgfqpoint{7.651400in}{1.100478in}}%
\pgfpathlineto{\pgfqpoint{7.722567in}{1.084051in}}%
\pgfpathlineto{\pgfqpoint{7.793733in}{1.063412in}}%
\pgfpathlineto{\pgfqpoint{7.864899in}{1.039368in}}%
\pgfpathlineto{\pgfqpoint{7.936065in}{1.013093in}}%
\pgfpathlineto{\pgfqpoint{8.007231in}{0.985997in}}%
\pgfpathlineto{\pgfqpoint{8.078398in}{0.959572in}}%
\pgfpathlineto{\pgfqpoint{8.149564in}{0.935230in}}%
\pgfpathlineto{\pgfqpoint{8.220730in}{0.914151in}}%
\pgfpathlineto{\pgfqpoint{8.291896in}{0.897154in}}%
\pgfpathlineto{\pgfqpoint{8.363062in}{0.884611in}}%
\pgfpathlineto{\pgfqpoint{8.434229in}{0.876406in}}%
\pgfpathlineto{\pgfqpoint{8.505395in}{0.871951in}}%
\pgfpathlineto{\pgfqpoint{8.576561in}{0.870248in}}%
\pgfpathlineto{\pgfqpoint{8.647727in}{0.870000in}}%
\pgfusepath{stroke}%
\end{pgfscope}%
\begin{pgfscope}%
\pgfpathrectangle{\pgfqpoint{1.250000in}{0.660000in}}{\pgfqpoint{7.750000in}{4.620000in}}%
\pgfusepath{clip}%
\pgfsetrectcap%
\pgfsetroundjoin%
\pgfsetlinewidth{1.505625pt}%
\definecolor{currentstroke}{rgb}{0.839216,0.152941,0.156863}%
\pgfsetstrokecolor{currentstroke}%
\pgfsetdash{}{0pt}%
\pgfpathmoveto{\pgfqpoint{1.602273in}{0.870000in}}%
\pgfpathlineto{\pgfqpoint{1.673439in}{1.397317in}}%
\pgfpathlineto{\pgfqpoint{1.744605in}{1.851379in}}%
\pgfpathlineto{\pgfqpoint{1.815771in}{2.176827in}}%
\pgfpathlineto{\pgfqpoint{1.886938in}{2.348886in}}%
\pgfpathlineto{\pgfqpoint{1.958104in}{2.377292in}}%
\pgfpathlineto{\pgfqpoint{2.029270in}{2.300653in}}%
\pgfpathlineto{\pgfqpoint{2.100436in}{2.173358in}}%
\pgfpathlineto{\pgfqpoint{2.171602in}{2.049353in}}%
\pgfpathlineto{\pgfqpoint{2.242769in}{1.967789in}}%
\pgfpathlineto{\pgfqpoint{2.313935in}{1.944679in}}%
\pgfpathlineto{\pgfqpoint{2.385101in}{1.972493in}}%
\pgfpathlineto{\pgfqpoint{2.456267in}{2.026998in}}%
\pgfpathlineto{\pgfqpoint{2.527433in}{2.078362in}}%
\pgfpathlineto{\pgfqpoint{2.598600in}{2.102478in}}%
\pgfpathlineto{\pgfqpoint{2.669766in}{2.088727in}}%
\pgfpathlineto{\pgfqpoint{2.740932in}{2.042016in}}%
\pgfpathlineto{\pgfqpoint{2.812098in}{1.979026in}}%
\pgfpathlineto{\pgfqpoint{2.883264in}{1.920665in}}%
\pgfpathlineto{\pgfqpoint{2.954431in}{1.883801in}}%
\pgfpathlineto{\pgfqpoint{3.025597in}{1.875353in}}%
\pgfpathlineto{\pgfqpoint{3.096763in}{1.890666in}}%
\pgfpathlineto{\pgfqpoint{3.167929in}{1.916360in}}%
\pgfpathlineto{\pgfqpoint{3.239096in}{1.936225in}}%
\pgfpathlineto{\pgfqpoint{3.310262in}{1.937662in}}%
\pgfpathlineto{\pgfqpoint{3.381428in}{1.916247in}}%
\pgfpathlineto{\pgfqpoint{3.452594in}{1.876794in}}%
\pgfpathlineto{\pgfqpoint{3.523760in}{1.830811in}}%
\pgfpathlineto{\pgfqpoint{3.594927in}{1.791556in}}%
\pgfpathlineto{\pgfqpoint{3.666093in}{1.768754in}}%
\pgfpathlineto{\pgfqpoint{3.737259in}{1.765041in}}%
\pgfpathlineto{\pgfqpoint{3.808425in}{1.775379in}}%
\pgfpathlineto{\pgfqpoint{3.879591in}{1.789493in}}%
\pgfpathlineto{\pgfqpoint{3.950758in}{1.796211in}}%
\pgfpathlineto{\pgfqpoint{4.021924in}{1.787903in}}%
\pgfpathlineto{\pgfqpoint{4.093090in}{1.763292in}}%
\pgfpathlineto{\pgfqpoint{4.164256in}{1.727627in}}%
\pgfpathlineto{\pgfqpoint{4.235422in}{1.690277in}}%
\pgfpathlineto{\pgfqpoint{4.306589in}{1.660828in}}%
\pgfpathlineto{\pgfqpoint{4.377755in}{1.645279in}}%
\pgfpathlineto{\pgfqpoint{4.448921in}{1.643832in}}%
\pgfpathlineto{\pgfqpoint{4.520087in}{1.651062in}}%
\pgfpathlineto{\pgfqpoint{4.591253in}{1.658321in}}%
\pgfpathlineto{\pgfqpoint{4.662420in}{1.657298in}}%
\pgfpathlineto{\pgfqpoint{4.733586in}{1.643320in}}%
\pgfpathlineto{\pgfqpoint{4.804752in}{1.617065in}}%
\pgfpathlineto{\pgfqpoint{4.875918in}{1.584093in}}%
\pgfpathlineto{\pgfqpoint{4.947084in}{1.552450in}}%
\pgfpathlineto{\pgfqpoint{5.018251in}{1.529369in}}%
\pgfpathlineto{\pgfqpoint{5.089417in}{1.518412in}}%
\pgfpathlineto{\pgfqpoint{5.160583in}{1.518164in}}%
\pgfpathlineto{\pgfqpoint{5.231749in}{1.522963in}}%
\pgfpathlineto{\pgfqpoint{5.302916in}{1.525295in}}%
\pgfpathlineto{\pgfqpoint{5.374082in}{1.518862in}}%
\pgfpathlineto{\pgfqpoint{5.445248in}{1.501068in}}%
\pgfpathlineto{\pgfqpoint{5.516414in}{1.473952in}}%
\pgfpathlineto{\pgfqpoint{5.587580in}{1.443247in}}%
\pgfpathlineto{\pgfqpoint{5.658747in}{1.415979in}}%
\pgfpathlineto{\pgfqpoint{5.729913in}{1.397611in}}%
\pgfpathlineto{\pgfqpoint{5.801079in}{1.389889in}}%
\pgfpathlineto{\pgfqpoint{5.872245in}{1.390229in}}%
\pgfpathlineto{\pgfqpoint{5.943411in}{1.392861in}}%
\pgfpathlineto{\pgfqpoint{6.014578in}{1.391218in}}%
\pgfpathlineto{\pgfqpoint{6.085744in}{1.380599in}}%
\pgfpathlineto{\pgfqpoint{6.156910in}{1.360023in}}%
\pgfpathlineto{\pgfqpoint{6.228076in}{1.332547in}}%
\pgfpathlineto{\pgfqpoint{6.299242in}{1.303952in}}%
\pgfpathlineto{\pgfqpoint{6.370409in}{1.280364in}}%
\pgfpathlineto{\pgfqpoint{6.441575in}{1.265774in}}%
\pgfpathlineto{\pgfqpoint{6.512741in}{1.260468in}}%
\pgfpathlineto{\pgfqpoint{6.583907in}{1.260971in}}%
\pgfpathlineto{\pgfqpoint{6.655073in}{1.261519in}}%
\pgfpathlineto{\pgfqpoint{6.726240in}{1.256416in}}%
\pgfpathlineto{\pgfqpoint{6.797406in}{1.242341in}}%
\pgfpathlineto{\pgfqpoint{6.868572in}{1.219652in}}%
\pgfpathlineto{\pgfqpoint{6.939738in}{1.192190in}}%
\pgfpathlineto{\pgfqpoint{7.010904in}{1.165679in}}%
\pgfpathlineto{\pgfqpoint{7.082071in}{1.145384in}}%
\pgfpathlineto{\pgfqpoint{7.153237in}{1.133990in}}%
\pgfpathlineto{\pgfqpoint{7.224403in}{1.130540in}}%
\pgfpathlineto{\pgfqpoint{7.295569in}{1.130865in}}%
\pgfpathlineto{\pgfqpoint{7.366736in}{1.129311in}}%
\pgfpathlineto{\pgfqpoint{7.437902in}{1.121039in}}%
\pgfpathlineto{\pgfqpoint{7.509068in}{1.103983in}}%
\pgfpathlineto{\pgfqpoint{7.580234in}{1.079662in}}%
\pgfpathlineto{\pgfqpoint{7.651400in}{1.052532in}}%
\pgfpathlineto{\pgfqpoint{7.722567in}{1.028157in}}%
\pgfpathlineto{\pgfqpoint{7.793733in}{1.010941in}}%
\pgfpathlineto{\pgfqpoint{7.864899in}{1.002349in}}%
\pgfpathlineto{\pgfqpoint{7.936065in}{1.000332in}}%
\pgfpathlineto{\pgfqpoint{8.007231in}{1.000176in}}%
\pgfpathlineto{\pgfqpoint{8.078398in}{0.996437in}}%
\pgfpathlineto{\pgfqpoint{8.149564in}{0.985150in}}%
\pgfpathlineto{\pgfqpoint{8.220730in}{0.965441in}}%
\pgfpathlineto{\pgfqpoint{8.291896in}{0.939867in}}%
\pgfpathlineto{\pgfqpoint{8.363062in}{0.913366in}}%
\pgfpathlineto{\pgfqpoint{8.434229in}{0.891239in}}%
\pgfpathlineto{\pgfqpoint{8.505395in}{0.876998in}}%
\pgfpathlineto{\pgfqpoint{8.576561in}{0.870932in}}%
\pgfpathlineto{\pgfqpoint{8.647727in}{0.870000in}}%
\pgfusepath{stroke}%
\end{pgfscope}%
\begin{pgfscope}%
\pgfpathrectangle{\pgfqpoint{1.250000in}{0.660000in}}{\pgfqpoint{7.750000in}{4.620000in}}%
\pgfusepath{clip}%
\pgfsetrectcap%
\pgfsetroundjoin%
\pgfsetlinewidth{1.505625pt}%
\definecolor{currentstroke}{rgb}{0.580392,0.403922,0.741176}%
\pgfsetstrokecolor{currentstroke}%
\pgfsetdash{}{0pt}%
\pgfpathmoveto{\pgfqpoint{1.602273in}{0.870000in}}%
\pgfpathlineto{\pgfqpoint{1.673439in}{2.040183in}}%
\pgfpathlineto{\pgfqpoint{1.744605in}{2.418677in}}%
\pgfpathlineto{\pgfqpoint{1.815771in}{2.181353in}}%
\pgfpathlineto{\pgfqpoint{1.886938in}{2.023958in}}%
\pgfpathlineto{\pgfqpoint{1.958104in}{2.143627in}}%
\pgfpathlineto{\pgfqpoint{2.029270in}{2.212880in}}%
\pgfpathlineto{\pgfqpoint{2.100436in}{2.102176in}}%
\pgfpathlineto{\pgfqpoint{2.171602in}{2.033916in}}%
\pgfpathlineto{\pgfqpoint{2.242769in}{2.098087in}}%
\pgfpathlineto{\pgfqpoint{2.313935in}{2.123162in}}%
\pgfpathlineto{\pgfqpoint{2.385101in}{2.044376in}}%
\pgfpathlineto{\pgfqpoint{2.456267in}{2.002894in}}%
\pgfpathlineto{\pgfqpoint{2.527433in}{2.046012in}}%
\pgfpathlineto{\pgfqpoint{2.598600in}{2.052960in}}%
\pgfpathlineto{\pgfqpoint{2.669766in}{1.989301in}}%
\pgfpathlineto{\pgfqpoint{2.740932in}{1.960986in}}%
\pgfpathlineto{\pgfqpoint{2.812098in}{1.992570in}}%
\pgfpathlineto{\pgfqpoint{2.883264in}{1.989473in}}%
\pgfpathlineto{\pgfqpoint{2.954431in}{1.935004in}}%
\pgfpathlineto{\pgfqpoint{3.025597in}{1.914635in}}%
\pgfpathlineto{\pgfqpoint{3.096763in}{1.938639in}}%
\pgfpathlineto{\pgfqpoint{3.167929in}{1.929085in}}%
\pgfpathlineto{\pgfqpoint{3.239096in}{1.881041in}}%
\pgfpathlineto{\pgfqpoint{3.310262in}{1.866031in}}%
\pgfpathlineto{\pgfqpoint{3.381428in}{1.884466in}}%
\pgfpathlineto{\pgfqpoint{3.452594in}{1.870389in}}%
\pgfpathlineto{\pgfqpoint{3.523760in}{1.827264in}}%
\pgfpathlineto{\pgfqpoint{3.594927in}{1.816121in}}%
\pgfpathlineto{\pgfqpoint{3.666093in}{1.830142in}}%
\pgfpathlineto{\pgfqpoint{3.737259in}{1.812725in}}%
\pgfpathlineto{\pgfqpoint{3.808425in}{1.773614in}}%
\pgfpathlineto{\pgfqpoint{3.879591in}{1.765379in}}%
\pgfpathlineto{\pgfqpoint{3.950758in}{1.775709in}}%
\pgfpathlineto{\pgfqpoint{4.021924in}{1.755745in}}%
\pgfpathlineto{\pgfqpoint{4.093090in}{1.720060in}}%
\pgfpathlineto{\pgfqpoint{4.164256in}{1.714066in}}%
\pgfpathlineto{\pgfqpoint{4.235422in}{1.721188in}}%
\pgfpathlineto{\pgfqpoint{4.306589in}{1.699247in}}%
\pgfpathlineto{\pgfqpoint{4.377755in}{1.666588in}}%
\pgfpathlineto{\pgfqpoint{4.448921in}{1.662339in}}%
\pgfpathlineto{\pgfqpoint{4.520087in}{1.666593in}}%
\pgfpathlineto{\pgfqpoint{4.591253in}{1.643110in}}%
\pgfpathlineto{\pgfqpoint{4.662420in}{1.613185in}}%
\pgfpathlineto{\pgfqpoint{4.733586in}{1.610295in}}%
\pgfpathlineto{\pgfqpoint{4.804752in}{1.611931in}}%
\pgfpathlineto{\pgfqpoint{4.875918in}{1.587254in}}%
\pgfpathlineto{\pgfqpoint{4.947084in}{1.559845in}}%
\pgfpathlineto{\pgfqpoint{5.018251in}{1.557999in}}%
\pgfpathlineto{\pgfqpoint{5.089417in}{1.557209in}}%
\pgfpathlineto{\pgfqpoint{5.160583in}{1.531626in}}%
\pgfpathlineto{\pgfqpoint{5.231749in}{1.506562in}}%
\pgfpathlineto{\pgfqpoint{5.302916in}{1.505495in}}%
\pgfpathlineto{\pgfqpoint{5.374082in}{1.502432in}}%
\pgfpathlineto{\pgfqpoint{5.445248in}{1.476191in}}%
\pgfpathlineto{\pgfqpoint{5.516414in}{1.453332in}}%
\pgfpathlineto{\pgfqpoint{5.587580in}{1.452812in}}%
\pgfpathlineto{\pgfqpoint{5.658747in}{1.447606in}}%
\pgfpathlineto{\pgfqpoint{5.729913in}{1.420922in}}%
\pgfpathlineto{\pgfqpoint{5.801079in}{1.400149in}}%
\pgfpathlineto{\pgfqpoint{5.872245in}{1.399972in}}%
\pgfpathlineto{\pgfqpoint{5.943411in}{1.392734in}}%
\pgfpathlineto{\pgfqpoint{6.014578in}{1.365802in}}%
\pgfpathlineto{\pgfqpoint{6.085744in}{1.347010in}}%
\pgfpathlineto{\pgfqpoint{6.156910in}{1.346990in}}%
\pgfpathlineto{\pgfqpoint{6.228076in}{1.337821in}}%
\pgfpathlineto{\pgfqpoint{6.299242in}{1.310818in}}%
\pgfpathlineto{\pgfqpoint{6.370409in}{1.293909in}}%
\pgfpathlineto{\pgfqpoint{6.441575in}{1.293878in}}%
\pgfpathlineto{\pgfqpoint{6.512741in}{1.282871in}}%
\pgfpathlineto{\pgfqpoint{6.583907in}{1.255959in}}%
\pgfpathlineto{\pgfqpoint{6.655073in}{1.240843in}}%
\pgfpathlineto{\pgfqpoint{6.726240in}{1.240644in}}%
\pgfpathlineto{\pgfqpoint{6.797406in}{1.227889in}}%
\pgfpathlineto{\pgfqpoint{6.868572in}{1.201220in}}%
\pgfpathlineto{\pgfqpoint{6.939738in}{1.187808in}}%
\pgfpathlineto{\pgfqpoint{7.010904in}{1.187293in}}%
\pgfpathlineto{\pgfqpoint{7.082071in}{1.172877in}}%
\pgfpathlineto{\pgfqpoint{7.153237in}{1.146596in}}%
\pgfpathlineto{\pgfqpoint{7.224403in}{1.134800in}}%
\pgfpathlineto{\pgfqpoint{7.295569in}{1.133829in}}%
\pgfpathlineto{\pgfqpoint{7.366736in}{1.117842in}}%
\pgfpathlineto{\pgfqpoint{7.437902in}{1.092081in}}%
\pgfpathlineto{\pgfqpoint{7.509068in}{1.081813in}}%
\pgfpathlineto{\pgfqpoint{7.580234in}{1.080258in}}%
\pgfpathlineto{\pgfqpoint{7.651400in}{1.062787in}}%
\pgfpathlineto{\pgfqpoint{7.722567in}{1.037674in}}%
\pgfpathlineto{\pgfqpoint{7.793733in}{1.028844in}}%
\pgfpathlineto{\pgfqpoint{7.864899in}{1.026579in}}%
\pgfpathlineto{\pgfqpoint{7.936065in}{1.007716in}}%
\pgfpathlineto{\pgfqpoint{8.007231in}{0.983372in}}%
\pgfpathlineto{\pgfqpoint{8.078398in}{0.975889in}}%
\pgfpathlineto{\pgfqpoint{8.149564in}{0.972796in}}%
\pgfpathlineto{\pgfqpoint{8.220730in}{0.952634in}}%
\pgfpathlineto{\pgfqpoint{8.291896in}{0.929175in}}%
\pgfpathlineto{\pgfqpoint{8.363062in}{0.922942in}}%
\pgfpathlineto{\pgfqpoint{8.434229in}{0.918909in}}%
\pgfpathlineto{\pgfqpoint{8.505395in}{0.897545in}}%
\pgfpathlineto{\pgfqpoint{8.576561in}{0.875081in}}%
\pgfpathlineto{\pgfqpoint{8.647727in}{0.870000in}}%
\pgfusepath{stroke}%
\end{pgfscope}%
\begin{pgfscope}%
\pgfpathrectangle{\pgfqpoint{1.250000in}{0.660000in}}{\pgfqpoint{7.750000in}{4.620000in}}%
\pgfusepath{clip}%
\pgfsetrectcap%
\pgfsetroundjoin%
\pgfsetlinewidth{1.505625pt}%
\definecolor{currentstroke}{rgb}{0.549020,0.337255,0.294118}%
\pgfsetstrokecolor{currentstroke}%
\pgfsetdash{}{0pt}%
\pgfpathmoveto{\pgfqpoint{1.602273in}{0.870000in}}%
\pgfpathlineto{\pgfqpoint{1.673439in}{2.432383in}}%
\pgfpathlineto{\pgfqpoint{1.744605in}{2.050563in}}%
\pgfpathlineto{\pgfqpoint{1.815771in}{2.253997in}}%
\pgfpathlineto{\pgfqpoint{1.886938in}{2.087128in}}%
\pgfpathlineto{\pgfqpoint{1.958104in}{2.191685in}}%
\pgfpathlineto{\pgfqpoint{2.029270in}{2.082718in}}%
\pgfpathlineto{\pgfqpoint{2.100436in}{2.148884in}}%
\pgfpathlineto{\pgfqpoint{2.171602in}{2.067430in}}%
\pgfpathlineto{\pgfqpoint{2.242769in}{2.112790in}}%
\pgfpathlineto{\pgfqpoint{2.313935in}{2.047707in}}%
\pgfpathlineto{\pgfqpoint{2.385101in}{2.079783in}}%
\pgfpathlineto{\pgfqpoint{2.456267in}{2.025745in}}%
\pgfpathlineto{\pgfqpoint{2.527433in}{2.048455in}}%
\pgfpathlineto{\pgfqpoint{2.598600in}{2.002491in}}%
\pgfpathlineto{\pgfqpoint{2.669766in}{2.018143in}}%
\pgfpathlineto{\pgfqpoint{2.740932in}{1.978421in}}%
\pgfpathlineto{\pgfqpoint{2.812098in}{1.988497in}}%
\pgfpathlineto{\pgfqpoint{2.883264in}{1.953800in}}%
\pgfpathlineto{\pgfqpoint{2.954431in}{1.959314in}}%
\pgfpathlineto{\pgfqpoint{3.025597in}{1.928786in}}%
\pgfpathlineto{\pgfqpoint{3.096763in}{1.930469in}}%
\pgfpathlineto{\pgfqpoint{3.167929in}{1.903478in}}%
\pgfpathlineto{\pgfqpoint{3.239096in}{1.901880in}}%
\pgfpathlineto{\pgfqpoint{3.310262in}{1.877944in}}%
\pgfpathlineto{\pgfqpoint{3.381428in}{1.873493in}}%
\pgfpathlineto{\pgfqpoint{3.452594in}{1.852229in}}%
\pgfpathlineto{\pgfqpoint{3.523760in}{1.845269in}}%
\pgfpathlineto{\pgfqpoint{3.594927in}{1.826366in}}%
\pgfpathlineto{\pgfqpoint{3.666093in}{1.817182in}}%
\pgfpathlineto{\pgfqpoint{3.737259in}{1.800377in}}%
\pgfpathlineto{\pgfqpoint{3.808425in}{1.789211in}}%
\pgfpathlineto{\pgfqpoint{3.879591in}{1.774280in}}%
\pgfpathlineto{\pgfqpoint{3.950758in}{1.761341in}}%
\pgfpathlineto{\pgfqpoint{4.021924in}{1.748087in}}%
\pgfpathlineto{\pgfqpoint{4.093090in}{1.733560in}}%
\pgfpathlineto{\pgfqpoint{4.164256in}{1.721810in}}%
\pgfpathlineto{\pgfqpoint{4.235422in}{1.705861in}}%
\pgfpathlineto{\pgfqpoint{4.306589in}{1.695455in}}%
\pgfpathlineto{\pgfqpoint{4.377755in}{1.678236in}}%
\pgfpathlineto{\pgfqpoint{4.448921in}{1.669029in}}%
\pgfpathlineto{\pgfqpoint{4.520087in}{1.650679in}}%
\pgfpathlineto{\pgfqpoint{4.591253in}{1.642537in}}%
\pgfpathlineto{\pgfqpoint{4.662420in}{1.623187in}}%
\pgfpathlineto{\pgfqpoint{4.733586in}{1.615983in}}%
\pgfpathlineto{\pgfqpoint{4.804752in}{1.595755in}}%
\pgfpathlineto{\pgfqpoint{4.875918in}{1.589370in}}%
\pgfpathlineto{\pgfqpoint{4.947084in}{1.568380in}}%
\pgfpathlineto{\pgfqpoint{5.018251in}{1.562701in}}%
\pgfpathlineto{\pgfqpoint{5.089417in}{1.541060in}}%
\pgfpathlineto{\pgfqpoint{5.160583in}{1.535978in}}%
\pgfpathlineto{\pgfqpoint{5.231749in}{1.513792in}}%
\pgfpathlineto{\pgfqpoint{5.302916in}{1.509205in}}%
\pgfpathlineto{\pgfqpoint{5.374082in}{1.486573in}}%
\pgfpathlineto{\pgfqpoint{5.445248in}{1.482383in}}%
\pgfpathlineto{\pgfqpoint{5.516414in}{1.459403in}}%
\pgfpathlineto{\pgfqpoint{5.587580in}{1.455513in}}%
\pgfpathlineto{\pgfqpoint{5.658747in}{1.432279in}}%
\pgfpathlineto{\pgfqpoint{5.729913in}{1.428598in}}%
\pgfpathlineto{\pgfqpoint{5.801079in}{1.405200in}}%
\pgfpathlineto{\pgfqpoint{5.872245in}{1.401640in}}%
\pgfpathlineto{\pgfqpoint{5.943411in}{1.378163in}}%
\pgfpathlineto{\pgfqpoint{6.014578in}{1.374641in}}%
\pgfpathlineto{\pgfqpoint{6.085744in}{1.351166in}}%
\pgfpathlineto{\pgfqpoint{6.156910in}{1.347601in}}%
\pgfpathlineto{\pgfqpoint{6.228076in}{1.324209in}}%
\pgfpathlineto{\pgfqpoint{6.299242in}{1.320523in}}%
\pgfpathlineto{\pgfqpoint{6.370409in}{1.297289in}}%
\pgfpathlineto{\pgfqpoint{6.441575in}{1.293409in}}%
\pgfpathlineto{\pgfqpoint{6.512741in}{1.270405in}}%
\pgfpathlineto{\pgfqpoint{6.583907in}{1.266260in}}%
\pgfpathlineto{\pgfqpoint{6.655073in}{1.243554in}}%
\pgfpathlineto{\pgfqpoint{6.726240in}{1.239079in}}%
\pgfpathlineto{\pgfqpoint{6.797406in}{1.216734in}}%
\pgfpathlineto{\pgfqpoint{6.868572in}{1.211867in}}%
\pgfpathlineto{\pgfqpoint{6.939738in}{1.189945in}}%
\pgfpathlineto{\pgfqpoint{7.010904in}{1.184626in}}%
\pgfpathlineto{\pgfqpoint{7.082071in}{1.163183in}}%
\pgfpathlineto{\pgfqpoint{7.153237in}{1.157359in}}%
\pgfpathlineto{\pgfqpoint{7.224403in}{1.136447in}}%
\pgfpathlineto{\pgfqpoint{7.295569in}{1.130067in}}%
\pgfpathlineto{\pgfqpoint{7.366736in}{1.109734in}}%
\pgfpathlineto{\pgfqpoint{7.437902in}{1.102753in}}%
\pgfpathlineto{\pgfqpoint{7.509068in}{1.083042in}}%
\pgfpathlineto{\pgfqpoint{7.580234in}{1.075418in}}%
\pgfpathlineto{\pgfqpoint{7.651400in}{1.056370in}}%
\pgfpathlineto{\pgfqpoint{7.722567in}{1.048066in}}%
\pgfpathlineto{\pgfqpoint{7.793733in}{1.029714in}}%
\pgfpathlineto{\pgfqpoint{7.864899in}{1.020698in}}%
\pgfpathlineto{\pgfqpoint{7.936065in}{1.003073in}}%
\pgfpathlineto{\pgfqpoint{8.007231in}{0.993316in}}%
\pgfpathlineto{\pgfqpoint{8.078398in}{0.976443in}}%
\pgfpathlineto{\pgfqpoint{8.149564in}{0.965924in}}%
\pgfpathlineto{\pgfqpoint{8.220730in}{0.949824in}}%
\pgfpathlineto{\pgfqpoint{8.291896in}{0.938523in}}%
\pgfpathlineto{\pgfqpoint{8.363062in}{0.923212in}}%
\pgfpathlineto{\pgfqpoint{8.434229in}{0.911116in}}%
\pgfpathlineto{\pgfqpoint{8.505395in}{0.896605in}}%
\pgfpathlineto{\pgfqpoint{8.576561in}{0.883706in}}%
\pgfpathlineto{\pgfqpoint{8.647727in}{0.870000in}}%
\pgfusepath{stroke}%
\end{pgfscope}%
\begin{pgfscope}%
\pgfpathrectangle{\pgfqpoint{1.250000in}{0.660000in}}{\pgfqpoint{7.750000in}{4.620000in}}%
\pgfusepath{clip}%
\pgfsetbuttcap%
\pgfsetroundjoin%
\pgfsetlinewidth{1.505625pt}%
\definecolor{currentstroke}{rgb}{0.000000,0.000000,0.000000}%
\pgfsetstrokecolor{currentstroke}%
\pgfsetdash{{5.550000pt}{2.400000pt}}{0.000000pt}%
\pgfpathmoveto{\pgfqpoint{1.602273in}{0.870000in}}%
\pgfpathlineto{\pgfqpoint{1.673439in}{0.912424in}}%
\pgfpathlineto{\pgfqpoint{1.744605in}{0.954848in}}%
\pgfpathlineto{\pgfqpoint{1.815771in}{0.997273in}}%
\pgfpathlineto{\pgfqpoint{1.886938in}{1.039697in}}%
\pgfpathlineto{\pgfqpoint{1.958104in}{1.082121in}}%
\pgfpathlineto{\pgfqpoint{2.029270in}{1.124545in}}%
\pgfpathlineto{\pgfqpoint{2.100436in}{1.166970in}}%
\pgfpathlineto{\pgfqpoint{2.171602in}{1.209394in}}%
\pgfpathlineto{\pgfqpoint{2.242769in}{1.251818in}}%
\pgfpathlineto{\pgfqpoint{2.313935in}{1.294242in}}%
\pgfpathlineto{\pgfqpoint{2.385101in}{1.336667in}}%
\pgfpathlineto{\pgfqpoint{2.456267in}{1.379091in}}%
\pgfpathlineto{\pgfqpoint{2.527433in}{1.421515in}}%
\pgfpathlineto{\pgfqpoint{2.598600in}{1.463939in}}%
\pgfpathlineto{\pgfqpoint{2.669766in}{1.506364in}}%
\pgfpathlineto{\pgfqpoint{2.740932in}{1.548788in}}%
\pgfpathlineto{\pgfqpoint{2.812098in}{1.591212in}}%
\pgfpathlineto{\pgfqpoint{2.883264in}{1.633636in}}%
\pgfpathlineto{\pgfqpoint{2.954431in}{1.676061in}}%
\pgfpathlineto{\pgfqpoint{3.025597in}{1.718485in}}%
\pgfpathlineto{\pgfqpoint{3.096763in}{1.760909in}}%
\pgfpathlineto{\pgfqpoint{3.167929in}{1.803333in}}%
\pgfpathlineto{\pgfqpoint{3.239096in}{1.845758in}}%
\pgfpathlineto{\pgfqpoint{3.310262in}{1.888182in}}%
\pgfpathlineto{\pgfqpoint{3.381428in}{1.930606in}}%
\pgfpathlineto{\pgfqpoint{3.452594in}{1.973030in}}%
\pgfpathlineto{\pgfqpoint{3.523760in}{2.015455in}}%
\pgfpathlineto{\pgfqpoint{3.594927in}{2.057879in}}%
\pgfpathlineto{\pgfqpoint{3.666093in}{2.100303in}}%
\pgfpathlineto{\pgfqpoint{3.737259in}{2.142727in}}%
\pgfpathlineto{\pgfqpoint{3.808425in}{2.185152in}}%
\pgfpathlineto{\pgfqpoint{3.879591in}{2.227576in}}%
\pgfpathlineto{\pgfqpoint{3.950758in}{2.270000in}}%
\pgfpathlineto{\pgfqpoint{4.021924in}{2.312424in}}%
\pgfpathlineto{\pgfqpoint{4.093090in}{2.354848in}}%
\pgfpathlineto{\pgfqpoint{4.164256in}{2.397273in}}%
\pgfpathlineto{\pgfqpoint{4.235422in}{2.439697in}}%
\pgfpathlineto{\pgfqpoint{4.306589in}{2.482121in}}%
\pgfpathlineto{\pgfqpoint{4.377755in}{2.524545in}}%
\pgfpathlineto{\pgfqpoint{4.448921in}{2.566970in}}%
\pgfpathlineto{\pgfqpoint{4.520087in}{2.609394in}}%
\pgfpathlineto{\pgfqpoint{4.591253in}{2.651818in}}%
\pgfpathlineto{\pgfqpoint{4.662420in}{2.694242in}}%
\pgfpathlineto{\pgfqpoint{4.733586in}{2.736667in}}%
\pgfpathlineto{\pgfqpoint{4.804752in}{2.779091in}}%
\pgfpathlineto{\pgfqpoint{4.875918in}{2.821515in}}%
\pgfpathlineto{\pgfqpoint{4.947084in}{2.863939in}}%
\pgfpathlineto{\pgfqpoint{5.018251in}{2.906364in}}%
\pgfpathlineto{\pgfqpoint{5.089417in}{2.948788in}}%
\pgfpathlineto{\pgfqpoint{5.160583in}{2.991212in}}%
\pgfpathlineto{\pgfqpoint{5.231749in}{3.033636in}}%
\pgfpathlineto{\pgfqpoint{5.302916in}{3.076061in}}%
\pgfpathlineto{\pgfqpoint{5.374082in}{3.118485in}}%
\pgfpathlineto{\pgfqpoint{5.445248in}{3.160909in}}%
\pgfpathlineto{\pgfqpoint{5.516414in}{3.203333in}}%
\pgfpathlineto{\pgfqpoint{5.587580in}{3.245758in}}%
\pgfpathlineto{\pgfqpoint{5.658747in}{3.288182in}}%
\pgfpathlineto{\pgfqpoint{5.729913in}{3.330606in}}%
\pgfpathlineto{\pgfqpoint{5.801079in}{3.373030in}}%
\pgfpathlineto{\pgfqpoint{5.872245in}{3.415455in}}%
\pgfpathlineto{\pgfqpoint{5.943411in}{3.457879in}}%
\pgfpathlineto{\pgfqpoint{6.014578in}{3.500303in}}%
\pgfpathlineto{\pgfqpoint{6.085744in}{3.542727in}}%
\pgfpathlineto{\pgfqpoint{6.156910in}{3.585152in}}%
\pgfpathlineto{\pgfqpoint{6.228076in}{3.627576in}}%
\pgfpathlineto{\pgfqpoint{6.299242in}{3.670000in}}%
\pgfpathlineto{\pgfqpoint{6.370409in}{3.712424in}}%
\pgfpathlineto{\pgfqpoint{6.441575in}{3.754848in}}%
\pgfpathlineto{\pgfqpoint{6.512741in}{3.797273in}}%
\pgfpathlineto{\pgfqpoint{6.583907in}{3.839697in}}%
\pgfpathlineto{\pgfqpoint{6.655073in}{3.882121in}}%
\pgfpathlineto{\pgfqpoint{6.726240in}{3.924545in}}%
\pgfpathlineto{\pgfqpoint{6.797406in}{3.966970in}}%
\pgfpathlineto{\pgfqpoint{6.868572in}{4.009394in}}%
\pgfpathlineto{\pgfqpoint{6.939738in}{4.051818in}}%
\pgfpathlineto{\pgfqpoint{7.010904in}{4.094242in}}%
\pgfpathlineto{\pgfqpoint{7.082071in}{4.136667in}}%
\pgfpathlineto{\pgfqpoint{7.153237in}{4.179091in}}%
\pgfpathlineto{\pgfqpoint{7.224403in}{4.221515in}}%
\pgfpathlineto{\pgfqpoint{7.295569in}{4.263939in}}%
\pgfpathlineto{\pgfqpoint{7.366736in}{4.306364in}}%
\pgfpathlineto{\pgfqpoint{7.437902in}{4.348788in}}%
\pgfpathlineto{\pgfqpoint{7.509068in}{4.391212in}}%
\pgfpathlineto{\pgfqpoint{7.580234in}{4.433636in}}%
\pgfpathlineto{\pgfqpoint{7.651400in}{4.476061in}}%
\pgfpathlineto{\pgfqpoint{7.722567in}{4.518485in}}%
\pgfpathlineto{\pgfqpoint{7.793733in}{4.560909in}}%
\pgfpathlineto{\pgfqpoint{7.864899in}{4.603333in}}%
\pgfpathlineto{\pgfqpoint{7.936065in}{4.645758in}}%
\pgfpathlineto{\pgfqpoint{8.007231in}{4.688182in}}%
\pgfpathlineto{\pgfqpoint{8.078398in}{4.730606in}}%
\pgfpathlineto{\pgfqpoint{8.149564in}{4.773030in}}%
\pgfpathlineto{\pgfqpoint{8.220730in}{4.815455in}}%
\pgfpathlineto{\pgfqpoint{8.291896in}{4.857879in}}%
\pgfpathlineto{\pgfqpoint{8.363062in}{4.900303in}}%
\pgfpathlineto{\pgfqpoint{8.434229in}{4.942727in}}%
\pgfpathlineto{\pgfqpoint{8.505395in}{4.985152in}}%
\pgfpathlineto{\pgfqpoint{8.576561in}{5.027576in}}%
\pgfpathlineto{\pgfqpoint{8.647727in}{5.070000in}}%
\pgfusepath{stroke}%
\end{pgfscope}%
\begin{pgfscope}%
\pgfsetrectcap%
\pgfsetmiterjoin%
\pgfsetlinewidth{0.803000pt}%
\definecolor{currentstroke}{rgb}{0.000000,0.000000,0.000000}%
\pgfsetstrokecolor{currentstroke}%
\pgfsetdash{}{0pt}%
\pgfpathmoveto{\pgfqpoint{1.250000in}{0.660000in}}%
\pgfpathlineto{\pgfqpoint{1.250000in}{5.280000in}}%
\pgfusepath{stroke}%
\end{pgfscope}%
\begin{pgfscope}%
\pgfsetrectcap%
\pgfsetmiterjoin%
\pgfsetlinewidth{0.803000pt}%
\definecolor{currentstroke}{rgb}{0.000000,0.000000,0.000000}%
\pgfsetstrokecolor{currentstroke}%
\pgfsetdash{}{0pt}%
\pgfpathmoveto{\pgfqpoint{9.000000in}{0.660000in}}%
\pgfpathlineto{\pgfqpoint{9.000000in}{5.280000in}}%
\pgfusepath{stroke}%
\end{pgfscope}%
\begin{pgfscope}%
\pgfsetrectcap%
\pgfsetmiterjoin%
\pgfsetlinewidth{0.803000pt}%
\definecolor{currentstroke}{rgb}{0.000000,0.000000,0.000000}%
\pgfsetstrokecolor{currentstroke}%
\pgfsetdash{}{0pt}%
\pgfpathmoveto{\pgfqpoint{1.250000in}{0.660000in}}%
\pgfpathlineto{\pgfqpoint{9.000000in}{0.660000in}}%
\pgfusepath{stroke}%
\end{pgfscope}%
\begin{pgfscope}%
\pgfsetrectcap%
\pgfsetmiterjoin%
\pgfsetlinewidth{0.803000pt}%
\definecolor{currentstroke}{rgb}{0.000000,0.000000,0.000000}%
\pgfsetstrokecolor{currentstroke}%
\pgfsetdash{}{0pt}%
\pgfpathmoveto{\pgfqpoint{1.250000in}{5.280000in}}%
\pgfpathlineto{\pgfqpoint{9.000000in}{5.280000in}}%
\pgfusepath{stroke}%
\end{pgfscope}%
\begin{pgfscope}%
\definecolor{textcolor}{rgb}{0.000000,0.000000,0.000000}%
\pgfsetstrokecolor{textcolor}%
\pgfsetfillcolor{textcolor}%
\pgftext[x=5.125000in,y=5.394325in,,base]{\color{textcolor}\sffamily\fontsize{12.000000}{14.400000}\selectfont Pointwise Convergence of Sum of Sequence \(\displaystyle f_n(x) = \frac{\sin(nx)}{n}\)}%
\end{pgfscope}%
\begin{pgfscope}%
\pgfsetbuttcap%
\pgfsetmiterjoin%
\definecolor{currentfill}{rgb}{1.000000,1.000000,1.000000}%
\pgfsetfillcolor{currentfill}%
\pgfsetfillopacity{0.800000}%
\pgfsetlinewidth{1.003750pt}%
\definecolor{currentstroke}{rgb}{0.800000,0.800000,0.800000}%
\pgfsetstrokecolor{currentstroke}%
\pgfsetstrokeopacity{0.800000}%
\pgfsetdash{}{0pt}%
\pgfpathmoveto{\pgfqpoint{1.347222in}{3.741888in}}%
\pgfpathlineto{\pgfqpoint{2.747477in}{3.741888in}}%
\pgfpathquadraticcurveto{\pgfqpoint{2.775255in}{3.741888in}}{\pgfqpoint{2.775255in}{3.769666in}}%
\pgfpathlineto{\pgfqpoint{2.775255in}{5.182778in}}%
\pgfpathquadraticcurveto{\pgfqpoint{2.775255in}{5.210556in}}{\pgfqpoint{2.747477in}{5.210556in}}%
\pgfpathlineto{\pgfqpoint{1.347222in}{5.210556in}}%
\pgfpathquadraticcurveto{\pgfqpoint{1.319444in}{5.210556in}}{\pgfqpoint{1.319444in}{5.182778in}}%
\pgfpathlineto{\pgfqpoint{1.319444in}{3.769666in}}%
\pgfpathquadraticcurveto{\pgfqpoint{1.319444in}{3.741888in}}{\pgfqpoint{1.347222in}{3.741888in}}%
\pgfpathlineto{\pgfqpoint{1.347222in}{3.741888in}}%
\pgfpathclose%
\pgfusepath{stroke,fill}%
\end{pgfscope}%
\begin{pgfscope}%
\pgfsetrectcap%
\pgfsetroundjoin%
\pgfsetlinewidth{1.505625pt}%
\definecolor{currentstroke}{rgb}{0.121569,0.466667,0.705882}%
\pgfsetstrokecolor{currentstroke}%
\pgfsetdash{}{0pt}%
\pgfpathmoveto{\pgfqpoint{1.375000in}{5.098088in}}%
\pgfpathlineto{\pgfqpoint{1.513889in}{5.098088in}}%
\pgfpathlineto{\pgfqpoint{1.652778in}{5.098088in}}%
\pgfusepath{stroke}%
\end{pgfscope}%
\begin{pgfscope}%
\definecolor{textcolor}{rgb}{0.000000,0.000000,0.000000}%
\pgfsetstrokecolor{textcolor}%
\pgfsetfillcolor{textcolor}%
\pgftext[x=1.763889in,y=5.049477in,left,base]{\color{textcolor}\sffamily\fontsize{10.000000}{12.000000}\selectfont N = 1}%
\end{pgfscope}%
\begin{pgfscope}%
\pgfsetrectcap%
\pgfsetroundjoin%
\pgfsetlinewidth{1.505625pt}%
\definecolor{currentstroke}{rgb}{1.000000,0.498039,0.054902}%
\pgfsetstrokecolor{currentstroke}%
\pgfsetdash{}{0pt}%
\pgfpathmoveto{\pgfqpoint{1.375000in}{4.894231in}}%
\pgfpathlineto{\pgfqpoint{1.513889in}{4.894231in}}%
\pgfpathlineto{\pgfqpoint{1.652778in}{4.894231in}}%
\pgfusepath{stroke}%
\end{pgfscope}%
\begin{pgfscope}%
\definecolor{textcolor}{rgb}{0.000000,0.000000,0.000000}%
\pgfsetstrokecolor{textcolor}%
\pgfsetfillcolor{textcolor}%
\pgftext[x=1.763889in,y=4.845620in,left,base]{\color{textcolor}\sffamily\fontsize{10.000000}{12.000000}\selectfont N = 5}%
\end{pgfscope}%
\begin{pgfscope}%
\pgfsetrectcap%
\pgfsetroundjoin%
\pgfsetlinewidth{1.505625pt}%
\definecolor{currentstroke}{rgb}{0.172549,0.627451,0.172549}%
\pgfsetstrokecolor{currentstroke}%
\pgfsetdash{}{0pt}%
\pgfpathmoveto{\pgfqpoint{1.375000in}{4.690374in}}%
\pgfpathlineto{\pgfqpoint{1.513889in}{4.690374in}}%
\pgfpathlineto{\pgfqpoint{1.652778in}{4.690374in}}%
\pgfusepath{stroke}%
\end{pgfscope}%
\begin{pgfscope}%
\definecolor{textcolor}{rgb}{0.000000,0.000000,0.000000}%
\pgfsetstrokecolor{textcolor}%
\pgfsetfillcolor{textcolor}%
\pgftext[x=1.763889in,y=4.641762in,left,base]{\color{textcolor}\sffamily\fontsize{10.000000}{12.000000}\selectfont N = 10}%
\end{pgfscope}%
\begin{pgfscope}%
\pgfsetrectcap%
\pgfsetroundjoin%
\pgfsetlinewidth{1.505625pt}%
\definecolor{currentstroke}{rgb}{0.839216,0.152941,0.156863}%
\pgfsetstrokecolor{currentstroke}%
\pgfsetdash{}{0pt}%
\pgfpathmoveto{\pgfqpoint{1.375000in}{4.486516in}}%
\pgfpathlineto{\pgfqpoint{1.513889in}{4.486516in}}%
\pgfpathlineto{\pgfqpoint{1.652778in}{4.486516in}}%
\pgfusepath{stroke}%
\end{pgfscope}%
\begin{pgfscope}%
\definecolor{textcolor}{rgb}{0.000000,0.000000,0.000000}%
\pgfsetstrokecolor{textcolor}%
\pgfsetfillcolor{textcolor}%
\pgftext[x=1.763889in,y=4.437905in,left,base]{\color{textcolor}\sffamily\fontsize{10.000000}{12.000000}\selectfont N = 20}%
\end{pgfscope}%
\begin{pgfscope}%
\pgfsetrectcap%
\pgfsetroundjoin%
\pgfsetlinewidth{1.505625pt}%
\definecolor{currentstroke}{rgb}{0.580392,0.403922,0.741176}%
\pgfsetstrokecolor{currentstroke}%
\pgfsetdash{}{0pt}%
\pgfpathmoveto{\pgfqpoint{1.375000in}{4.282659in}}%
\pgfpathlineto{\pgfqpoint{1.513889in}{4.282659in}}%
\pgfpathlineto{\pgfqpoint{1.652778in}{4.282659in}}%
\pgfusepath{stroke}%
\end{pgfscope}%
\begin{pgfscope}%
\definecolor{textcolor}{rgb}{0.000000,0.000000,0.000000}%
\pgfsetstrokecolor{textcolor}%
\pgfsetfillcolor{textcolor}%
\pgftext[x=1.763889in,y=4.234048in,left,base]{\color{textcolor}\sffamily\fontsize{10.000000}{12.000000}\selectfont N = 50}%
\end{pgfscope}%
\begin{pgfscope}%
\pgfsetrectcap%
\pgfsetroundjoin%
\pgfsetlinewidth{1.505625pt}%
\definecolor{currentstroke}{rgb}{0.549020,0.337255,0.294118}%
\pgfsetstrokecolor{currentstroke}%
\pgfsetdash{}{0pt}%
\pgfpathmoveto{\pgfqpoint{1.375000in}{4.078802in}}%
\pgfpathlineto{\pgfqpoint{1.513889in}{4.078802in}}%
\pgfpathlineto{\pgfqpoint{1.652778in}{4.078802in}}%
\pgfusepath{stroke}%
\end{pgfscope}%
\begin{pgfscope}%
\definecolor{textcolor}{rgb}{0.000000,0.000000,0.000000}%
\pgfsetstrokecolor{textcolor}%
\pgfsetfillcolor{textcolor}%
\pgftext[x=1.763889in,y=4.030191in,left,base]{\color{textcolor}\sffamily\fontsize{10.000000}{12.000000}\selectfont N = 100}%
\end{pgfscope}%
\begin{pgfscope}%
\pgfsetbuttcap%
\pgfsetroundjoin%
\pgfsetlinewidth{1.505625pt}%
\definecolor{currentstroke}{rgb}{0.000000,0.000000,0.000000}%
\pgfsetstrokecolor{currentstroke}%
\pgfsetdash{{5.550000pt}{2.400000pt}}{0.000000pt}%
\pgfpathmoveto{\pgfqpoint{1.375000in}{3.874945in}}%
\pgfpathlineto{\pgfqpoint{1.513889in}{3.874945in}}%
\pgfpathlineto{\pgfqpoint{1.652778in}{3.874945in}}%
\pgfusepath{stroke}%
\end{pgfscope}%
\begin{pgfscope}%
\definecolor{textcolor}{rgb}{0.000000,0.000000,0.000000}%
\pgfsetstrokecolor{textcolor}%
\pgfsetfillcolor{textcolor}%
\pgftext[x=1.763889in,y=3.826334in,left,base]{\color{textcolor}\sffamily\fontsize{10.000000}{12.000000}\selectfont Limit function}%
\end{pgfscope}%
\end{pgfpicture}%
\makeatother%
\endgroup%
}
  \label{fig:example}
\end{figure}

\vspace{-10pt}

\proposition{}{Let $f_n:J\rightarrow\mathbb{R}$ be a sequence of functions defined on an interval J of $\mathbb{R}$, which converges pointwise to a function $f$.

(1) If for all n, the function $f_n$ is increasing (respectively decreasing), the function $f$ is increasing (not necessarily strictly) (respectively decreasing (not necessarily strictly)), \\

(2) Assume that the interval $J$ is centered at 0. If for all $n$ the functions $f_n$ are even (respectively odd), the function $f$ is even (respectively odd), \\

(3) Assume that $J=\mathbb{R}$ and let $T>0$. If for all $n$, the function $f_n$ is $T$-periodic, then the function $f$ is $T$-periodic.
}

\proposition{}{Let $g_n:J\rightarrow\mathbb{R}$ be a sequence of functions. Assume that the series $\sum g_n$ converges pointwise to a function $g$.

(1) If for all n, the function $g_n$ is increasing (respectively decreasing), the function $g$ is increasing (not necessarily strictly) (respectively decreasing (not necessarily strictly)), \\

(2) Assume that the interval $J$ is centered at 0. If for all $n$ the function $g_n$ is even (respectively odd), the function $g$ is even (respectively odd), \\

(3) Assume that $J=\mathbb{R}$ and let $T>0$. If for all $n$, the function $g_n$ is $T$-periodic, then the function $g$ is $T$-periodic.}

\subsection{Uniform convergence}

\definition{Uniform convergence}{
(1) We say that a sequence $(f_n)_{n\in\mathbb{N}}$ of function \textbf{converges uniformly} to a function $f$ if for all $\varepsilon > 0$ there exists $N$ such that for all $n\geq N$ and for all $x\in\Omega$ we have $$|f_n(x)-f(x)| \leq \varepsilon.$$

(2) We say that a series of functions $\sum f_n$ is \textbf{uniformly convergent} if the sequence of partial sums $\displaystyle (\sum_{n\leq N} f_n)_N$ is uniformly convergent.
}

\vspace{-10pt}

\begin{figure}[htbp]
    \centering
  \scalebox{0.5}{%% Creator: Matplotlib, PGF backend
%%
%% To include the figure in your LaTeX document, write
%%   \input{<filename>.pgf}
%%
%% Make sure the required packages are loaded in your preamble
%%   \usepackage{pgf}
%%
%% Also ensure that all the required font packages are loaded; for instance,
%% the lmodern package is sometimes necessary when using math font.
%%   \usepackage{lmodern}
%%
%% Figures using additional raster images can only be included by \input if
%% they are in the same directory as the main LaTeX file. For loading figures
%% from other directories you can use the `import` package
%%   \usepackage{import}
%%
%% and then include the figures with
%%   \import{<path to file>}{<filename>.pgf}
%%
%% Matplotlib used the following preamble
%%   
%%   \usepackage{fontspec}
%%   \setmainfont{DejaVuSerif.ttf}[Path=\detokenize{/usr/local/Anaconda3-2023.07-1/lib/python3.11/site-packages/matplotlib/mpl-data/fonts/ttf/}]
%%   \setsansfont{DejaVuSans.ttf}[Path=\detokenize{/usr/local/Anaconda3-2023.07-1/lib/python3.11/site-packages/matplotlib/mpl-data/fonts/ttf/}]
%%   \setmonofont{DejaVuSansMono.ttf}[Path=\detokenize{/usr/local/Anaconda3-2023.07-1/lib/python3.11/site-packages/matplotlib/mpl-data/fonts/ttf/}]
%%   \makeatletter\@ifpackageloaded{underscore}{}{\usepackage[strings]{underscore}}\makeatother
%%
\begingroup%
\makeatletter%
\begin{pgfpicture}%
\pgfpathrectangle{\pgfpointorigin}{\pgfqpoint{6.400000in}{4.800000in}}%
\pgfusepath{use as bounding box, clip}%
\begin{pgfscope}%
\pgfsetbuttcap%
\pgfsetmiterjoin%
\definecolor{currentfill}{rgb}{1.000000,1.000000,1.000000}%
\pgfsetfillcolor{currentfill}%
\pgfsetlinewidth{0.000000pt}%
\definecolor{currentstroke}{rgb}{1.000000,1.000000,1.000000}%
\pgfsetstrokecolor{currentstroke}%
\pgfsetdash{}{0pt}%
\pgfpathmoveto{\pgfqpoint{0.000000in}{0.000000in}}%
\pgfpathlineto{\pgfqpoint{6.400000in}{0.000000in}}%
\pgfpathlineto{\pgfqpoint{6.400000in}{4.800000in}}%
\pgfpathlineto{\pgfqpoint{0.000000in}{4.800000in}}%
\pgfpathlineto{\pgfqpoint{0.000000in}{0.000000in}}%
\pgfpathclose%
\pgfusepath{fill}%
\end{pgfscope}%
\begin{pgfscope}%
\pgfsetbuttcap%
\pgfsetmiterjoin%
\definecolor{currentfill}{rgb}{1.000000,1.000000,1.000000}%
\pgfsetfillcolor{currentfill}%
\pgfsetlinewidth{0.000000pt}%
\definecolor{currentstroke}{rgb}{0.000000,0.000000,0.000000}%
\pgfsetstrokecolor{currentstroke}%
\pgfsetstrokeopacity{0.000000}%
\pgfsetdash{}{0pt}%
\pgfpathmoveto{\pgfqpoint{0.800000in}{0.528000in}}%
\pgfpathlineto{\pgfqpoint{5.760000in}{0.528000in}}%
\pgfpathlineto{\pgfqpoint{5.760000in}{4.224000in}}%
\pgfpathlineto{\pgfqpoint{0.800000in}{4.224000in}}%
\pgfpathlineto{\pgfqpoint{0.800000in}{0.528000in}}%
\pgfpathclose%
\pgfusepath{fill}%
\end{pgfscope}%
\begin{pgfscope}%
\pgfsetbuttcap%
\pgfsetroundjoin%
\definecolor{currentfill}{rgb}{0.000000,0.000000,0.000000}%
\pgfsetfillcolor{currentfill}%
\pgfsetlinewidth{0.803000pt}%
\definecolor{currentstroke}{rgb}{0.000000,0.000000,0.000000}%
\pgfsetstrokecolor{currentstroke}%
\pgfsetdash{}{0pt}%
\pgfsys@defobject{currentmarker}{\pgfqpoint{0.000000in}{-0.048611in}}{\pgfqpoint{0.000000in}{0.000000in}}{%
\pgfpathmoveto{\pgfqpoint{0.000000in}{0.000000in}}%
\pgfpathlineto{\pgfqpoint{0.000000in}{-0.048611in}}%
\pgfusepath{stroke,fill}%
}%
\begin{pgfscope}%
\pgfsys@transformshift{1.025455in}{0.528000in}%
\pgfsys@useobject{currentmarker}{}%
\end{pgfscope}%
\end{pgfscope}%
\begin{pgfscope}%
\definecolor{textcolor}{rgb}{0.000000,0.000000,0.000000}%
\pgfsetstrokecolor{textcolor}%
\pgfsetfillcolor{textcolor}%
\pgftext[x=1.025455in,y=0.430778in,,top]{\color{textcolor}\sffamily\fontsize{10.000000}{12.000000}\selectfont \ensuremath{-}1.00}%
\end{pgfscope}%
\begin{pgfscope}%
\pgfsetbuttcap%
\pgfsetroundjoin%
\definecolor{currentfill}{rgb}{0.000000,0.000000,0.000000}%
\pgfsetfillcolor{currentfill}%
\pgfsetlinewidth{0.803000pt}%
\definecolor{currentstroke}{rgb}{0.000000,0.000000,0.000000}%
\pgfsetstrokecolor{currentstroke}%
\pgfsetdash{}{0pt}%
\pgfsys@defobject{currentmarker}{\pgfqpoint{0.000000in}{-0.048611in}}{\pgfqpoint{0.000000in}{0.000000in}}{%
\pgfpathmoveto{\pgfqpoint{0.000000in}{0.000000in}}%
\pgfpathlineto{\pgfqpoint{0.000000in}{-0.048611in}}%
\pgfusepath{stroke,fill}%
}%
\begin{pgfscope}%
\pgfsys@transformshift{1.589091in}{0.528000in}%
\pgfsys@useobject{currentmarker}{}%
\end{pgfscope}%
\end{pgfscope}%
\begin{pgfscope}%
\definecolor{textcolor}{rgb}{0.000000,0.000000,0.000000}%
\pgfsetstrokecolor{textcolor}%
\pgfsetfillcolor{textcolor}%
\pgftext[x=1.589091in,y=0.430778in,,top]{\color{textcolor}\sffamily\fontsize{10.000000}{12.000000}\selectfont \ensuremath{-}0.75}%
\end{pgfscope}%
\begin{pgfscope}%
\pgfsetbuttcap%
\pgfsetroundjoin%
\definecolor{currentfill}{rgb}{0.000000,0.000000,0.000000}%
\pgfsetfillcolor{currentfill}%
\pgfsetlinewidth{0.803000pt}%
\definecolor{currentstroke}{rgb}{0.000000,0.000000,0.000000}%
\pgfsetstrokecolor{currentstroke}%
\pgfsetdash{}{0pt}%
\pgfsys@defobject{currentmarker}{\pgfqpoint{0.000000in}{-0.048611in}}{\pgfqpoint{0.000000in}{0.000000in}}{%
\pgfpathmoveto{\pgfqpoint{0.000000in}{0.000000in}}%
\pgfpathlineto{\pgfqpoint{0.000000in}{-0.048611in}}%
\pgfusepath{stroke,fill}%
}%
\begin{pgfscope}%
\pgfsys@transformshift{2.152727in}{0.528000in}%
\pgfsys@useobject{currentmarker}{}%
\end{pgfscope}%
\end{pgfscope}%
\begin{pgfscope}%
\definecolor{textcolor}{rgb}{0.000000,0.000000,0.000000}%
\pgfsetstrokecolor{textcolor}%
\pgfsetfillcolor{textcolor}%
\pgftext[x=2.152727in,y=0.430778in,,top]{\color{textcolor}\sffamily\fontsize{10.000000}{12.000000}\selectfont \ensuremath{-}0.50}%
\end{pgfscope}%
\begin{pgfscope}%
\pgfsetbuttcap%
\pgfsetroundjoin%
\definecolor{currentfill}{rgb}{0.000000,0.000000,0.000000}%
\pgfsetfillcolor{currentfill}%
\pgfsetlinewidth{0.803000pt}%
\definecolor{currentstroke}{rgb}{0.000000,0.000000,0.000000}%
\pgfsetstrokecolor{currentstroke}%
\pgfsetdash{}{0pt}%
\pgfsys@defobject{currentmarker}{\pgfqpoint{0.000000in}{-0.048611in}}{\pgfqpoint{0.000000in}{0.000000in}}{%
\pgfpathmoveto{\pgfqpoint{0.000000in}{0.000000in}}%
\pgfpathlineto{\pgfqpoint{0.000000in}{-0.048611in}}%
\pgfusepath{stroke,fill}%
}%
\begin{pgfscope}%
\pgfsys@transformshift{2.716364in}{0.528000in}%
\pgfsys@useobject{currentmarker}{}%
\end{pgfscope}%
\end{pgfscope}%
\begin{pgfscope}%
\definecolor{textcolor}{rgb}{0.000000,0.000000,0.000000}%
\pgfsetstrokecolor{textcolor}%
\pgfsetfillcolor{textcolor}%
\pgftext[x=2.716364in,y=0.430778in,,top]{\color{textcolor}\sffamily\fontsize{10.000000}{12.000000}\selectfont \ensuremath{-}0.25}%
\end{pgfscope}%
\begin{pgfscope}%
\pgfsetbuttcap%
\pgfsetroundjoin%
\definecolor{currentfill}{rgb}{0.000000,0.000000,0.000000}%
\pgfsetfillcolor{currentfill}%
\pgfsetlinewidth{0.803000pt}%
\definecolor{currentstroke}{rgb}{0.000000,0.000000,0.000000}%
\pgfsetstrokecolor{currentstroke}%
\pgfsetdash{}{0pt}%
\pgfsys@defobject{currentmarker}{\pgfqpoint{0.000000in}{-0.048611in}}{\pgfqpoint{0.000000in}{0.000000in}}{%
\pgfpathmoveto{\pgfqpoint{0.000000in}{0.000000in}}%
\pgfpathlineto{\pgfqpoint{0.000000in}{-0.048611in}}%
\pgfusepath{stroke,fill}%
}%
\begin{pgfscope}%
\pgfsys@transformshift{3.280000in}{0.528000in}%
\pgfsys@useobject{currentmarker}{}%
\end{pgfscope}%
\end{pgfscope}%
\begin{pgfscope}%
\definecolor{textcolor}{rgb}{0.000000,0.000000,0.000000}%
\pgfsetstrokecolor{textcolor}%
\pgfsetfillcolor{textcolor}%
\pgftext[x=3.280000in,y=0.430778in,,top]{\color{textcolor}\sffamily\fontsize{10.000000}{12.000000}\selectfont 0.00}%
\end{pgfscope}%
\begin{pgfscope}%
\pgfsetbuttcap%
\pgfsetroundjoin%
\definecolor{currentfill}{rgb}{0.000000,0.000000,0.000000}%
\pgfsetfillcolor{currentfill}%
\pgfsetlinewidth{0.803000pt}%
\definecolor{currentstroke}{rgb}{0.000000,0.000000,0.000000}%
\pgfsetstrokecolor{currentstroke}%
\pgfsetdash{}{0pt}%
\pgfsys@defobject{currentmarker}{\pgfqpoint{0.000000in}{-0.048611in}}{\pgfqpoint{0.000000in}{0.000000in}}{%
\pgfpathmoveto{\pgfqpoint{0.000000in}{0.000000in}}%
\pgfpathlineto{\pgfqpoint{0.000000in}{-0.048611in}}%
\pgfusepath{stroke,fill}%
}%
\begin{pgfscope}%
\pgfsys@transformshift{3.843636in}{0.528000in}%
\pgfsys@useobject{currentmarker}{}%
\end{pgfscope}%
\end{pgfscope}%
\begin{pgfscope}%
\definecolor{textcolor}{rgb}{0.000000,0.000000,0.000000}%
\pgfsetstrokecolor{textcolor}%
\pgfsetfillcolor{textcolor}%
\pgftext[x=3.843636in,y=0.430778in,,top]{\color{textcolor}\sffamily\fontsize{10.000000}{12.000000}\selectfont 0.25}%
\end{pgfscope}%
\begin{pgfscope}%
\pgfsetbuttcap%
\pgfsetroundjoin%
\definecolor{currentfill}{rgb}{0.000000,0.000000,0.000000}%
\pgfsetfillcolor{currentfill}%
\pgfsetlinewidth{0.803000pt}%
\definecolor{currentstroke}{rgb}{0.000000,0.000000,0.000000}%
\pgfsetstrokecolor{currentstroke}%
\pgfsetdash{}{0pt}%
\pgfsys@defobject{currentmarker}{\pgfqpoint{0.000000in}{-0.048611in}}{\pgfqpoint{0.000000in}{0.000000in}}{%
\pgfpathmoveto{\pgfqpoint{0.000000in}{0.000000in}}%
\pgfpathlineto{\pgfqpoint{0.000000in}{-0.048611in}}%
\pgfusepath{stroke,fill}%
}%
\begin{pgfscope}%
\pgfsys@transformshift{4.407273in}{0.528000in}%
\pgfsys@useobject{currentmarker}{}%
\end{pgfscope}%
\end{pgfscope}%
\begin{pgfscope}%
\definecolor{textcolor}{rgb}{0.000000,0.000000,0.000000}%
\pgfsetstrokecolor{textcolor}%
\pgfsetfillcolor{textcolor}%
\pgftext[x=4.407273in,y=0.430778in,,top]{\color{textcolor}\sffamily\fontsize{10.000000}{12.000000}\selectfont 0.50}%
\end{pgfscope}%
\begin{pgfscope}%
\pgfsetbuttcap%
\pgfsetroundjoin%
\definecolor{currentfill}{rgb}{0.000000,0.000000,0.000000}%
\pgfsetfillcolor{currentfill}%
\pgfsetlinewidth{0.803000pt}%
\definecolor{currentstroke}{rgb}{0.000000,0.000000,0.000000}%
\pgfsetstrokecolor{currentstroke}%
\pgfsetdash{}{0pt}%
\pgfsys@defobject{currentmarker}{\pgfqpoint{0.000000in}{-0.048611in}}{\pgfqpoint{0.000000in}{0.000000in}}{%
\pgfpathmoveto{\pgfqpoint{0.000000in}{0.000000in}}%
\pgfpathlineto{\pgfqpoint{0.000000in}{-0.048611in}}%
\pgfusepath{stroke,fill}%
}%
\begin{pgfscope}%
\pgfsys@transformshift{4.970909in}{0.528000in}%
\pgfsys@useobject{currentmarker}{}%
\end{pgfscope}%
\end{pgfscope}%
\begin{pgfscope}%
\definecolor{textcolor}{rgb}{0.000000,0.000000,0.000000}%
\pgfsetstrokecolor{textcolor}%
\pgfsetfillcolor{textcolor}%
\pgftext[x=4.970909in,y=0.430778in,,top]{\color{textcolor}\sffamily\fontsize{10.000000}{12.000000}\selectfont 0.75}%
\end{pgfscope}%
\begin{pgfscope}%
\pgfsetbuttcap%
\pgfsetroundjoin%
\definecolor{currentfill}{rgb}{0.000000,0.000000,0.000000}%
\pgfsetfillcolor{currentfill}%
\pgfsetlinewidth{0.803000pt}%
\definecolor{currentstroke}{rgb}{0.000000,0.000000,0.000000}%
\pgfsetstrokecolor{currentstroke}%
\pgfsetdash{}{0pt}%
\pgfsys@defobject{currentmarker}{\pgfqpoint{0.000000in}{-0.048611in}}{\pgfqpoint{0.000000in}{0.000000in}}{%
\pgfpathmoveto{\pgfqpoint{0.000000in}{0.000000in}}%
\pgfpathlineto{\pgfqpoint{0.000000in}{-0.048611in}}%
\pgfusepath{stroke,fill}%
}%
\begin{pgfscope}%
\pgfsys@transformshift{5.534545in}{0.528000in}%
\pgfsys@useobject{currentmarker}{}%
\end{pgfscope}%
\end{pgfscope}%
\begin{pgfscope}%
\definecolor{textcolor}{rgb}{0.000000,0.000000,0.000000}%
\pgfsetstrokecolor{textcolor}%
\pgfsetfillcolor{textcolor}%
\pgftext[x=5.534545in,y=0.430778in,,top]{\color{textcolor}\sffamily\fontsize{10.000000}{12.000000}\selectfont 1.00}%
\end{pgfscope}%
\begin{pgfscope}%
\definecolor{textcolor}{rgb}{0.000000,0.000000,0.000000}%
\pgfsetstrokecolor{textcolor}%
\pgfsetfillcolor{textcolor}%
\pgftext[x=3.280000in,y=0.240809in,,top]{\color{textcolor}\sffamily\fontsize{10.000000}{12.000000}\selectfont x}%
\end{pgfscope}%
\begin{pgfscope}%
\pgfsetbuttcap%
\pgfsetroundjoin%
\definecolor{currentfill}{rgb}{0.000000,0.000000,0.000000}%
\pgfsetfillcolor{currentfill}%
\pgfsetlinewidth{0.803000pt}%
\definecolor{currentstroke}{rgb}{0.000000,0.000000,0.000000}%
\pgfsetstrokecolor{currentstroke}%
\pgfsetdash{}{0pt}%
\pgfsys@defobject{currentmarker}{\pgfqpoint{-0.048611in}{0.000000in}}{\pgfqpoint{-0.000000in}{0.000000in}}{%
\pgfpathmoveto{\pgfqpoint{-0.000000in}{0.000000in}}%
\pgfpathlineto{\pgfqpoint{-0.048611in}{0.000000in}}%
\pgfusepath{stroke,fill}%
}%
\begin{pgfscope}%
\pgfsys@transformshift{0.800000in}{0.696000in}%
\pgfsys@useobject{currentmarker}{}%
\end{pgfscope}%
\end{pgfscope}%
\begin{pgfscope}%
\definecolor{textcolor}{rgb}{0.000000,0.000000,0.000000}%
\pgfsetstrokecolor{textcolor}%
\pgfsetfillcolor{textcolor}%
\pgftext[x=0.481898in, y=0.643238in, left, base]{\color{textcolor}\sffamily\fontsize{10.000000}{12.000000}\selectfont 0.0}%
\end{pgfscope}%
\begin{pgfscope}%
\pgfsetbuttcap%
\pgfsetroundjoin%
\definecolor{currentfill}{rgb}{0.000000,0.000000,0.000000}%
\pgfsetfillcolor{currentfill}%
\pgfsetlinewidth{0.803000pt}%
\definecolor{currentstroke}{rgb}{0.000000,0.000000,0.000000}%
\pgfsetstrokecolor{currentstroke}%
\pgfsetdash{}{0pt}%
\pgfsys@defobject{currentmarker}{\pgfqpoint{-0.048611in}{0.000000in}}{\pgfqpoint{-0.000000in}{0.000000in}}{%
\pgfpathmoveto{\pgfqpoint{-0.000000in}{0.000000in}}%
\pgfpathlineto{\pgfqpoint{-0.048611in}{0.000000in}}%
\pgfusepath{stroke,fill}%
}%
\begin{pgfscope}%
\pgfsys@transformshift{0.800000in}{1.368004in}%
\pgfsys@useobject{currentmarker}{}%
\end{pgfscope}%
\end{pgfscope}%
\begin{pgfscope}%
\definecolor{textcolor}{rgb}{0.000000,0.000000,0.000000}%
\pgfsetstrokecolor{textcolor}%
\pgfsetfillcolor{textcolor}%
\pgftext[x=0.481898in, y=1.315243in, left, base]{\color{textcolor}\sffamily\fontsize{10.000000}{12.000000}\selectfont 0.2}%
\end{pgfscope}%
\begin{pgfscope}%
\pgfsetbuttcap%
\pgfsetroundjoin%
\definecolor{currentfill}{rgb}{0.000000,0.000000,0.000000}%
\pgfsetfillcolor{currentfill}%
\pgfsetlinewidth{0.803000pt}%
\definecolor{currentstroke}{rgb}{0.000000,0.000000,0.000000}%
\pgfsetstrokecolor{currentstroke}%
\pgfsetdash{}{0pt}%
\pgfsys@defobject{currentmarker}{\pgfqpoint{-0.048611in}{0.000000in}}{\pgfqpoint{-0.000000in}{0.000000in}}{%
\pgfpathmoveto{\pgfqpoint{-0.000000in}{0.000000in}}%
\pgfpathlineto{\pgfqpoint{-0.048611in}{0.000000in}}%
\pgfusepath{stroke,fill}%
}%
\begin{pgfscope}%
\pgfsys@transformshift{0.800000in}{2.040008in}%
\pgfsys@useobject{currentmarker}{}%
\end{pgfscope}%
\end{pgfscope}%
\begin{pgfscope}%
\definecolor{textcolor}{rgb}{0.000000,0.000000,0.000000}%
\pgfsetstrokecolor{textcolor}%
\pgfsetfillcolor{textcolor}%
\pgftext[x=0.481898in, y=1.987247in, left, base]{\color{textcolor}\sffamily\fontsize{10.000000}{12.000000}\selectfont 0.4}%
\end{pgfscope}%
\begin{pgfscope}%
\pgfsetbuttcap%
\pgfsetroundjoin%
\definecolor{currentfill}{rgb}{0.000000,0.000000,0.000000}%
\pgfsetfillcolor{currentfill}%
\pgfsetlinewidth{0.803000pt}%
\definecolor{currentstroke}{rgb}{0.000000,0.000000,0.000000}%
\pgfsetstrokecolor{currentstroke}%
\pgfsetdash{}{0pt}%
\pgfsys@defobject{currentmarker}{\pgfqpoint{-0.048611in}{0.000000in}}{\pgfqpoint{-0.000000in}{0.000000in}}{%
\pgfpathmoveto{\pgfqpoint{-0.000000in}{0.000000in}}%
\pgfpathlineto{\pgfqpoint{-0.048611in}{0.000000in}}%
\pgfusepath{stroke,fill}%
}%
\begin{pgfscope}%
\pgfsys@transformshift{0.800000in}{2.712013in}%
\pgfsys@useobject{currentmarker}{}%
\end{pgfscope}%
\end{pgfscope}%
\begin{pgfscope}%
\definecolor{textcolor}{rgb}{0.000000,0.000000,0.000000}%
\pgfsetstrokecolor{textcolor}%
\pgfsetfillcolor{textcolor}%
\pgftext[x=0.481898in, y=2.659251in, left, base]{\color{textcolor}\sffamily\fontsize{10.000000}{12.000000}\selectfont 0.6}%
\end{pgfscope}%
\begin{pgfscope}%
\pgfsetbuttcap%
\pgfsetroundjoin%
\definecolor{currentfill}{rgb}{0.000000,0.000000,0.000000}%
\pgfsetfillcolor{currentfill}%
\pgfsetlinewidth{0.803000pt}%
\definecolor{currentstroke}{rgb}{0.000000,0.000000,0.000000}%
\pgfsetstrokecolor{currentstroke}%
\pgfsetdash{}{0pt}%
\pgfsys@defobject{currentmarker}{\pgfqpoint{-0.048611in}{0.000000in}}{\pgfqpoint{-0.000000in}{0.000000in}}{%
\pgfpathmoveto{\pgfqpoint{-0.000000in}{0.000000in}}%
\pgfpathlineto{\pgfqpoint{-0.048611in}{0.000000in}}%
\pgfusepath{stroke,fill}%
}%
\begin{pgfscope}%
\pgfsys@transformshift{0.800000in}{3.384017in}%
\pgfsys@useobject{currentmarker}{}%
\end{pgfscope}%
\end{pgfscope}%
\begin{pgfscope}%
\definecolor{textcolor}{rgb}{0.000000,0.000000,0.000000}%
\pgfsetstrokecolor{textcolor}%
\pgfsetfillcolor{textcolor}%
\pgftext[x=0.481898in, y=3.331255in, left, base]{\color{textcolor}\sffamily\fontsize{10.000000}{12.000000}\selectfont 0.8}%
\end{pgfscope}%
\begin{pgfscope}%
\pgfsetbuttcap%
\pgfsetroundjoin%
\definecolor{currentfill}{rgb}{0.000000,0.000000,0.000000}%
\pgfsetfillcolor{currentfill}%
\pgfsetlinewidth{0.803000pt}%
\definecolor{currentstroke}{rgb}{0.000000,0.000000,0.000000}%
\pgfsetstrokecolor{currentstroke}%
\pgfsetdash{}{0pt}%
\pgfsys@defobject{currentmarker}{\pgfqpoint{-0.048611in}{0.000000in}}{\pgfqpoint{-0.000000in}{0.000000in}}{%
\pgfpathmoveto{\pgfqpoint{-0.000000in}{0.000000in}}%
\pgfpathlineto{\pgfqpoint{-0.048611in}{0.000000in}}%
\pgfusepath{stroke,fill}%
}%
\begin{pgfscope}%
\pgfsys@transformshift{0.800000in}{4.056021in}%
\pgfsys@useobject{currentmarker}{}%
\end{pgfscope}%
\end{pgfscope}%
\begin{pgfscope}%
\definecolor{textcolor}{rgb}{0.000000,0.000000,0.000000}%
\pgfsetstrokecolor{textcolor}%
\pgfsetfillcolor{textcolor}%
\pgftext[x=0.481898in, y=4.003260in, left, base]{\color{textcolor}\sffamily\fontsize{10.000000}{12.000000}\selectfont 1.0}%
\end{pgfscope}%
\begin{pgfscope}%
\definecolor{textcolor}{rgb}{0.000000,0.000000,0.000000}%
\pgfsetstrokecolor{textcolor}%
\pgfsetfillcolor{textcolor}%
\pgftext[x=0.426343in,y=2.376000in,,bottom,rotate=90.000000]{\color{textcolor}\sffamily\fontsize{10.000000}{12.000000}\selectfont $g_n(x)$}%
\end{pgfscope}%
\begin{pgfscope}%
\pgfpathrectangle{\pgfqpoint{0.800000in}{0.528000in}}{\pgfqpoint{4.960000in}{3.696000in}}%
\pgfusepath{clip}%
\pgfsetrectcap%
\pgfsetroundjoin%
\pgfsetlinewidth{1.505625pt}%
\definecolor{currentstroke}{rgb}{0.121569,0.466667,0.705882}%
\pgfsetstrokecolor{currentstroke}%
\pgfsetdash{}{0pt}%
\pgfpathmoveto{\pgfqpoint{1.025455in}{2.376011in}}%
\pgfpathlineto{\pgfqpoint{1.115862in}{2.444729in}}%
\pgfpathlineto{\pgfqpoint{1.206270in}{2.516132in}}%
\pgfpathlineto{\pgfqpoint{1.296678in}{2.590175in}}%
\pgfpathlineto{\pgfqpoint{1.398387in}{2.676520in}}%
\pgfpathlineto{\pgfqpoint{1.500096in}{2.765911in}}%
\pgfpathlineto{\pgfqpoint{1.613105in}{2.868470in}}%
\pgfpathlineto{\pgfqpoint{1.737416in}{2.984618in}}%
\pgfpathlineto{\pgfqpoint{1.884329in}{3.125130in}}%
\pgfpathlineto{\pgfqpoint{2.223358in}{3.450899in}}%
\pgfpathlineto{\pgfqpoint{2.325067in}{3.544919in}}%
\pgfpathlineto{\pgfqpoint{2.415475in}{3.625296in}}%
\pgfpathlineto{\pgfqpoint{2.494582in}{3.692374in}}%
\pgfpathlineto{\pgfqpoint{2.562388in}{3.746925in}}%
\pgfpathlineto{\pgfqpoint{2.630194in}{3.798309in}}%
\pgfpathlineto{\pgfqpoint{2.686699in}{3.838403in}}%
\pgfpathlineto{\pgfqpoint{2.743203in}{3.875762in}}%
\pgfpathlineto{\pgfqpoint{2.799708in}{3.910154in}}%
\pgfpathlineto{\pgfqpoint{2.844912in}{3.935379in}}%
\pgfpathlineto{\pgfqpoint{2.890116in}{3.958455in}}%
\pgfpathlineto{\pgfqpoint{2.935320in}{3.979281in}}%
\pgfpathlineto{\pgfqpoint{2.980524in}{3.997764in}}%
\pgfpathlineto{\pgfqpoint{3.025728in}{4.013819in}}%
\pgfpathlineto{\pgfqpoint{3.070932in}{4.027374in}}%
\pgfpathlineto{\pgfqpoint{3.116136in}{4.038365in}}%
\pgfpathlineto{\pgfqpoint{3.161340in}{4.046739in}}%
\pgfpathlineto{\pgfqpoint{3.206544in}{4.052458in}}%
\pgfpathlineto{\pgfqpoint{3.251748in}{4.055494in}}%
\pgfpathlineto{\pgfqpoint{3.296951in}{4.055831in}}%
\pgfpathlineto{\pgfqpoint{3.342155in}{4.053469in}}%
\pgfpathlineto{\pgfqpoint{3.387359in}{4.048419in}}%
\pgfpathlineto{\pgfqpoint{3.432563in}{4.040705in}}%
\pgfpathlineto{\pgfqpoint{3.477767in}{4.030364in}}%
\pgfpathlineto{\pgfqpoint{3.522971in}{4.017445in}}%
\pgfpathlineto{\pgfqpoint{3.568175in}{4.002008in}}%
\pgfpathlineto{\pgfqpoint{3.613379in}{3.984125in}}%
\pgfpathlineto{\pgfqpoint{3.658583in}{3.963877in}}%
\pgfpathlineto{\pgfqpoint{3.703787in}{3.941354in}}%
\pgfpathlineto{\pgfqpoint{3.748991in}{3.916656in}}%
\pgfpathlineto{\pgfqpoint{3.805496in}{3.882885in}}%
\pgfpathlineto{\pgfqpoint{3.862000in}{3.846101in}}%
\pgfpathlineto{\pgfqpoint{3.918505in}{3.806535in}}%
\pgfpathlineto{\pgfqpoint{3.975010in}{3.764426in}}%
\pgfpathlineto{\pgfqpoint{4.042816in}{3.710884in}}%
\pgfpathlineto{\pgfqpoint{4.110622in}{3.654458in}}%
\pgfpathlineto{\pgfqpoint{4.189729in}{3.585547in}}%
\pgfpathlineto{\pgfqpoint{4.280137in}{3.503530in}}%
\pgfpathlineto{\pgfqpoint{4.381846in}{3.408212in}}%
\pgfpathlineto{\pgfqpoint{4.528758in}{3.267205in}}%
\pgfpathlineto{\pgfqpoint{4.811283in}{2.995321in}}%
\pgfpathlineto{\pgfqpoint{4.935594in}{2.878896in}}%
\pgfpathlineto{\pgfqpoint{5.048603in}{2.776020in}}%
\pgfpathlineto{\pgfqpoint{5.150312in}{2.686306in}}%
\pgfpathlineto{\pgfqpoint{5.252021in}{2.599612in}}%
\pgfpathlineto{\pgfqpoint{5.342429in}{2.525245in}}%
\pgfpathlineto{\pgfqpoint{5.432837in}{2.453508in}}%
\pgfpathlineto{\pgfqpoint{5.523244in}{2.384453in}}%
\pgfpathlineto{\pgfqpoint{5.534545in}{2.376011in}}%
\pgfpathlineto{\pgfqpoint{5.534545in}{2.376011in}}%
\pgfusepath{stroke}%
\end{pgfscope}%
\begin{pgfscope}%
\pgfpathrectangle{\pgfqpoint{0.800000in}{0.528000in}}{\pgfqpoint{4.960000in}{3.696000in}}%
\pgfusepath{clip}%
\pgfsetrectcap%
\pgfsetroundjoin%
\pgfsetlinewidth{1.505625pt}%
\definecolor{currentstroke}{rgb}{1.000000,0.498039,0.054902}%
\pgfsetstrokecolor{currentstroke}%
\pgfsetdash{}{0pt}%
\pgfpathmoveto{\pgfqpoint{1.025455in}{1.816007in}}%
\pgfpathlineto{\pgfqpoint{1.093260in}{1.862063in}}%
\pgfpathlineto{\pgfqpoint{1.161066in}{1.910480in}}%
\pgfpathlineto{\pgfqpoint{1.228872in}{1.961364in}}%
\pgfpathlineto{\pgfqpoint{1.296678in}{2.014823in}}%
\pgfpathlineto{\pgfqpoint{1.364484in}{2.070960in}}%
\pgfpathlineto{\pgfqpoint{1.432290in}{2.129872in}}%
\pgfpathlineto{\pgfqpoint{1.500096in}{2.191644in}}%
\pgfpathlineto{\pgfqpoint{1.567902in}{2.256353in}}%
\pgfpathlineto{\pgfqpoint{1.635707in}{2.324055in}}%
\pgfpathlineto{\pgfqpoint{1.703513in}{2.394786in}}%
\pgfpathlineto{\pgfqpoint{1.771319in}{2.468551in}}%
\pgfpathlineto{\pgfqpoint{1.839125in}{2.545324in}}%
\pgfpathlineto{\pgfqpoint{1.918232in}{2.638595in}}%
\pgfpathlineto{\pgfqpoint{1.997339in}{2.735658in}}%
\pgfpathlineto{\pgfqpoint{2.076446in}{2.836197in}}%
\pgfpathlineto{\pgfqpoint{2.166853in}{2.954766in}}%
\pgfpathlineto{\pgfqpoint{2.279863in}{3.107075in}}%
\pgfpathlineto{\pgfqpoint{2.551087in}{3.475044in}}%
\pgfpathlineto{\pgfqpoint{2.630194in}{3.577313in}}%
\pgfpathlineto{\pgfqpoint{2.686699in}{3.647259in}}%
\pgfpathlineto{\pgfqpoint{2.743203in}{3.713860in}}%
\pgfpathlineto{\pgfqpoint{2.788407in}{3.764268in}}%
\pgfpathlineto{\pgfqpoint{2.833611in}{3.811735in}}%
\pgfpathlineto{\pgfqpoint{2.878815in}{3.855908in}}%
\pgfpathlineto{\pgfqpoint{2.924019in}{3.896442in}}%
\pgfpathlineto{\pgfqpoint{2.957922in}{3.924256in}}%
\pgfpathlineto{\pgfqpoint{2.991825in}{3.949704in}}%
\pgfpathlineto{\pgfqpoint{3.025728in}{3.972664in}}%
\pgfpathlineto{\pgfqpoint{3.059631in}{3.993022in}}%
\pgfpathlineto{\pgfqpoint{3.093534in}{4.010674in}}%
\pgfpathlineto{\pgfqpoint{3.127437in}{4.025529in}}%
\pgfpathlineto{\pgfqpoint{3.161340in}{4.037509in}}%
\pgfpathlineto{\pgfqpoint{3.195243in}{4.046550in}}%
\pgfpathlineto{\pgfqpoint{3.229146in}{4.052605in}}%
\pgfpathlineto{\pgfqpoint{3.263049in}{4.055641in}}%
\pgfpathlineto{\pgfqpoint{3.296951in}{4.055641in}}%
\pgfpathlineto{\pgfqpoint{3.330854in}{4.052605in}}%
\pgfpathlineto{\pgfqpoint{3.364757in}{4.046550in}}%
\pgfpathlineto{\pgfqpoint{3.398660in}{4.037509in}}%
\pgfpathlineto{\pgfqpoint{3.432563in}{4.025529in}}%
\pgfpathlineto{\pgfqpoint{3.466466in}{4.010674in}}%
\pgfpathlineto{\pgfqpoint{3.500369in}{3.993022in}}%
\pgfpathlineto{\pgfqpoint{3.534272in}{3.972664in}}%
\pgfpathlineto{\pgfqpoint{3.568175in}{3.949704in}}%
\pgfpathlineto{\pgfqpoint{3.602078in}{3.924256in}}%
\pgfpathlineto{\pgfqpoint{3.635981in}{3.896442in}}%
\pgfpathlineto{\pgfqpoint{3.681185in}{3.855908in}}%
\pgfpathlineto{\pgfqpoint{3.726389in}{3.811735in}}%
\pgfpathlineto{\pgfqpoint{3.771593in}{3.764268in}}%
\pgfpathlineto{\pgfqpoint{3.816797in}{3.713860in}}%
\pgfpathlineto{\pgfqpoint{3.873301in}{3.647259in}}%
\pgfpathlineto{\pgfqpoint{3.929806in}{3.577313in}}%
\pgfpathlineto{\pgfqpoint{3.997612in}{3.489907in}}%
\pgfpathlineto{\pgfqpoint{4.088020in}{3.369270in}}%
\pgfpathlineto{\pgfqpoint{4.257535in}{3.137893in}}%
\pgfpathlineto{\pgfqpoint{4.381846in}{2.969821in}}%
\pgfpathlineto{\pgfqpoint{4.472253in}{2.850819in}}%
\pgfpathlineto{\pgfqpoint{4.562661in}{2.735658in}}%
\pgfpathlineto{\pgfqpoint{4.641768in}{2.638595in}}%
\pgfpathlineto{\pgfqpoint{4.720875in}{2.545324in}}%
\pgfpathlineto{\pgfqpoint{4.788681in}{2.468551in}}%
\pgfpathlineto{\pgfqpoint{4.856487in}{2.394786in}}%
\pgfpathlineto{\pgfqpoint{4.924293in}{2.324055in}}%
\pgfpathlineto{\pgfqpoint{4.992098in}{2.256353in}}%
\pgfpathlineto{\pgfqpoint{5.059904in}{2.191644in}}%
\pgfpathlineto{\pgfqpoint{5.127710in}{2.129872in}}%
\pgfpathlineto{\pgfqpoint{5.195516in}{2.070960in}}%
\pgfpathlineto{\pgfqpoint{5.263322in}{2.014823in}}%
\pgfpathlineto{\pgfqpoint{5.331128in}{1.961364in}}%
\pgfpathlineto{\pgfqpoint{5.398934in}{1.910480in}}%
\pgfpathlineto{\pgfqpoint{5.466740in}{1.862063in}}%
\pgfpathlineto{\pgfqpoint{5.534545in}{1.816007in}}%
\pgfpathlineto{\pgfqpoint{5.534545in}{1.816007in}}%
\pgfusepath{stroke}%
\end{pgfscope}%
\begin{pgfscope}%
\pgfpathrectangle{\pgfqpoint{0.800000in}{0.528000in}}{\pgfqpoint{4.960000in}{3.696000in}}%
\pgfusepath{clip}%
\pgfsetrectcap%
\pgfsetroundjoin%
\pgfsetlinewidth{1.505625pt}%
\definecolor{currentstroke}{rgb}{0.172549,0.627451,0.172549}%
\pgfsetstrokecolor{currentstroke}%
\pgfsetdash{}{0pt}%
\pgfpathmoveto{\pgfqpoint{1.025455in}{1.256004in}}%
\pgfpathlineto{\pgfqpoint{1.104561in}{1.290139in}}%
\pgfpathlineto{\pgfqpoint{1.172367in}{1.321749in}}%
\pgfpathlineto{\pgfqpoint{1.240173in}{1.355737in}}%
\pgfpathlineto{\pgfqpoint{1.307979in}{1.392322in}}%
\pgfpathlineto{\pgfqpoint{1.375785in}{1.431743in}}%
\pgfpathlineto{\pgfqpoint{1.432290in}{1.466947in}}%
\pgfpathlineto{\pgfqpoint{1.488795in}{1.504467in}}%
\pgfpathlineto{\pgfqpoint{1.545300in}{1.544477in}}%
\pgfpathlineto{\pgfqpoint{1.601805in}{1.587166in}}%
\pgfpathlineto{\pgfqpoint{1.658309in}{1.632735in}}%
\pgfpathlineto{\pgfqpoint{1.714814in}{1.681397in}}%
\pgfpathlineto{\pgfqpoint{1.771319in}{1.733376in}}%
\pgfpathlineto{\pgfqpoint{1.827824in}{1.788907in}}%
\pgfpathlineto{\pgfqpoint{1.884329in}{1.848231in}}%
\pgfpathlineto{\pgfqpoint{1.929533in}{1.898589in}}%
\pgfpathlineto{\pgfqpoint{1.974737in}{1.951660in}}%
\pgfpathlineto{\pgfqpoint{2.019941in}{2.007570in}}%
\pgfpathlineto{\pgfqpoint{2.065145in}{2.066441in}}%
\pgfpathlineto{\pgfqpoint{2.110349in}{2.128386in}}%
\pgfpathlineto{\pgfqpoint{2.155553in}{2.193508in}}%
\pgfpathlineto{\pgfqpoint{2.200756in}{2.261894in}}%
\pgfpathlineto{\pgfqpoint{2.245960in}{2.333610in}}%
\pgfpathlineto{\pgfqpoint{2.302465in}{2.427992in}}%
\pgfpathlineto{\pgfqpoint{2.358970in}{2.527625in}}%
\pgfpathlineto{\pgfqpoint{2.415475in}{2.632385in}}%
\pgfpathlineto{\pgfqpoint{2.471980in}{2.742001in}}%
\pgfpathlineto{\pgfqpoint{2.528485in}{2.856014in}}%
\pgfpathlineto{\pgfqpoint{2.596291in}{2.997656in}}%
\pgfpathlineto{\pgfqpoint{2.698000in}{3.216277in}}%
\pgfpathlineto{\pgfqpoint{2.811009in}{3.458354in}}%
\pgfpathlineto{\pgfqpoint{2.867514in}{3.574290in}}%
\pgfpathlineto{\pgfqpoint{2.912718in}{3.662398in}}%
\pgfpathlineto{\pgfqpoint{2.957922in}{3.744908in}}%
\pgfpathlineto{\pgfqpoint{2.991825in}{3.802272in}}%
\pgfpathlineto{\pgfqpoint{3.025728in}{3.855106in}}%
\pgfpathlineto{\pgfqpoint{3.059631in}{3.902831in}}%
\pgfpathlineto{\pgfqpoint{3.082233in}{3.931539in}}%
\pgfpathlineto{\pgfqpoint{3.104835in}{3.957580in}}%
\pgfpathlineto{\pgfqpoint{3.127437in}{3.980813in}}%
\pgfpathlineto{\pgfqpoint{3.150039in}{4.001109in}}%
\pgfpathlineto{\pgfqpoint{3.172641in}{4.018353in}}%
\pgfpathlineto{\pgfqpoint{3.195243in}{4.032444in}}%
\pgfpathlineto{\pgfqpoint{3.217845in}{4.043301in}}%
\pgfpathlineto{\pgfqpoint{3.240447in}{4.050858in}}%
\pgfpathlineto{\pgfqpoint{3.263049in}{4.055072in}}%
\pgfpathlineto{\pgfqpoint{3.285650in}{4.055916in}}%
\pgfpathlineto{\pgfqpoint{3.308252in}{4.053385in}}%
\pgfpathlineto{\pgfqpoint{3.330854in}{4.047495in}}%
\pgfpathlineto{\pgfqpoint{3.353456in}{4.038281in}}%
\pgfpathlineto{\pgfqpoint{3.376058in}{4.025798in}}%
\pgfpathlineto{\pgfqpoint{3.398660in}{4.010119in}}%
\pgfpathlineto{\pgfqpoint{3.421262in}{3.991336in}}%
\pgfpathlineto{\pgfqpoint{3.443864in}{3.969556in}}%
\pgfpathlineto{\pgfqpoint{3.466466in}{3.944902in}}%
\pgfpathlineto{\pgfqpoint{3.489068in}{3.917509in}}%
\pgfpathlineto{\pgfqpoint{3.511670in}{3.887525in}}%
\pgfpathlineto{\pgfqpoint{3.545573in}{3.838034in}}%
\pgfpathlineto{\pgfqpoint{3.579476in}{3.783625in}}%
\pgfpathlineto{\pgfqpoint{3.613379in}{3.724882in}}%
\pgfpathlineto{\pgfqpoint{3.658583in}{3.640842in}}%
\pgfpathlineto{\pgfqpoint{3.703787in}{3.551550in}}%
\pgfpathlineto{\pgfqpoint{3.760292in}{3.434595in}}%
\pgfpathlineto{\pgfqpoint{3.839398in}{3.265180in}}%
\pgfpathlineto{\pgfqpoint{3.986311in}{2.949941in}}%
\pgfpathlineto{\pgfqpoint{4.054117in}{2.809919in}}%
\pgfpathlineto{\pgfqpoint{4.110622in}{2.697597in}}%
\pgfpathlineto{\pgfqpoint{4.167127in}{2.589880in}}%
\pgfpathlineto{\pgfqpoint{4.223632in}{2.487146in}}%
\pgfpathlineto{\pgfqpoint{4.280137in}{2.389606in}}%
\pgfpathlineto{\pgfqpoint{4.336642in}{2.297333in}}%
\pgfpathlineto{\pgfqpoint{4.381846in}{2.227288in}}%
\pgfpathlineto{\pgfqpoint{4.427049in}{2.160544in}}%
\pgfpathlineto{\pgfqpoint{4.472253in}{2.097022in}}%
\pgfpathlineto{\pgfqpoint{4.517457in}{2.036628in}}%
\pgfpathlineto{\pgfqpoint{4.562661in}{1.979253in}}%
\pgfpathlineto{\pgfqpoint{4.607865in}{1.924777in}}%
\pgfpathlineto{\pgfqpoint{4.664370in}{1.860573in}}%
\pgfpathlineto{\pgfqpoint{4.720875in}{1.800460in}}%
\pgfpathlineto{\pgfqpoint{4.777380in}{1.744190in}}%
\pgfpathlineto{\pgfqpoint{4.833885in}{1.691520in}}%
\pgfpathlineto{\pgfqpoint{4.890390in}{1.642213in}}%
\pgfpathlineto{\pgfqpoint{4.946895in}{1.596043in}}%
\pgfpathlineto{\pgfqpoint{5.003399in}{1.552794in}}%
\pgfpathlineto{\pgfqpoint{5.059904in}{1.512264in}}%
\pgfpathlineto{\pgfqpoint{5.116409in}{1.474261in}}%
\pgfpathlineto{\pgfqpoint{5.172914in}{1.438606in}}%
\pgfpathlineto{\pgfqpoint{5.240720in}{1.398688in}}%
\pgfpathlineto{\pgfqpoint{5.308526in}{1.361648in}}%
\pgfpathlineto{\pgfqpoint{5.376332in}{1.327243in}}%
\pgfpathlineto{\pgfqpoint{5.444138in}{1.295251in}}%
\pgfpathlineto{\pgfqpoint{5.511943in}{1.265471in}}%
\pgfpathlineto{\pgfqpoint{5.534545in}{1.256004in}}%
\pgfpathlineto{\pgfqpoint{5.534545in}{1.256004in}}%
\pgfusepath{stroke}%
\end{pgfscope}%
\begin{pgfscope}%
\pgfpathrectangle{\pgfqpoint{0.800000in}{0.528000in}}{\pgfqpoint{4.960000in}{3.696000in}}%
\pgfusepath{clip}%
\pgfsetrectcap%
\pgfsetroundjoin%
\pgfsetlinewidth{1.505625pt}%
\definecolor{currentstroke}{rgb}{0.839216,0.152941,0.156863}%
\pgfsetstrokecolor{currentstroke}%
\pgfsetdash{}{0pt}%
\pgfpathmoveto{\pgfqpoint{1.025455in}{1.001456in}}%
\pgfpathlineto{\pgfqpoint{1.115862in}{1.024960in}}%
\pgfpathlineto{\pgfqpoint{1.194969in}{1.047733in}}%
\pgfpathlineto{\pgfqpoint{1.274076in}{1.072849in}}%
\pgfpathlineto{\pgfqpoint{1.341882in}{1.096481in}}%
\pgfpathlineto{\pgfqpoint{1.409688in}{1.122294in}}%
\pgfpathlineto{\pgfqpoint{1.477494in}{1.150549in}}%
\pgfpathlineto{\pgfqpoint{1.545300in}{1.181543in}}%
\pgfpathlineto{\pgfqpoint{1.601805in}{1.209707in}}%
\pgfpathlineto{\pgfqpoint{1.658309in}{1.240230in}}%
\pgfpathlineto{\pgfqpoint{1.714814in}{1.273360in}}%
\pgfpathlineto{\pgfqpoint{1.771319in}{1.309375in}}%
\pgfpathlineto{\pgfqpoint{1.827824in}{1.348586in}}%
\pgfpathlineto{\pgfqpoint{1.873028in}{1.382488in}}%
\pgfpathlineto{\pgfqpoint{1.918232in}{1.418852in}}%
\pgfpathlineto{\pgfqpoint{1.963436in}{1.457894in}}%
\pgfpathlineto{\pgfqpoint{2.008640in}{1.499844in}}%
\pgfpathlineto{\pgfqpoint{2.053844in}{1.544955in}}%
\pgfpathlineto{\pgfqpoint{2.099048in}{1.593498in}}%
\pgfpathlineto{\pgfqpoint{2.144252in}{1.645766in}}%
\pgfpathlineto{\pgfqpoint{2.189455in}{1.702071in}}%
\pgfpathlineto{\pgfqpoint{2.234659in}{1.762742in}}%
\pgfpathlineto{\pgfqpoint{2.279863in}{1.828125in}}%
\pgfpathlineto{\pgfqpoint{2.325067in}{1.898575in}}%
\pgfpathlineto{\pgfqpoint{2.370271in}{1.974453in}}%
\pgfpathlineto{\pgfqpoint{2.415475in}{2.056109in}}%
\pgfpathlineto{\pgfqpoint{2.460679in}{2.143875in}}%
\pgfpathlineto{\pgfqpoint{2.505883in}{2.238037in}}%
\pgfpathlineto{\pgfqpoint{2.551087in}{2.338815in}}%
\pgfpathlineto{\pgfqpoint{2.596291in}{2.446325in}}%
\pgfpathlineto{\pgfqpoint{2.641495in}{2.560536in}}%
\pgfpathlineto{\pgfqpoint{2.686699in}{2.681217in}}%
\pgfpathlineto{\pgfqpoint{2.731902in}{2.807874in}}%
\pgfpathlineto{\pgfqpoint{2.788407in}{2.973306in}}%
\pgfpathlineto{\pgfqpoint{2.867514in}{3.213371in}}%
\pgfpathlineto{\pgfqpoint{2.957922in}{3.486526in}}%
\pgfpathlineto{\pgfqpoint{3.003126in}{3.615687in}}%
\pgfpathlineto{\pgfqpoint{3.037029in}{3.706387in}}%
\pgfpathlineto{\pgfqpoint{3.070932in}{3.789965in}}%
\pgfpathlineto{\pgfqpoint{3.093534in}{3.840897in}}%
\pgfpathlineto{\pgfqpoint{3.116136in}{3.887430in}}%
\pgfpathlineto{\pgfqpoint{3.138738in}{3.929095in}}%
\pgfpathlineto{\pgfqpoint{3.161340in}{3.965454in}}%
\pgfpathlineto{\pgfqpoint{3.183942in}{3.996114in}}%
\pgfpathlineto{\pgfqpoint{3.206544in}{4.020727in}}%
\pgfpathlineto{\pgfqpoint{3.217845in}{4.030676in}}%
\pgfpathlineto{\pgfqpoint{3.229146in}{4.039012in}}%
\pgfpathlineto{\pgfqpoint{3.240447in}{4.045711in}}%
\pgfpathlineto{\pgfqpoint{3.251748in}{4.050753in}}%
\pgfpathlineto{\pgfqpoint{3.263049in}{4.054123in}}%
\pgfpathlineto{\pgfqpoint{3.274350in}{4.055810in}}%
\pgfpathlineto{\pgfqpoint{3.285650in}{4.055810in}}%
\pgfpathlineto{\pgfqpoint{3.296951in}{4.054123in}}%
\pgfpathlineto{\pgfqpoint{3.308252in}{4.050753in}}%
\pgfpathlineto{\pgfqpoint{3.319553in}{4.045711in}}%
\pgfpathlineto{\pgfqpoint{3.330854in}{4.039012in}}%
\pgfpathlineto{\pgfqpoint{3.342155in}{4.030676in}}%
\pgfpathlineto{\pgfqpoint{3.353456in}{4.020727in}}%
\pgfpathlineto{\pgfqpoint{3.376058in}{3.996114in}}%
\pgfpathlineto{\pgfqpoint{3.398660in}{3.965454in}}%
\pgfpathlineto{\pgfqpoint{3.421262in}{3.929095in}}%
\pgfpathlineto{\pgfqpoint{3.443864in}{3.887430in}}%
\pgfpathlineto{\pgfqpoint{3.466466in}{3.840897in}}%
\pgfpathlineto{\pgfqpoint{3.489068in}{3.789965in}}%
\pgfpathlineto{\pgfqpoint{3.522971in}{3.706387in}}%
\pgfpathlineto{\pgfqpoint{3.556874in}{3.615687in}}%
\pgfpathlineto{\pgfqpoint{3.602078in}{3.486526in}}%
\pgfpathlineto{\pgfqpoint{3.658583in}{3.316981in}}%
\pgfpathlineto{\pgfqpoint{3.794195in}{2.906307in}}%
\pgfpathlineto{\pgfqpoint{3.850699in}{2.743841in}}%
\pgfpathlineto{\pgfqpoint{3.895903in}{2.620092in}}%
\pgfpathlineto{\pgfqpoint{3.941107in}{2.502602in}}%
\pgfpathlineto{\pgfqpoint{3.986311in}{2.391727in}}%
\pgfpathlineto{\pgfqpoint{4.031515in}{2.287589in}}%
\pgfpathlineto{\pgfqpoint{4.076719in}{2.190140in}}%
\pgfpathlineto{\pgfqpoint{4.121923in}{2.099209in}}%
\pgfpathlineto{\pgfqpoint{4.167127in}{2.014537in}}%
\pgfpathlineto{\pgfqpoint{4.212331in}{1.935813in}}%
\pgfpathlineto{\pgfqpoint{4.257535in}{1.862694in}}%
\pgfpathlineto{\pgfqpoint{4.302739in}{1.794822in}}%
\pgfpathlineto{\pgfqpoint{4.347943in}{1.731839in}}%
\pgfpathlineto{\pgfqpoint{4.393147in}{1.673394in}}%
\pgfpathlineto{\pgfqpoint{4.438350in}{1.619147in}}%
\pgfpathlineto{\pgfqpoint{4.483554in}{1.568780in}}%
\pgfpathlineto{\pgfqpoint{4.528758in}{1.521988in}}%
\pgfpathlineto{\pgfqpoint{4.573962in}{1.478490in}}%
\pgfpathlineto{\pgfqpoint{4.619166in}{1.438024in}}%
\pgfpathlineto{\pgfqpoint{4.664370in}{1.400349in}}%
\pgfpathlineto{\pgfqpoint{4.709574in}{1.365242in}}%
\pgfpathlineto{\pgfqpoint{4.754778in}{1.332497in}}%
\pgfpathlineto{\pgfqpoint{4.811283in}{1.294604in}}%
\pgfpathlineto{\pgfqpoint{4.867788in}{1.259778in}}%
\pgfpathlineto{\pgfqpoint{4.924293in}{1.227723in}}%
\pgfpathlineto{\pgfqpoint{4.980797in}{1.198172in}}%
\pgfpathlineto{\pgfqpoint{5.037302in}{1.170888in}}%
\pgfpathlineto{\pgfqpoint{5.105108in}{1.140842in}}%
\pgfpathlineto{\pgfqpoint{5.172914in}{1.113432in}}%
\pgfpathlineto{\pgfqpoint{5.240720in}{1.088373in}}%
\pgfpathlineto{\pgfqpoint{5.319827in}{1.061779in}}%
\pgfpathlineto{\pgfqpoint{5.398934in}{1.037703in}}%
\pgfpathlineto{\pgfqpoint{5.478041in}{1.015849in}}%
\pgfpathlineto{\pgfqpoint{5.534545in}{1.001456in}}%
\pgfpathlineto{\pgfqpoint{5.534545in}{1.001456in}}%
\pgfusepath{stroke}%
\end{pgfscope}%
\begin{pgfscope}%
\pgfpathrectangle{\pgfqpoint{0.800000in}{0.528000in}}{\pgfqpoint{4.960000in}{3.696000in}}%
\pgfusepath{clip}%
\pgfsetrectcap%
\pgfsetroundjoin%
\pgfsetlinewidth{1.505625pt}%
\definecolor{currentstroke}{rgb}{0.580392,0.403922,0.741176}%
\pgfsetstrokecolor{currentstroke}%
\pgfsetdash{}{0pt}%
\pgfpathmoveto{\pgfqpoint{1.025455in}{0.856001in}}%
\pgfpathlineto{\pgfqpoint{1.138464in}{0.872423in}}%
\pgfpathlineto{\pgfqpoint{1.240173in}{0.889417in}}%
\pgfpathlineto{\pgfqpoint{1.330581in}{0.906623in}}%
\pgfpathlineto{\pgfqpoint{1.420989in}{0.926169in}}%
\pgfpathlineto{\pgfqpoint{1.500096in}{0.945530in}}%
\pgfpathlineto{\pgfqpoint{1.579203in}{0.967364in}}%
\pgfpathlineto{\pgfqpoint{1.647008in}{0.988366in}}%
\pgfpathlineto{\pgfqpoint{1.714814in}{1.011813in}}%
\pgfpathlineto{\pgfqpoint{1.771319in}{1.033492in}}%
\pgfpathlineto{\pgfqpoint{1.827824in}{1.057387in}}%
\pgfpathlineto{\pgfqpoint{1.884329in}{1.083796in}}%
\pgfpathlineto{\pgfqpoint{1.940834in}{1.113064in}}%
\pgfpathlineto{\pgfqpoint{1.986038in}{1.138806in}}%
\pgfpathlineto{\pgfqpoint{2.031242in}{1.166870in}}%
\pgfpathlineto{\pgfqpoint{2.076446in}{1.197527in}}%
\pgfpathlineto{\pgfqpoint{2.121650in}{1.231079in}}%
\pgfpathlineto{\pgfqpoint{2.166853in}{1.267873in}}%
\pgfpathlineto{\pgfqpoint{2.212057in}{1.308301in}}%
\pgfpathlineto{\pgfqpoint{2.245960in}{1.341273in}}%
\pgfpathlineto{\pgfqpoint{2.279863in}{1.376748in}}%
\pgfpathlineto{\pgfqpoint{2.313766in}{1.414956in}}%
\pgfpathlineto{\pgfqpoint{2.347669in}{1.456151in}}%
\pgfpathlineto{\pgfqpoint{2.381572in}{1.500606in}}%
\pgfpathlineto{\pgfqpoint{2.415475in}{1.548622in}}%
\pgfpathlineto{\pgfqpoint{2.449378in}{1.600525in}}%
\pgfpathlineto{\pgfqpoint{2.483281in}{1.656665in}}%
\pgfpathlineto{\pgfqpoint{2.517184in}{1.717420in}}%
\pgfpathlineto{\pgfqpoint{2.551087in}{1.783187in}}%
\pgfpathlineto{\pgfqpoint{2.584990in}{1.854382in}}%
\pgfpathlineto{\pgfqpoint{2.618893in}{1.931432in}}%
\pgfpathlineto{\pgfqpoint{2.652796in}{2.014765in}}%
\pgfpathlineto{\pgfqpoint{2.686699in}{2.104789in}}%
\pgfpathlineto{\pgfqpoint{2.720602in}{2.201876in}}%
\pgfpathlineto{\pgfqpoint{2.754504in}{2.306323in}}%
\pgfpathlineto{\pgfqpoint{2.788407in}{2.418316in}}%
\pgfpathlineto{\pgfqpoint{2.822310in}{2.537873in}}%
\pgfpathlineto{\pgfqpoint{2.856213in}{2.664777in}}%
\pgfpathlineto{\pgfqpoint{2.901417in}{2.844432in}}%
\pgfpathlineto{\pgfqpoint{2.946621in}{3.033717in}}%
\pgfpathlineto{\pgfqpoint{3.070932in}{3.562952in}}%
\pgfpathlineto{\pgfqpoint{3.104835in}{3.694070in}}%
\pgfpathlineto{\pgfqpoint{3.127437in}{3.774120in}}%
\pgfpathlineto{\pgfqpoint{3.150039in}{3.846639in}}%
\pgfpathlineto{\pgfqpoint{3.172641in}{3.910250in}}%
\pgfpathlineto{\pgfqpoint{3.195243in}{3.963657in}}%
\pgfpathlineto{\pgfqpoint{3.206544in}{3.986167in}}%
\pgfpathlineto{\pgfqpoint{3.217845in}{4.005710in}}%
\pgfpathlineto{\pgfqpoint{3.229146in}{4.022175in}}%
\pgfpathlineto{\pgfqpoint{3.240447in}{4.035464in}}%
\pgfpathlineto{\pgfqpoint{3.251748in}{4.045501in}}%
\pgfpathlineto{\pgfqpoint{3.263049in}{4.052226in}}%
\pgfpathlineto{\pgfqpoint{3.274350in}{4.055599in}}%
\pgfpathlineto{\pgfqpoint{3.285650in}{4.055599in}}%
\pgfpathlineto{\pgfqpoint{3.296951in}{4.052226in}}%
\pgfpathlineto{\pgfqpoint{3.308252in}{4.045501in}}%
\pgfpathlineto{\pgfqpoint{3.319553in}{4.035464in}}%
\pgfpathlineto{\pgfqpoint{3.330854in}{4.022175in}}%
\pgfpathlineto{\pgfqpoint{3.342155in}{4.005710in}}%
\pgfpathlineto{\pgfqpoint{3.353456in}{3.986167in}}%
\pgfpathlineto{\pgfqpoint{3.364757in}{3.963657in}}%
\pgfpathlineto{\pgfqpoint{3.387359in}{3.910250in}}%
\pgfpathlineto{\pgfqpoint{3.409961in}{3.846639in}}%
\pgfpathlineto{\pgfqpoint{3.432563in}{3.774120in}}%
\pgfpathlineto{\pgfqpoint{3.455165in}{3.694070in}}%
\pgfpathlineto{\pgfqpoint{3.489068in}{3.562952in}}%
\pgfpathlineto{\pgfqpoint{3.534272in}{3.374598in}}%
\pgfpathlineto{\pgfqpoint{3.658583in}{2.844432in}}%
\pgfpathlineto{\pgfqpoint{3.703787in}{2.664777in}}%
\pgfpathlineto{\pgfqpoint{3.737690in}{2.537873in}}%
\pgfpathlineto{\pgfqpoint{3.771593in}{2.418316in}}%
\pgfpathlineto{\pgfqpoint{3.805496in}{2.306323in}}%
\pgfpathlineto{\pgfqpoint{3.839398in}{2.201876in}}%
\pgfpathlineto{\pgfqpoint{3.873301in}{2.104789in}}%
\pgfpathlineto{\pgfqpoint{3.907204in}{2.014765in}}%
\pgfpathlineto{\pgfqpoint{3.941107in}{1.931432in}}%
\pgfpathlineto{\pgfqpoint{3.975010in}{1.854382in}}%
\pgfpathlineto{\pgfqpoint{4.008913in}{1.783187in}}%
\pgfpathlineto{\pgfqpoint{4.042816in}{1.717420in}}%
\pgfpathlineto{\pgfqpoint{4.076719in}{1.656665in}}%
\pgfpathlineto{\pgfqpoint{4.110622in}{1.600525in}}%
\pgfpathlineto{\pgfqpoint{4.144525in}{1.548622in}}%
\pgfpathlineto{\pgfqpoint{4.178428in}{1.500606in}}%
\pgfpathlineto{\pgfqpoint{4.212331in}{1.456151in}}%
\pgfpathlineto{\pgfqpoint{4.246234in}{1.414956in}}%
\pgfpathlineto{\pgfqpoint{4.280137in}{1.376748in}}%
\pgfpathlineto{\pgfqpoint{4.314040in}{1.341273in}}%
\pgfpathlineto{\pgfqpoint{4.347943in}{1.308301in}}%
\pgfpathlineto{\pgfqpoint{4.393147in}{1.267873in}}%
\pgfpathlineto{\pgfqpoint{4.438350in}{1.231079in}}%
\pgfpathlineto{\pgfqpoint{4.483554in}{1.197527in}}%
\pgfpathlineto{\pgfqpoint{4.528758in}{1.166870in}}%
\pgfpathlineto{\pgfqpoint{4.573962in}{1.138806in}}%
\pgfpathlineto{\pgfqpoint{4.619166in}{1.113064in}}%
\pgfpathlineto{\pgfqpoint{4.675671in}{1.083796in}}%
\pgfpathlineto{\pgfqpoint{4.732176in}{1.057387in}}%
\pgfpathlineto{\pgfqpoint{4.788681in}{1.033492in}}%
\pgfpathlineto{\pgfqpoint{4.845186in}{1.011813in}}%
\pgfpathlineto{\pgfqpoint{4.912992in}{0.988366in}}%
\pgfpathlineto{\pgfqpoint{4.980797in}{0.967364in}}%
\pgfpathlineto{\pgfqpoint{5.048603in}{0.948489in}}%
\pgfpathlineto{\pgfqpoint{5.127710in}{0.928798in}}%
\pgfpathlineto{\pgfqpoint{5.206817in}{0.911275in}}%
\pgfpathlineto{\pgfqpoint{5.297225in}{0.893520in}}%
\pgfpathlineto{\pgfqpoint{5.398934in}{0.876004in}}%
\pgfpathlineto{\pgfqpoint{5.500643in}{0.860683in}}%
\pgfpathlineto{\pgfqpoint{5.534545in}{0.856001in}}%
\pgfpathlineto{\pgfqpoint{5.534545in}{0.856001in}}%
\pgfusepath{stroke}%
\end{pgfscope}%
\begin{pgfscope}%
\pgfpathrectangle{\pgfqpoint{0.800000in}{0.528000in}}{\pgfqpoint{4.960000in}{3.696000in}}%
\pgfusepath{clip}%
\pgfsetrectcap%
\pgfsetroundjoin%
\pgfsetlinewidth{1.505625pt}%
\definecolor{currentstroke}{rgb}{0.549020,0.337255,0.294118}%
\pgfsetstrokecolor{currentstroke}%
\pgfsetdash{}{0pt}%
\pgfpathmoveto{\pgfqpoint{1.025455in}{0.777952in}}%
\pgfpathlineto{\pgfqpoint{1.172367in}{0.789446in}}%
\pgfpathlineto{\pgfqpoint{1.307979in}{0.802319in}}%
\pgfpathlineto{\pgfqpoint{1.420989in}{0.815166in}}%
\pgfpathlineto{\pgfqpoint{1.522698in}{0.828799in}}%
\pgfpathlineto{\pgfqpoint{1.613105in}{0.842948in}}%
\pgfpathlineto{\pgfqpoint{1.703513in}{0.859442in}}%
\pgfpathlineto{\pgfqpoint{1.782620in}{0.876216in}}%
\pgfpathlineto{\pgfqpoint{1.850426in}{0.892693in}}%
\pgfpathlineto{\pgfqpoint{1.918232in}{0.911481in}}%
\pgfpathlineto{\pgfqpoint{1.974737in}{0.929218in}}%
\pgfpathlineto{\pgfqpoint{2.031242in}{0.949175in}}%
\pgfpathlineto{\pgfqpoint{2.087747in}{0.971725in}}%
\pgfpathlineto{\pgfqpoint{2.132951in}{0.991934in}}%
\pgfpathlineto{\pgfqpoint{2.178154in}{1.014366in}}%
\pgfpathlineto{\pgfqpoint{2.223358in}{1.039345in}}%
\pgfpathlineto{\pgfqpoint{2.268562in}{1.067255in}}%
\pgfpathlineto{\pgfqpoint{2.302465in}{1.090378in}}%
\pgfpathlineto{\pgfqpoint{2.336368in}{1.115623in}}%
\pgfpathlineto{\pgfqpoint{2.370271in}{1.143241in}}%
\pgfpathlineto{\pgfqpoint{2.404174in}{1.173520in}}%
\pgfpathlineto{\pgfqpoint{2.438077in}{1.206789in}}%
\pgfpathlineto{\pgfqpoint{2.471980in}{1.243421in}}%
\pgfpathlineto{\pgfqpoint{2.505883in}{1.283848in}}%
\pgfpathlineto{\pgfqpoint{2.539786in}{1.328560in}}%
\pgfpathlineto{\pgfqpoint{2.573689in}{1.378119in}}%
\pgfpathlineto{\pgfqpoint{2.607592in}{1.433167in}}%
\pgfpathlineto{\pgfqpoint{2.641495in}{1.494433in}}%
\pgfpathlineto{\pgfqpoint{2.664097in}{1.539134in}}%
\pgfpathlineto{\pgfqpoint{2.686699in}{1.587233in}}%
\pgfpathlineto{\pgfqpoint{2.709301in}{1.639018in}}%
\pgfpathlineto{\pgfqpoint{2.731902in}{1.694801in}}%
\pgfpathlineto{\pgfqpoint{2.754504in}{1.754907in}}%
\pgfpathlineto{\pgfqpoint{2.777106in}{1.819682in}}%
\pgfpathlineto{\pgfqpoint{2.799708in}{1.889480in}}%
\pgfpathlineto{\pgfqpoint{2.822310in}{1.964659in}}%
\pgfpathlineto{\pgfqpoint{2.844912in}{2.045575in}}%
\pgfpathlineto{\pgfqpoint{2.867514in}{2.132559in}}%
\pgfpathlineto{\pgfqpoint{2.890116in}{2.225908in}}%
\pgfpathlineto{\pgfqpoint{2.912718in}{2.325851in}}%
\pgfpathlineto{\pgfqpoint{2.946621in}{2.488377in}}%
\pgfpathlineto{\pgfqpoint{2.980524in}{2.665792in}}%
\pgfpathlineto{\pgfqpoint{3.014427in}{2.856755in}}%
\pgfpathlineto{\pgfqpoint{3.059631in}{3.126996in}}%
\pgfpathlineto{\pgfqpoint{3.127437in}{3.535859in}}%
\pgfpathlineto{\pgfqpoint{3.150039in}{3.661823in}}%
\pgfpathlineto{\pgfqpoint{3.172641in}{3.776601in}}%
\pgfpathlineto{\pgfqpoint{3.195243in}{3.876235in}}%
\pgfpathlineto{\pgfqpoint{3.206544in}{3.919159in}}%
\pgfpathlineto{\pgfqpoint{3.217845in}{3.956884in}}%
\pgfpathlineto{\pgfqpoint{3.229146in}{3.989003in}}%
\pgfpathlineto{\pgfqpoint{3.240447in}{4.015157in}}%
\pgfpathlineto{\pgfqpoint{3.251748in}{4.035047in}}%
\pgfpathlineto{\pgfqpoint{3.263049in}{4.048440in}}%
\pgfpathlineto{\pgfqpoint{3.274350in}{4.055177in}}%
\pgfpathlineto{\pgfqpoint{3.285650in}{4.055177in}}%
\pgfpathlineto{\pgfqpoint{3.296951in}{4.048440in}}%
\pgfpathlineto{\pgfqpoint{3.308252in}{4.035047in}}%
\pgfpathlineto{\pgfqpoint{3.319553in}{4.015157in}}%
\pgfpathlineto{\pgfqpoint{3.330854in}{3.989003in}}%
\pgfpathlineto{\pgfqpoint{3.342155in}{3.956884in}}%
\pgfpathlineto{\pgfqpoint{3.353456in}{3.919159in}}%
\pgfpathlineto{\pgfqpoint{3.364757in}{3.876235in}}%
\pgfpathlineto{\pgfqpoint{3.387359in}{3.776601in}}%
\pgfpathlineto{\pgfqpoint{3.409961in}{3.661823in}}%
\pgfpathlineto{\pgfqpoint{3.432563in}{3.535859in}}%
\pgfpathlineto{\pgfqpoint{3.466466in}{3.334174in}}%
\pgfpathlineto{\pgfqpoint{3.545573in}{2.856755in}}%
\pgfpathlineto{\pgfqpoint{3.579476in}{2.665792in}}%
\pgfpathlineto{\pgfqpoint{3.613379in}{2.488377in}}%
\pgfpathlineto{\pgfqpoint{3.647282in}{2.325851in}}%
\pgfpathlineto{\pgfqpoint{3.681185in}{2.178422in}}%
\pgfpathlineto{\pgfqpoint{3.703787in}{2.088289in}}%
\pgfpathlineto{\pgfqpoint{3.726389in}{2.004378in}}%
\pgfpathlineto{\pgfqpoint{3.748991in}{1.926374in}}%
\pgfpathlineto{\pgfqpoint{3.771593in}{1.853930in}}%
\pgfpathlineto{\pgfqpoint{3.794195in}{1.786689in}}%
\pgfpathlineto{\pgfqpoint{3.816797in}{1.724292in}}%
\pgfpathlineto{\pgfqpoint{3.839398in}{1.666390in}}%
\pgfpathlineto{\pgfqpoint{3.862000in}{1.612646in}}%
\pgfpathlineto{\pgfqpoint{3.895903in}{1.539134in}}%
\pgfpathlineto{\pgfqpoint{3.929806in}{1.473271in}}%
\pgfpathlineto{\pgfqpoint{3.963709in}{1.414165in}}%
\pgfpathlineto{\pgfqpoint{3.997612in}{1.361023in}}%
\pgfpathlineto{\pgfqpoint{4.031515in}{1.313147in}}%
\pgfpathlineto{\pgfqpoint{4.065418in}{1.269922in}}%
\pgfpathlineto{\pgfqpoint{4.099321in}{1.230811in}}%
\pgfpathlineto{\pgfqpoint{4.133224in}{1.195345in}}%
\pgfpathlineto{\pgfqpoint{4.167127in}{1.163112in}}%
\pgfpathlineto{\pgfqpoint{4.201030in}{1.133754in}}%
\pgfpathlineto{\pgfqpoint{4.234933in}{1.106957in}}%
\pgfpathlineto{\pgfqpoint{4.268836in}{1.082446in}}%
\pgfpathlineto{\pgfqpoint{4.314040in}{1.052907in}}%
\pgfpathlineto{\pgfqpoint{4.359244in}{1.026515in}}%
\pgfpathlineto{\pgfqpoint{4.404447in}{1.002853in}}%
\pgfpathlineto{\pgfqpoint{4.449651in}{0.981570in}}%
\pgfpathlineto{\pgfqpoint{4.494855in}{0.962367in}}%
\pgfpathlineto{\pgfqpoint{4.551360in}{0.940904in}}%
\pgfpathlineto{\pgfqpoint{4.607865in}{0.921876in}}%
\pgfpathlineto{\pgfqpoint{4.664370in}{0.904936in}}%
\pgfpathlineto{\pgfqpoint{4.732176in}{0.886963in}}%
\pgfpathlineto{\pgfqpoint{4.799982in}{0.871174in}}%
\pgfpathlineto{\pgfqpoint{4.879089in}{0.855071in}}%
\pgfpathlineto{\pgfqpoint{4.969496in}{0.839209in}}%
\pgfpathlineto{\pgfqpoint{5.059904in}{0.825577in}}%
\pgfpathlineto{\pgfqpoint{5.161613in}{0.812419in}}%
\pgfpathlineto{\pgfqpoint{5.274623in}{0.799998in}}%
\pgfpathlineto{\pgfqpoint{5.398934in}{0.788479in}}%
\pgfpathlineto{\pgfqpoint{5.534545in}{0.777952in}}%
\pgfpathlineto{\pgfqpoint{5.534545in}{0.777952in}}%
\pgfusepath{stroke}%
\end{pgfscope}%
\begin{pgfscope}%
\pgfpathrectangle{\pgfqpoint{0.800000in}{0.528000in}}{\pgfqpoint{4.960000in}{3.696000in}}%
\pgfusepath{clip}%
\pgfsetbuttcap%
\pgfsetroundjoin%
\pgfsetlinewidth{1.505625pt}%
\definecolor{currentstroke}{rgb}{0.000000,0.000000,0.000000}%
\pgfsetstrokecolor{currentstroke}%
\pgfsetdash{{5.550000pt}{2.400000pt}}{0.000000pt}%
\pgfpathmoveto{\pgfqpoint{1.025455in}{0.696000in}}%
\pgfpathlineto{\pgfqpoint{5.534545in}{0.696000in}}%
\pgfpathlineto{\pgfqpoint{5.534545in}{0.696000in}}%
\pgfusepath{stroke}%
\end{pgfscope}%
\begin{pgfscope}%
\pgfsetrectcap%
\pgfsetmiterjoin%
\pgfsetlinewidth{0.803000pt}%
\definecolor{currentstroke}{rgb}{0.000000,0.000000,0.000000}%
\pgfsetstrokecolor{currentstroke}%
\pgfsetdash{}{0pt}%
\pgfpathmoveto{\pgfqpoint{0.800000in}{0.528000in}}%
\pgfpathlineto{\pgfqpoint{0.800000in}{4.224000in}}%
\pgfusepath{stroke}%
\end{pgfscope}%
\begin{pgfscope}%
\pgfsetrectcap%
\pgfsetmiterjoin%
\pgfsetlinewidth{0.803000pt}%
\definecolor{currentstroke}{rgb}{0.000000,0.000000,0.000000}%
\pgfsetstrokecolor{currentstroke}%
\pgfsetdash{}{0pt}%
\pgfpathmoveto{\pgfqpoint{5.760000in}{0.528000in}}%
\pgfpathlineto{\pgfqpoint{5.760000in}{4.224000in}}%
\pgfusepath{stroke}%
\end{pgfscope}%
\begin{pgfscope}%
\pgfsetrectcap%
\pgfsetmiterjoin%
\pgfsetlinewidth{0.803000pt}%
\definecolor{currentstroke}{rgb}{0.000000,0.000000,0.000000}%
\pgfsetstrokecolor{currentstroke}%
\pgfsetdash{}{0pt}%
\pgfpathmoveto{\pgfqpoint{0.800000in}{0.528000in}}%
\pgfpathlineto{\pgfqpoint{5.760000in}{0.528000in}}%
\pgfusepath{stroke}%
\end{pgfscope}%
\begin{pgfscope}%
\pgfsetrectcap%
\pgfsetmiterjoin%
\pgfsetlinewidth{0.803000pt}%
\definecolor{currentstroke}{rgb}{0.000000,0.000000,0.000000}%
\pgfsetstrokecolor{currentstroke}%
\pgfsetdash{}{0pt}%
\pgfpathmoveto{\pgfqpoint{0.800000in}{4.224000in}}%
\pgfpathlineto{\pgfqpoint{5.760000in}{4.224000in}}%
\pgfusepath{stroke}%
\end{pgfscope}%
\begin{pgfscope}%
\definecolor{textcolor}{rgb}{0.000000,0.000000,0.000000}%
\pgfsetstrokecolor{textcolor}%
\pgfsetfillcolor{textcolor}%
\pgftext[x=3.280000in,y=4.307333in,,base]{\color{textcolor}\sffamily\fontsize{12.000000}{14.400000}\selectfont Uniform Convergence of $\displaystyle g_n(x) = 1 / (1 + nx^2)$}%
\end{pgfscope}%
\begin{pgfscope}%
\pgfsetbuttcap%
\pgfsetmiterjoin%
\definecolor{currentfill}{rgb}{1.000000,1.000000,1.000000}%
\pgfsetfillcolor{currentfill}%
\pgfsetfillopacity{0.800000}%
\pgfsetlinewidth{1.003750pt}%
\definecolor{currentstroke}{rgb}{0.800000,0.800000,0.800000}%
\pgfsetstrokecolor{currentstroke}%
\pgfsetstrokeopacity{0.800000}%
\pgfsetdash{}{0pt}%
\pgfpathmoveto{\pgfqpoint{4.262523in}{2.685888in}}%
\pgfpathlineto{\pgfqpoint{5.662778in}{2.685888in}}%
\pgfpathquadraticcurveto{\pgfqpoint{5.690556in}{2.685888in}}{\pgfqpoint{5.690556in}{2.713666in}}%
\pgfpathlineto{\pgfqpoint{5.690556in}{4.126778in}}%
\pgfpathquadraticcurveto{\pgfqpoint{5.690556in}{4.154556in}}{\pgfqpoint{5.662778in}{4.154556in}}%
\pgfpathlineto{\pgfqpoint{4.262523in}{4.154556in}}%
\pgfpathquadraticcurveto{\pgfqpoint{4.234745in}{4.154556in}}{\pgfqpoint{4.234745in}{4.126778in}}%
\pgfpathlineto{\pgfqpoint{4.234745in}{2.713666in}}%
\pgfpathquadraticcurveto{\pgfqpoint{4.234745in}{2.685888in}}{\pgfqpoint{4.262523in}{2.685888in}}%
\pgfpathlineto{\pgfqpoint{4.262523in}{2.685888in}}%
\pgfpathclose%
\pgfusepath{stroke,fill}%
\end{pgfscope}%
\begin{pgfscope}%
\pgfsetrectcap%
\pgfsetroundjoin%
\pgfsetlinewidth{1.505625pt}%
\definecolor{currentstroke}{rgb}{0.121569,0.466667,0.705882}%
\pgfsetstrokecolor{currentstroke}%
\pgfsetdash{}{0pt}%
\pgfpathmoveto{\pgfqpoint{4.290301in}{4.042088in}}%
\pgfpathlineto{\pgfqpoint{4.429189in}{4.042088in}}%
\pgfpathlineto{\pgfqpoint{4.568078in}{4.042088in}}%
\pgfusepath{stroke}%
\end{pgfscope}%
\begin{pgfscope}%
\definecolor{textcolor}{rgb}{0.000000,0.000000,0.000000}%
\pgfsetstrokecolor{textcolor}%
\pgfsetfillcolor{textcolor}%
\pgftext[x=4.679189in,y=3.993477in,left,base]{\color{textcolor}\sffamily\fontsize{10.000000}{12.000000}\selectfont n = 1}%
\end{pgfscope}%
\begin{pgfscope}%
\pgfsetrectcap%
\pgfsetroundjoin%
\pgfsetlinewidth{1.505625pt}%
\definecolor{currentstroke}{rgb}{1.000000,0.498039,0.054902}%
\pgfsetstrokecolor{currentstroke}%
\pgfsetdash{}{0pt}%
\pgfpathmoveto{\pgfqpoint{4.290301in}{3.838231in}}%
\pgfpathlineto{\pgfqpoint{4.429189in}{3.838231in}}%
\pgfpathlineto{\pgfqpoint{4.568078in}{3.838231in}}%
\pgfusepath{stroke}%
\end{pgfscope}%
\begin{pgfscope}%
\definecolor{textcolor}{rgb}{0.000000,0.000000,0.000000}%
\pgfsetstrokecolor{textcolor}%
\pgfsetfillcolor{textcolor}%
\pgftext[x=4.679189in,y=3.789620in,left,base]{\color{textcolor}\sffamily\fontsize{10.000000}{12.000000}\selectfont n = 2}%
\end{pgfscope}%
\begin{pgfscope}%
\pgfsetrectcap%
\pgfsetroundjoin%
\pgfsetlinewidth{1.505625pt}%
\definecolor{currentstroke}{rgb}{0.172549,0.627451,0.172549}%
\pgfsetstrokecolor{currentstroke}%
\pgfsetdash{}{0pt}%
\pgfpathmoveto{\pgfqpoint{4.290301in}{3.634374in}}%
\pgfpathlineto{\pgfqpoint{4.429189in}{3.634374in}}%
\pgfpathlineto{\pgfqpoint{4.568078in}{3.634374in}}%
\pgfusepath{stroke}%
\end{pgfscope}%
\begin{pgfscope}%
\definecolor{textcolor}{rgb}{0.000000,0.000000,0.000000}%
\pgfsetstrokecolor{textcolor}%
\pgfsetfillcolor{textcolor}%
\pgftext[x=4.679189in,y=3.585762in,left,base]{\color{textcolor}\sffamily\fontsize{10.000000}{12.000000}\selectfont n = 5}%
\end{pgfscope}%
\begin{pgfscope}%
\pgfsetrectcap%
\pgfsetroundjoin%
\pgfsetlinewidth{1.505625pt}%
\definecolor{currentstroke}{rgb}{0.839216,0.152941,0.156863}%
\pgfsetstrokecolor{currentstroke}%
\pgfsetdash{}{0pt}%
\pgfpathmoveto{\pgfqpoint{4.290301in}{3.430516in}}%
\pgfpathlineto{\pgfqpoint{4.429189in}{3.430516in}}%
\pgfpathlineto{\pgfqpoint{4.568078in}{3.430516in}}%
\pgfusepath{stroke}%
\end{pgfscope}%
\begin{pgfscope}%
\definecolor{textcolor}{rgb}{0.000000,0.000000,0.000000}%
\pgfsetstrokecolor{textcolor}%
\pgfsetfillcolor{textcolor}%
\pgftext[x=4.679189in,y=3.381905in,left,base]{\color{textcolor}\sffamily\fontsize{10.000000}{12.000000}\selectfont n = 10}%
\end{pgfscope}%
\begin{pgfscope}%
\pgfsetrectcap%
\pgfsetroundjoin%
\pgfsetlinewidth{1.505625pt}%
\definecolor{currentstroke}{rgb}{0.580392,0.403922,0.741176}%
\pgfsetstrokecolor{currentstroke}%
\pgfsetdash{}{0pt}%
\pgfpathmoveto{\pgfqpoint{4.290301in}{3.226659in}}%
\pgfpathlineto{\pgfqpoint{4.429189in}{3.226659in}}%
\pgfpathlineto{\pgfqpoint{4.568078in}{3.226659in}}%
\pgfusepath{stroke}%
\end{pgfscope}%
\begin{pgfscope}%
\definecolor{textcolor}{rgb}{0.000000,0.000000,0.000000}%
\pgfsetstrokecolor{textcolor}%
\pgfsetfillcolor{textcolor}%
\pgftext[x=4.679189in,y=3.178048in,left,base]{\color{textcolor}\sffamily\fontsize{10.000000}{12.000000}\selectfont n = 20}%
\end{pgfscope}%
\begin{pgfscope}%
\pgfsetrectcap%
\pgfsetroundjoin%
\pgfsetlinewidth{1.505625pt}%
\definecolor{currentstroke}{rgb}{0.549020,0.337255,0.294118}%
\pgfsetstrokecolor{currentstroke}%
\pgfsetdash{}{0pt}%
\pgfpathmoveto{\pgfqpoint{4.290301in}{3.022802in}}%
\pgfpathlineto{\pgfqpoint{4.429189in}{3.022802in}}%
\pgfpathlineto{\pgfqpoint{4.568078in}{3.022802in}}%
\pgfusepath{stroke}%
\end{pgfscope}%
\begin{pgfscope}%
\definecolor{textcolor}{rgb}{0.000000,0.000000,0.000000}%
\pgfsetstrokecolor{textcolor}%
\pgfsetfillcolor{textcolor}%
\pgftext[x=4.679189in,y=2.974191in,left,base]{\color{textcolor}\sffamily\fontsize{10.000000}{12.000000}\selectfont n = 40}%
\end{pgfscope}%
\begin{pgfscope}%
\pgfsetbuttcap%
\pgfsetroundjoin%
\pgfsetlinewidth{1.505625pt}%
\definecolor{currentstroke}{rgb}{0.000000,0.000000,0.000000}%
\pgfsetstrokecolor{currentstroke}%
\pgfsetdash{{5.550000pt}{2.400000pt}}{0.000000pt}%
\pgfpathmoveto{\pgfqpoint{4.290301in}{2.818945in}}%
\pgfpathlineto{\pgfqpoint{4.429189in}{2.818945in}}%
\pgfpathlineto{\pgfqpoint{4.568078in}{2.818945in}}%
\pgfusepath{stroke}%
\end{pgfscope}%
\begin{pgfscope}%
\definecolor{textcolor}{rgb}{0.000000,0.000000,0.000000}%
\pgfsetstrokecolor{textcolor}%
\pgfsetfillcolor{textcolor}%
\pgftext[x=4.679189in,y=2.770334in,left,base]{\color{textcolor}\sffamily\fontsize{10.000000}{12.000000}\selectfont Limit function}%
\end{pgfscope}%
\end{pgfpicture}%
\makeatother%
\endgroup%
}
  \vspace{-16pt}
  \caption{A sequence of uniformly converging functions}
  \label{fig:example}
\end{figure}

\proposition{}{If the sequence $(f_n)_n$ converges uniformly to $f$, then it also converges pointwise.}

\proposition{}{A sequence $(f_n)_n$ of functions $\Omega\rightarrow\mathbb{K}$ converges uniformly to $f$ if and only if the quanity $\| f-f_n \|_\infty$ is finite for n large enough and converges to 0 as $n\rightarrow\infty$.}

\theorem{Uniform convergence and continuity}{
Let $f_n:\Omega\rightarrow\mathbb{K}$ be a sequence of functions such that $f_n$ converges uniformly to a limit $f$. Let $x_0\in\Omega$. If the $f_n$ are continuous at $x_0$ then $f$ is also continuous at $x_0$.
}

\corrolary{Let $f_n:\Omega\rightarrow\mathbb{K}$ be a sequence of functions such that $f_n$ converges uniformly to a limit $f$. If the $f_n$ are continuous in $\Omega$ then $f$ is continuous in $\Omega$.}

\theorem{Uniform Cauchy criterion}{
    A sequence of functions $f_n:\Omega\rightarrow\mathbb{K}$ converges uniformly on $\Omega$ if and only if all $\varepsilon > 0$, there exists $N$ such that for all $m\geq n\geq N$ and all $x\in\Omega$ $$|f_n(x)-f_m(x)|\leq\varepsilon.$$
}

% \newpage

\definition{Normal convergence of series}{
Let $\sum f_n$ be a series of functions $\Omega\rightarrow\mathbb{K}$. We say that the series $\sum f_n$ is \textbf{normally convergent} if there exists a convergent numerical series of non negative numbers $\sum a_n$ such that $|f_n(x)|\leq a_n$ for all $n$ and all $x$.
}

\theorem{}{A series $\sum f_n$ which is normally convergent is uniformly convergent and thus, in particular, also pointwise convergent.}

\proposition{}{Let $g_n:\Omega\rightarrow\mathbb{R}$ be a sequence of functions such that $0\leq g_{n+1}(x)\leq g_n(x)$ for all $n\in\mathbb{N}$ and $x\in\Omega$. Assume that there exists a numerical sequence $(a_n)$ with $\displaystyle\lim_{n\rightarrow\infty}a_n=0$ such that $g_n(x)\leq a_n$, for every $n\in\mathbb{N}$ and $x\in\Omega$. Then the series of function $\sum (-1)^n g_n$ is uniformly convergent on $\Omega$.}

\proposition{Criteria of alternated series}{Let $(a_n)$ be a decreasing sequence of positive real numbers converging to 0. Then the series $\sum (-1)^n a_n$ is convergent. Furthermore, if we denote by $S_N=\displaystyle\sum_{n\leq N} (-1)^n a_n$ the sequence of partial sums, and by $S=\displaystyle\sum^\infty_{n=0} (-1)^n a_n$ the sum of the series, then for all $n$, $S$ is between $S_n$ and $S_{n+1}$ (meaning that $S_n\leq S\leq S_{n+1}$ or $S_{n+1}\leq S\leq S_n$ depending on the relative position of $S_n$, $S_{n+1}$) and we have $|S-S_n|\leq a_{n+1}$.}

\theorem{Exchange of limits}{
Let $f_n:\Omega\rightarrow\mathbb{K}$ be a sequence of functions that converges uniformly to a limit $f$. Let $x_0\in\bar{\Omega}$ where $\bar{\Omega}$ denotes the adherence of $\Omega$ or let $x_0=\pm\infty$ in the case where $\Omega=[a, \infty )$ or $\Omega=(-\infty , b]$. Assume that for every $n$, the limit $$\displaystyle\lim_{x\rightarrow x_0}f_n(x)$$ exists and denote it by $\alpha_n$. Then, the two limits $$\displaystyle\lim_{n\rightarrow\infty}\alpha_n \text{ and } \lim_{x\rightarrow x_0} f(x)$$ exists and we have the equality $$\displaystyle\lim_{n\rightarrow\infty}\alpha_n = \lim_{x\rightarrow x_0} f(x).$$ 
}

\subsection{Integration and derivation}

\proposition{}{Let $(f_n)_n$ be a sequence of continuous functions on a compact interval $I$ of $\mathbb{R}$. Assume that $f_n$ converges uniformly to a function $f$. Then, $$\displaystyle\int_I f_n\rightarrow\int_I f.$$}

\proposition{}{Let $a\in\mathbb{R}, b\in\mathbb{R}\cup\{+\infty\}$ and let $f_n: [a, b) \rightarrow\mathbb{R}$ be a sequence of continuous functions, all integrable on $[a,b)$. Assume that: \\
$\bullet$ the series of functions $\displaystyle\sum_n f_n$ is normally convergent on all segments $[a,u]\subset [a, b)$ \\
$\bullet$ the numerical series $\displaystyle\sum_n\int^b_a |f_n(t)|dt$ is convergent. \\
Then, the integral $\displaystyle\int_a^b\sum^\infty_{n=0}f_n(t)dt$ is convergent and we have the equality $$\displaystyle\int_a^b\left(\sum_{n=0}^\infty f_n(t)\right)dt = \sum^\infty_{n=0}\int_a^b f_n(t)dt.$$
}

\theorem{Uniform convergence and derivation}{
Let $(f_n)$ be a sequence of functions which are derivable on the segment $[a,b]$. We assume that \\
$\bullet$ there exists a point $x_0\in [a,b]$ such that the numerical sequence $(f_n(x_0))$ converges, \\
$\bullet$ the sequence of functions $(f'_n)_n$ is uniformly convergent on $[a,b].$\\
Then, $(f_n)$ converges uniformly to a function $f$ which is derivable and we have $$f'(x)=\displaystyle\lim_{n\rightarrow\infty} f'_n(x)$$ for every $x\in [a,b].$
}

\subsection{More involved topics}

\theorem{Dini's theorem}{
Let $K$ be a compact subset of $E$ and assume that \\
$(1)$ $(f_n)$ is a sequence of continuous functions $K\rightarrow\mathbb{R}$, \\
$(2)$ $(f_n)$ converges pointwise to a continous function $f$, \\
$(3)$ $(f_n)$ is decreasing in the sense that $f_n(x)\geq f_{n+1}(x)$ for every $n$ and $x\in K$ \\
Then the sequence $(f_n)$ converges uniformly to $f$ on $K$.
}

\proposition{}{Let $N: B(\Omega)\rightarrow\mathbb{R}^+$, $f\mapsto\displaystyle\sup_{x\in\Omega} |f(x)|$. $N$ is a norm on the vector space $B(\Omega)$.}

\proposition{}{A sequence of functions $(f_n)$ of $B(\Omega)$ converges uniformly to a function $f\in B(\Omega)$ if and only if $\| f_n - f \|_\infty$ tends to zero.}

\corrolary{The space of bounded continuous functions is a closed subspace of $B(\Omega)$.}

\theorem{}{$B(\Omega )$ is a complete normed vector space.}

\theorem{Weierstrass theorem}{
Let $f: [a,b]\rightarrow\mathbb{C}$ be a continuous function. There exists a sequence of polynomials which converges uniformly to $f$ on $[a,b]$. Moreover, if $f$ is real valued, one can choose polynomials $P_n$ with real coefficients.
}

\section{Chapter II: Power series}
\subsection{First definitions, examples, and reminders}

\definition{Power series}{A power series is a series of functions $\mathbb{C}\rightarrow\mathbb{C}$ of the form $\sum a_nz^n$ where $(a_n)_n$ is a sequence of complex numbers. The $a_n$ are called the coefficients of the power series.}

\proposition{\textbf{Cauchy rule} or \textbf{root test}}{Consider a series of complex numbers $\sum b_n$ and set $\beta=\lim\sup_{n\rightarrow\infty}(|b_n|)^\frac{1}{n}\in\mathbb{R}_+\cup\{+\infty\}.$ Then \\
(1) If $\beta < 1$ the series $\sum b_n$ is absolutely convergent \\
(2) If $\beta > 1$ the series $\sum b_n$ is divergent}

\proposition{\textbf{D'Alembert rule} or \textbf{ratio test}}{Let $\sum a_n$ be a numerical series\\
(1) If $\lim\sup_{n\rightarrow\infty} \left| \frac{a_{n+1}}{a_n} \right| < 1$, the series $\sum a_n$ is absolutely convergent\\
(2) If $\lim\inf_{n\rightarrow\infty}\left|\frac{a_{n+1}}{a_n}\right| > 1$, the series $\sum a_n$ is divergent.}

\theorem{}{Let $\sum a_nz^n$ be a power series. Set $\alpha=\lim\sup_{n\rightarrow\infty}|a_n|^\frac{1}{n}$ and $R=\frac{1}{\alpha}$ (with the convention that $R=+\infty$ if $\alpha=0$, and $R=0$ if $\alpha=+\infty$). Then, the series $\sum a_nz^n$ is absolutely convergent for $|z|<R$ and divergent for $|z|>R.$}

\definition{Radius of convergence}{$R$ is called the radius of convergence of the power series $\sum a_nz^n$.}

\proposition{}{Consider a power series $\sum a_nz^n$ and let $R$ be its radius of convergence. Then $R$ is the supremum of the $r\geq 0$ such that the sequence $|a_n|r^n$ is bounded.}

\theorem{}{Let $\sum a_nz^n$ be a power series and $R$ be its radius of convergence. Then, for every $\varepsilon > 0$ the series $\sum a_nz^n$ converges normally on the closed ball $\bar{B}_{R-\varepsilon}=\{z\in\mathbb{C} | |z|\leq R-\varepsilon\}.$ In particular, the sum $f(z)=\sum a_nz^n$ defines a continuous function on the open ball $B_R=\{z\in\mathbb{C} | |z|<R\}.$}

\subsection{Basic operations on power series}
\proposition{}{Let $\sum a_nz^n$ be a power series with radius of convergence $R$, and $\sum b_nz^n$ a power series with radius of convergence $R'$. Let $R''$ be the radius of convergence of the power series $\sum (a_n+b_n)z^n$. We have $R''\geq\min (R, R')$. Moreover, if $R\neq R'$ then $R''=min(R, R')$.}

\definition{Product}{The product of the numerical series $\sum u_n$ and $\sum v_n$ is the series $\sum w_n$ defined by $$w_n=u_0v_n+u_1v_{n-1}+\ldots+u_nv_0=\displaystyle\sum_{i+j=n}u_iv_j.$$}

\proposition{}{If the series $\sum u_n$ and $\sum v_n$ are absolutely convergent, then their product $\sum w_n$ is also absolutely convergent and we have $$\displaystyle\sum^\infty_{n=0}w_n=\left(\sum^\infty_{n=0}u_n\right)\left(\sum^\infty_{n=0}v_n\right).$$}

\proposition{Product of two power series}{Let $\sum a_n$ be a power series with radius of convergence $R$, and $\sum b_nz^n$ be a power series with radius of convergence $R'$. Set $$c_n=\displaystyle\sum_{i+j=n}a_ib_j$$ for every $n\geq 0$. Then, the radius of convergence $R''$ of the power series $\sum c_nz^n$ satisfies $R""\geq\min(R, R')$ and for every $|z| < \min(R, R')$ we have $$\displaystyle\sum^\infty_{n=0}c_nz^n=\left(\sum^\infty_{n=0}a_nz^n\right)\left(\sum^\infty_{n=0}b_nz^n\right)$$}

\subsection{Functions defined by a power series on $\mathbb{R}$}

\proposition{}{Let $\sum a_nz^n$ be a power series of radius of convergence $R>0$. Then, the function $$f:(-R, R)\rightarrow\mathbb{C}, x\mapsto\displaystyle\sum^\infty_{n=0}a_nx^n$$ is $C^\infty$ and for every $k\geq 0$ and $x\in(-R, R)$, we have $$f^{(k)}(x)=\displaystyle\sum^\infty_{n=k}n(n-1)\ldots(n-k+1)a_nx^{n-k}.$$}

\proposition{}{For every sequence $(a_n)_n$ of complex numbers, the power series $\sum a_nz^n$ and $\sum_{n\geq 1}na_nz^{n-1}$ have the same radius of convergence.}

\corrolary{Let $f$ be a function on $(-R, R)$ defined by a power series $\displaystyle\sum^\infty_{n=0}a_nx^n$. Then, for every $n$ we have $a_n=\frac{f^{(n)}(0)}{n!}$.}

\remark{Let $f$ be a function on $(-R, R)$ defined by a power series $\displaystyle\sum^\infty_{n=0}a_nx^n$. Then, for all $n$, the Taylor expansion of order $n$ of $f$ at $x=0$ is given by $$f(x)=a_0+a_1x+\ldots+a_nx^n+o(x^n).$$}

\proposition{}{Let $\sum a_nx^n$ be a power series with radius of convergence $R>0$ and let $f:(-R,R)\rightarrow\mathbb{C}$ be the function defined by its sum. Let $x_0\in(-R,R)$ and set $R_0=R-|x_0|$. \textit{(Note that $R_0$ is the largest radius s.t. $(x_0-R_0, x_0+R_0)\subseteq(-R, R))$}. Then, for every $x\in(x_0-R_0, x_0+R_0)$, we have $$f(x)=\displaystyle\sum^\infty_{n=0}\frac{f^{(n)(x_0)}}{n!}(x-x_0)^n.$$ In other words, $f$ can be written as a power series centered at $x_0$ on the segment $(x_0-R_0, x_0+R_0)$ with coefficients equal to those of the Taylor expansion of $f$ at $x_0$.}

\theorem{}{Let $(a_{i,j})$ be a sequence indexed by two indices $i,j\in\mathbb{N}$. Set $$b_i=\displaystyle\sum^\infty_{j=0}|a_{i,j}|\in\mathbb{R}\cup\{+\infty\}$$ for every $i\in\mathbb{N}$. Then, if the series $\sum_i b_i$ is convergent we have:
\begin{itemize}
    \item For each $j\in\mathbb{N}$, the series $\sum_ia_{i,j}$ is absolutely convergent and similarly, for each $i\in\mathbb{N}$, the series $\sum_ia_{i,j}$ is absolutely convergent
    \item The series $$\displaystyle\sum_i\sum^\infty_{j=0}a_{i,j} \text{ and }\sum_j\sum^\infty_{i=0}a_{i,j}$$ are both absolutely convergent
    \item The equality $$\displaystyle\sum_{i=0}^\infty\sum^\infty_{j=0}a_{i,j} =\sum_{j=0}^\infty\sum^\infty_{i=0}a_{i,j}$$
\end{itemize}
}

\definition{Accumulation point}{Let $E$ be a subset of $\mathbb{R}$. Then, we say that $x\in\mathbb{R}$ is an \textbf{accumulation point} of $E$ if there exists a sequence $x_n\in E$ such that the $x_n$ are all distinct and $x_n$ converges to $x$ as $x\rightarrow\infty$.}

\theorem{}{Let $\sum a_nz^n$ be a power series with radius of convergence $R>0$. If the sets of $x\in(-R, R)$ such that $\displaystyle\sum^\infty_{n=0}a_nx^n=0$ has an accumulation point in $(-R, R)$ then we have $a_n=0$ for all $n$.}

\remark{Theorem 3.8 allows to construct $C^\infty$ functions which cannot be written as power series, even on a small neighbourhood of zero.}

\subsection{Development in power series of usual functions}

\proposition{Usual identities}{For my sanity please refer to the lecture notes page 22.}

\subsection{Applications of power series}

\proposition{}{For all $u, v\in\mathbb{C}$, we have $$\exp(u+v) = \exp(u)\exp(v).$$}

\proposition{}{For all $t\in\mathbb{R}$, we have $$\exp(it)=\cos(t)+i\sin(t).$$}

\section{Fourier series}

\subsection{Introduction}

\subsection{Trigonometric series}
\definition{Trigonometric polynomial and trigonometric series}{(1) A \textbf{trigonometric polynomial} is a function of the form $x\mapsto\sum^N_{n=-N}c_ne^{ixn}$ or, equivalently, a function of the form $x\mapsto\frac{a_0}{2}+\sum^N_{n=1}a_n\cos(nx)+b_n\sin(nx)$, for some integer $N\geq 0$ and coefficients $c_n\in\mathbb{C}$ (respectively $a_n, b_n\in\mathbb{C}$. \\
(2) A \textbf{trigonometric series} is a series of functions of the form $\displaystyle c_0+\sum_{n\geq 1}c_ne^{inx}+c_{-n}e^{-inx}$ or, equivalently, a series of functions of the form $\displaystyle\frac{a_0}{2}+\sum_{n\geq 1}(a_n\cos(nx)+b_n\sin(nx)$ for some coefficients $c_n\in\mathbb{C}$ (respectively $a_n, b_n\in\mathbb{C}$). In the former case, we will, for brevity, also denote such a series by $\displaystyle\sum_{n\in\mathbb{Z}}c_ne^{inx}$.
}

\remark{The series $\displaystyle\sum_{n\in\mathbb{Z}c_ne^{inx}}$ is pointwise (resp. uniformly) convergent if and only if the sequence of partial sums $\displaystyle\sum^N_{n=-N}c_ne^{inx}$ is pointwise (resp. uniformly) convergent.}

\proposition{}{If the series $\sum_{n\in\mathbb{N}}|c_n|$ and $\sum_{n\in\mathbb{N}}|c_{-n}|$ are both convergent, then the trigonometric series $\sum_{n\in\mathbb{Z}}c_ne^{inx}$ is normally convergent on $\mathbb{R}$ and the function $x\in\mathbb{R}\mapsto\sum^\infty_{n=-\infty}c_ne^{inx}$ is continuous and $2\pi$-periodic.}

\proposition{}{If the sequences $(c_n)_{n\in\mathbb{N}}$ and $(c_{-n})_{n\in\mathbb{N}}$ are real, decreasing, and tend to 0 at infinity, then the trigonometric series $\sum_{n\in\mathbb{Z}}c_ne^{inx}$ converges pointwise on $\mathbb{R} \backslash 2\pi\mathbb{Z}$, and uniformly on every interval of the form $[2k\pi+\alpha, 2(k+1)\pi-\alpha]$ where $k\in\mathbb{Z}$ and $0<\alpha<2\pi$.}

\proposition{Lemma}{Let $f_n$ and $g_n$ be two sequences f functions $\Omega\rightarrow\mathbb{K}$ such that 
\begin{itemize}
    \item The partial sums $\displaystyle\sum^N_{n=1}f_n$ are uiformly bounded,
    \item The sequence $(g_n)_n$ converges uniformly to zero on $\Omega$
    \item For all $x\in\Omega$, the sequence $(g_n(x))_n$ is decreasing
\end{itemize}
Then the series $\sum f_ng_n$ converges uniformly on $\Omega$.}

\definition{Piecewise function}{Let $m\in\N\cup\{\infty\}$. A function $f:[a,b]\rightarrow\C$ is \textbf{piecewise} $C^m$ if there exist an integer $k\geq 1$ and a sequence $a=x_0<x_1<\ldots<x_k=b$ such that for every $0\leq i\leq k-1$, the restriction of $f$ to $(x_i,x_{i+1})$ is $C^m$ and can be extended by continuity to a $C^m$ function on $[x_i,x_{i+1}].$\\

A function $f:\R\rightarrow\C$ is \textbf{piecewise} $C^m$ on every segment $[a,b]$.\\

Piecewise $C^0$ functions will also be called \textbf{piecewise continuous} functions.}

\definition{Fourier coefficient}{Let $f:\R\rightarrow\R$ be a piecewise continuous function which is $2\pi$-periodic. We define the \textbf{Fourier coefficients} of $f$ by $$\displaystyle c_n(f)=\frac{1}{2\pi}\int^{2\pi}_0 e^{-inx}f(x)dx\text{, for } n\in\Z$$ and $$a_n(f)=\frac{1}{\pi}\int^{2\pi}_0\cos(nx)f(x)dx\text{, for }n\in\N, \quad b_n(f)=\frac{1}{\pi}\int^{2\pi}_0\sin(nx)f(x)dx\text{, for }n\in\N^*.$$
The \textbf{Fourier series} associated to $f$ is the trigonometric series $\displaystyle\sum_{n\in\Z}c_n(f)e^{inx}$, or equivalently $\frac{a_0(f)}{2}+\sum_{n\geq 1}a_n(f)\cos(nx)+\sum_{n\geq 1}b_n(f)\sin(nx).$}

\remark{If $g:\R\rightarrow\C$ is piecewise continuous and $2\pi$-periodic then for every $a\in\R$ we have $\int^{a+2\pi}_ag(x)dx=\int^{2\pi}_0g(x)dx$. In particular, we can compute the Fourier coefficients of a ($2\pi$-periodic and piecewise continuous) function $f:\R\rightarrow\C$ by taking integrals over an arbitrary interval of length $2\pi$ that is to say: for every $a\in\R$ we have $$c_n(f)=\frac{1}{2\pi}\int^{a+2\pi}_a e^{-inx}f(x)dx\text{, for }n\in\Z,$$ and $$a_n(f)=\frac{1}{\pi}\int^{a+2\pi}_a\cos(nx)f(x)dx\text{, for }n\in\N, \quad b_n(f)=\frac{1}{\pi}\int^{a+2\pi}_a\sin(nx)f(x)dx\text{, for }n\in\N^*.$$}

\proposition{}{Let $f$ be a trigonometric polynomial. Then, there exists $N>0$ such that $c_n(f)=0$ for $|n|>N$ and we have $f(x)=\sum c_n(f)e^{inx}$ for all $x\in\R$.}

\proposition{}{Let $\sum_{n\in\Z}c_ne^{inx}$ be a trigonometric series that converges uniformly on $\R$ and let $f(x)=\displaystyle\sum^\infty_{n=-\infty}c_ne^{inx}$ be its sum. Then, the Fourier series of $f$ is the same trigonometric series $\sum_{n\in\Z}c_ne^{inx}$ (i.e. $c_n(f)=c_n$ for all $n$) and therefore $f$ is equal to its Fourier series everywhere.}

\proposition{}{The Fourier coefficients satisfy the following properties:
\begin{enumerate}
    \item If $f$ is real valued, then for all $n\in\N, a_n(f), b_n(f)\in\R$ and $c_{-n}(f)=\overline{c_n(f)}$.
    \item If $f$ is even, then for all $n\in\N, b_n(f)=0$ and $c_n(f)=c_{-n}(f)$.
    \item If $f$ is odd, then for all $n\in\N, a_n(f)=0$ and $c_n(f)=-c_{-n}(f).$
\end{enumerate}}

\remark{If $f:\R\rightarrow\C$ is $T$ periodic, we can alsodefine Fourier coefficients by $c^T_n(f)=\frac{1}{T}\int^T_0e^{-i\frac{2\pi}{T}nx}f(x)dx$, and the Fourier series associated to $f$ by $\sum_n c_n^T(f)e^{i\frac{2\pi}{T}nx}.$}

\theorem{Fejer's theorem}{Let $f:\R\rightarrow\C$ be a continous $2\pi$ periodic function. Then, there exists a sequence $(C_n)_n$ of trigonometric polynomials such that $(C_n)_n$ converges to $f$ uniformly.}

\subsection{Prehilbertian structure}
\proposition{}{$\inner{\cdot}{\cdot}$ is an inner product on $D$. More precisely we have that:
\begin{enumerate}
    \item It is sesquilinear (i.e antilinear in the first variable and linear in the 2nd variable)\\ $\inner{f+\lambda h}{g}=\inner{f}{g}+\bar{\lambda}\inner{h}{g}$ for all $f,g,h\in D$ and $\lambda\in\C$\\
    $\inner{f}{g+\lambda h}=\inner{f}{g}+\lambda\inner{f}{h}$ for all $f,g,h\in D$ and $\lambda\in\C$,
    \item It is Herimitian $\inner{f}{g}=\overline{\inner{g}{f}}$ for all $f,g\in D$,
    \item It is positive $\inner{f}{f}\geq 0$ for all $f\in D$
    \item It is definite $\inner{f}{f}=0\Rightarrow f=0$ $\forall f\in D$
\end{enumerate}}

\proposition{}{Let $n\in\N$ and set $P_n=Vect(e^{ikx}, -n\leq k \leq n)$. Then, $P_n$ is a vector subspace of $D$ and we have $D=P_n\bigoplus P_n^\perp$ where $P_n^\perp$ denotes the orthogonal of $P_n$ for $\inner{\cdot}{\cdot}.$ Moreover, if we denote by $p_n$ the orthogonal projection onto $P_n$, we have $$p_n(f)=\sum c_k(f)e^{ikx},$$ and $$\inf_{g\in P_n}\|f-g\|^2=\|f-p_n(f)\|^2=\frac{1}{2\pi}$$} 
\end{document}
