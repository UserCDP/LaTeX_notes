\documentclass{article}
\usepackage{graphicx} % Required for inserting images
\usepackage[margin=2cm]{geometry}
\usepackage{cancel}
\usepackage{amsmath}
\usepackage{amssymb}
\usepackage{fontawesome5}
\usepackage{mathtools}

\title{MAA201 Additions}
\author{daniela.cojocaru}
\date{January 2024}

\begin{document}

\maketitle

\section{LU decomposition}
(Gaussian elimination)

Linear system
$\left\{\begin{array}{cc}
    a_{11} x_1 + a_{12} x_2 + \ldots + a_{1N}x_N = y_1 & (1)  \\
    a_{21} x_1 + a_{22} x_2 + \ldots + a_{2N}x_N = y_2 & (2) \\
    \multicolumn{2}{l}{\quad\vdots} \\
    \cancel{a_{k1}}x_1 \qquad\text{ (cancel the first coeff)} & (k) \\ %\qquad\qquad\qquad\qquad\qquad\qquad + a_{k2} x_2 + \ldots + a_{kN}x_N = y_k & (k) \\
    \multicolumn{2}{l}{\quad\vdots} \\
    a_{N1}x_1 + a_{N2} x_2 + \ldots + a_{NN}x_N = y_N & (N)
\end{array}, \text{ if lucky } a_{11} \neq 0 \right.$ \\

\vspace{10pt}

$(k) \rightarrow (k) - \displaystyle\frac{a_{k1}}{a_{11}}$ etc.

\vspace{10pt}

Triangular system:
$\left\{\begin{array}{cccc}
    a_{11}x_1 + &a_{12}x_2 + &\ldots &+ a_{1N}x_N = y_1 \\
    \left| \texttt{No} \right| &a_{22}x_2 + &\ldots &+ a_{2N}x_N = y_2 \\
    \left| x_1\right|  &\left|\texttt{No}\right|  &\ldots \\
    \left|...\right|  &\left|x_2\right| &\ldots &+ a_{NN}x_N = y_N
\end{array}\right.$

\vspace{10pt}

Backward substitution
$\rightarrow$ Solve $(x_N, x_{N-1},\ldots,x_2, x_1)$ in terns of matrices

\vspace{10pt}

M = 
$\left(\begin{array}{ccc}
    a_{11} & \ldots & a_{1N} \\
    \vdots & & \vdots \\
    a_{N1} & \ldots & a_{NN}
\end{array}\right)$
=
$\left(\begin{array}{c}
    L_1 \\
    \vdots \\
    L_N \\
\end{array}\right)$

\vspace{10pt}

Sheer matrices transvections
$T^\lambda_{jk} = \lambda E_{ij} + id$

$L_k \rightarrow L_k - \lambda L_j \longleftrightarrow$ Multiplying $M$ to the left by $T^\lambda_{kj}$

$M\rightarrow T^\lambda_{kj}M$

In Gaussian elimination, all $T^\lambda_{jk}$ used are lower triangular.

$\begin{array}{cc}
    & \text{upper-triang} \\
    M\rightarrow T^\lambda_{jk}M\rightarrow T_2 T_1 M\rightarrow & \overbrace{\underbrace{T_k\ldots T_2 T_1} M}\\
     & \text{a lower triangular} \\
     & \text{matrix \& invertible}
\end{array}$

\vspace{10pt}

$\rightarrow M = \underbrace{T^{-1}_1 T^{-1}_2\ldots T_k^{-1}}_{\text{Lower-triang. } L} U$ \\
$\begin{array}{cc}
     M = L U \\
     \qquad\nearrow\quad \nwarrow \\
     \text{lower tr.} \quad \text{upper tr.}
\end{array}$

\vspace{10pt}

\underline{\textbf{What if No Luck?\faIcon[regular]{frown}}}

$M$ non invertible $\rightarrow$ OTHER STORY

$M$ invertible $\rightarrow$ switch rows

\vspace{2pt}

$\sigma$: permutation of rows

$M\rightarrow P_\sigma M$

$M\displaystyle\xrightarrow[\text{switch rows}]{\text{step 0}} P_\sigma M \xrightarrow[]{\text{steps 1-k}} \overbrace{T_k - T_1}^{\text{Lower-}\triangle}(P_\sigma M) = U \text{ upper-}\triangle$

\vspace{6pt}

$M=P^{-1}_\sigma (\underbrace{T_1^{-1}\ldots T_k^{-1}}_{\text{lower-}\triangle})U$

\vspace{10pt}

\textbf{\underline{Theorem}}: Every invertible $M\in\mathcal{M}_N(\mathbb{K})$ has a PLU decomposiiton.

\centerline{$M=P\times L\times U$}
\centerline{$\hspace{-18pt}\prescript{}{\text{permutation}}{\nearrow}\quad\qquad\hspace{-10pt}\nwarrow_\triangle \nwarrow_\triangledown$}

\vspace{20pt}

\underline{\textbf{EXAMPLE}}

$\left(\begin{array}{ccc}
    1 & 2 & 3 \\
    4 & 5 & 6 \\
    7 & 8 & 11
\end{array}\right)\in GL_3(\mathbb{R})$
\hspace{10pt}$\begin{array}{c}L_2\leftarrow L_2 - 4 L_1 \\
L_3\leftarrow L_3 - 7 L_1 \\
L_3\leftarrow L_3 - 2 L_2
\end{array}$

$U=\left(\begin{array}{ccc}
    1 & 2 & 3 \\
    0 & -3 & -6 \\
    0 & 0 & 2
\end{array}\right)$ \\

\noindent Until now: just Gaussian elimination \\
Now: want to $M=LU$ \\

\noindent$T_{kl}(\lambda): I_3 + \lambda E_{kl}$ \\
Act on rows $\leftrightarrow$ Multiply by $T_{kl}(\lambda)$'s to the left

\centerline{\underline{$T_{kl}(\lambda)^{-1}=T_{kl}(-\lambda)$}}


\noindent\texttt{Step 1}\hspace{20pt} $M_1=T_{21}(-4)M$ \\
\texttt{Step 2}\hspace{20pt} $M_2=T_{31}(-7)T_{24}(-4)M$ \\
\texttt{Step 3}\hspace{20pt} $U=M_3=T_{32}(-2)T_{31}(-7)T_{21}(-4)M$ \\

\centerline{$M=\underbrace{T_{21}(-4)^{-1}T_{31}(-7)^{-1}T_{32}(-2)^{-1}}_L U=\underbrace{T_{21}(4)T_{31}(7)T_{32}(2)}_L U$}

$T_{31}(7)T_{32}(2) = \left(\begin{array}{ccc}
    1 & 0 & 0 \\
    0 & 1 & 0 \\
    0 & 2 & 1
\end{array}\right)$

$L_3\leftarrow L_3 + 7 L_1$

$T_{21}(4)...$

$L_2\leftarrow L_2 + 4 L_1$

$T_{21}(4) T_{31}(7) T_{32}(2)=
\left(\begin{array}{ccc}
    1 & 0 & 0 \\
    4 & 1 & 0 \\
    7 & 2 & 0
\end{array}\right)$

\underline{Reality check \faIcon[regular]{umbrella}}

$M = LU =
\begin{pmatrix}
    1 & 0 & 0 \\
    4 & 1 & 0 \\
    7 & 2 & 1
\end{pmatrix}
\begin{pmatrix}
    1 & 2 & 3 \\
    0 & -3 & -6 \\
    0 & 0 & 2
\end{pmatrix}
$

\section{Diagram chase}

$A\hookrightarrow B$ $f$ injective \\

\noindent$A\twoheadrightarrow B$ $f$ surjective \\

\hspace{-16pt}$\begin{array}{c}
    A \longrightarrow^{f_1} B \\
    g_1\big\downarrow \hspace{10pt} \circlearrowright \hspace{10pt}\big\downarrow f_2 \\
    C \longrightarrow_{g_2} D
\end{array}$ the diagram commutes

\section{Kernel lemma}
$\ker P(u) \bigoplus \ker Q(u)= \ker PQ(u)$

\noindent$Sp(u) \equiv$ roots of $P_u$

\section{Cramer's rule}
$A\in \mathcal{M}_N(\mathbb{K})$

\noindent$(Com(A))_{ij}=(-1)^{1+j}\det(A_{\hat{i}\hat{j}})\longleftarrow$remove row $i$ and column $j$\\

\noindent\underline{\textbf{Theorem}}: $A\times Com(A)^t = det(A)\times I_n$

\noindent\underline{Corrolary}: If $A\in GL_n(\mathbb{K}), det(A)\neq 0, so A^{-1}=\frac{1}{det(A)}Com(A)^t$

\section{3 ways to invert a matrix}
    \subsection{Cramer's rule}
    $A\in \mathcal{M}_n(\mathbb{K})$ (all matrices)
    [explained above]

    \subsection{Schur Complements ("Block" and LU decomposition)}
    $\begin{array}{c}
         k\updownarrow \\
         l\updownarrow
    \end{array}\begin{pmatrix}
        A_{(k\times k)} & B \\
        C & D_{(l\times l)} \\
    \end{pmatrix}$

    $\hspace{20pt}\begin{array}{cc}
        \longleftrightarrow_k & \longleftrightarrow_l 
    \end{array}$

    Assume that $A\in GL_k(\mathbb{K})$\\

    $\begin{pmatrix}
        I_k & 0 \\
        -CA^{-1} & I_k
    \end{pmatrix}
    \begin{pmatrix}
        A & B \\
        C & D
    \end{pmatrix} = 
    \begin{pmatrix}
        A & B \\
        0 & D-CA^{-1}B
    \end{pmatrix}$\\

    $D-CA^{-1}B$ - Schur complement

    Let's assume that $\begin{pmatrix}
        A & B \\
        C & D
    \end{pmatrix}\in GL_n(\mathbb{K}) \Leftrightarrow D-CA^{-1}B\in GL_l$ \\

    $\begin{pmatrix}
        A & B \\
        C & D
    \end{pmatrix}^{-1} = 
    \begin{pmatrix}
        A & B \\
        0 & D-CA^{-1}B
    \end{pmatrix}^{-1}
    \begin{pmatrix}
        I_k & 0 \\
        -CA^{-1} & I_l
    \end{pmatrix} =
    \begin{pmatrix}
        \bigstar & \bigstar \\
        \bigstar & (D-CA^{-1}B)^{-1}
    \end{pmatrix}$\\


    \underline{\textbf{Theorem}}:
    If $\begin{pmatrix}
        A & B \\
        C & D
    \end{pmatrix}\in GL_n(k)$, and $A\in GL_k (\mathbb{K})$, then the Schur complement $D-CA^{-1}B\in GL_l$ and 
    $\begin{pmatrix}
        A & B \\
        C & D
    \end{pmatrix}^{-1} =
    \begin{pmatrix}
        \bigstar & \bigstar \\
        \bigstar & (D-CA^{-1}B)^{-1}
    \end{pmatrix}$

    \subsection{Diagonalization}
    $\begin{pmatrix}
        \lambda_1 & \ldots\ldots & 0 \\
        \vdots & 0 \ddots 0 & \vdots \\
        0 & \ldots\ldots & \lambda_n
    \end{pmatrix}^{-1}=
    \begin{pmatrix}
        \lambda_1^{-1} & \ldots\ldots & 0 \\
        \vdots & 0 \ddots 0 & \vdots \\
        0 & \ldots\ldots & \lambda_n^{-1}
    \end{pmatrix}$

    Diagonisable: $M=P\triangle P^{-1}$, $\triangle$ chanage of basis \\

    $    M^{-1} = P\triangle^{-1} P^{-1}
          = P\begin{pmatrix}
             \lambda_1^{-1} & \ldots\ldots & 0 \\
        \vdots & 0 \ddots 0 & \vdots \\
        0 & \ldots\ldots & \lambda_n^{-1}
         \end{pmatrix} P^{-1}
    $\\
    
    $M^{k} = P\triangle^k P^{-1} = P\begin{pmatrix}
             \lambda_1^k & \ldots\ldots & 0 \\
        \vdots & 0 \ddots 0 & \vdots \\
        0 & \ldots\ldots & \lambda_n^k
         \end{pmatrix} P^{-1}$

\section{Sequences}
\hspace{10pt}$u_{n+2}=u_{n+1}+u_n$

$u_0=0, u_1 = 1$

Introduce $v_n=\begin{pmatrix}
     u_{n+1} \\
     u_n
\end{pmatrix}, \begin{pmatrix}
     u_{u+2} \\
        u_{n+1}
\end{pmatrix}=\begin{pmatrix}
    u_n+u_{n+1} \\
    u_{n+1}
\end{pmatrix} = \begin{pmatrix}
    1 & 1 \\
    1 & 0
\end{pmatrix}\begin{pmatrix}
    u_{n+1} \\
    u_n
\end{pmatrix}$ \\

$v_n=A^n v_0, v_0=\begin{pmatrix}
    1\\
    0
\end{pmatrix}$

Compute $A^n$?

$\rightarrow$ Diagonalize

$\chi_A(X)=det(A-XI_2) = \begin{pmatrix}
    1-X & 1 \\
    1 & -X
\end{pmatrix}=(1-X)(-X)-1 = X^2-X-1 = (X-\varphi)(X-\overline{\varphi})$ \\

\noindent$\Rightarrow$ A is diagonizable $\Leftrightarrow$ $P_A=(X-\varphi)(X-\overline{\varphi})$ \\
$\Leftrightarrow E_\varphi \oplus E_{\overline{\varphi}} = \mathbb{R}^2$

$\longrightarrow$ Now, find $P\in GL_2(\mathbb{R})$ such that $A=P\begin{pmatrix}
    \varphi & 0 \\
    0 & \overline{\varphi}
\end{pmatrix}P^{-1}$

Switch from the canonical basis $(e_1=\begin{pmatrix} 1 \\ 0 \end{pmatrix}, e_2=\begin{pmatrix}0 \\ 1 \end{pmatrix})$ to a basis of eigenvectors

$P^{-1}=\begin{pmatrix}
    1 & -\overline{\varphi} \\
    -1 & \varphi
\end{pmatrix}$

$A^k=P\begin{pmatrix}
    \varphi^k & 0 \\
    0 & \overline{\varphi}^k
\end{pmatrix}P^{-1}$

We get $u_n=\frac{1}{\sqrt{5}}(\varphi^n-\overline{\varphi}^n)$


\section{Field expansion}
\underline{\textbf{Theorem}}: For all fields $\mathbb{K}$, $P\in\mathbb{K}[X]$ irreductible, there exists a field $\mathbb{L}$ $(\mathbb{K}$ subfield of $\mathbb{L})$ such that $P (\in\mathbb{L}[X])$ has a root in $\mathbb{L}.$ (principal ideal domain: every ideal is of the form $(P(X))=P(X)\cdot\mathbb{K}[X]$\\

\noindent\textbf{FACT}: For any field $\mathbb{K}$, there exists $\overline{\mathbb{K}}$ "algebraic closure" such that every polynomial splits in $\overline{\mathbb{K}}$ \\

\noindent\underline{\textbf{Definition}}: "generalized eigenspaces"

$\lambda_i\in Sp(M), v_{\lambda_i}=Ker((M-\lambda_i I_n)^{m_i})$

\section{Remarks on triangular matrices}
(1) Eigenvalues of $T\equiv\{T_{ii}, i=1...n\}$

\noindent(2) If $T_{ii}=0, \forall i=1...n$ then $T^n=O_{nn}$, nilpotent with index $\leq n$\\

\underline{\textbf{Theorem (Cayley-Hamilton)}}: $\chi_M(M) = 0_{n\times n}$
    
\end{document}
